\chapter{Polia revealed}

Hovoril som ti, že pole nie je premenná, ale veľa premenných, a preto sa
správa zvláštne (napr. si nesmel posielať pole ako parameter do funkcie,
nesmel si priraďovať a pod.). V skutočnosti je pole konštantný\footnote{%
S označením \prg!const! sme sa už stretli v tvare \prg!const int n = 100;!: 
ak ho napíšeš pred definíciu premennej,
znamená to, že premenná sa už potom nesmie meniť. 
Podobne môže byť \prg!const int *a = &x;!
}pointer.
Keď napíšeš \prg!int a[10];! vyrobí sa v pamäti 10 premenných typu \prg!int!
a \vb{a} je \prg!const! pointer na prvú z nich. Označenie indexu \vb{[ ]}
je v skutočnosti iba skratka za výraz s hviezdičkou: \prg!a[i]! znamená
\prg!*(a+i)! pre hocijaký pointer \vb{a}. Rôzne spôsoby prístupu k
prvkom poľa sú v nasledujúcom programe:

\vbox{
\begin{lstlisting}[] 
#include <iostream>
using namespace std;

int main() {
  int a[10];
  int *b, *c;
  int i;
  a = b; // <-- !! chyba, a je const
  c = a + 1;  // c ukazuje na a[1]

  b = a;  // b ukazuje na a[0]
  for (i = 0; i < 10; i++) {
    *b = i + 1;  // *b zapisuje do poľa a
    b = b + 1;   // presuň b na ďalší prvok poľa
  }

  cout << a[0] << endl;
  for (i = 0; i < 9; i++)
    cout << c[i] << endl;  // c ukazuje na a[1], preto
                           // c[i] je a[i+1]

  *(a + 5) = 14;                // *(a+5) je a[5]
  for (b = a; b < a + 10; b++)  // k b sa dá pripočítavať číslo aj takto
    cout << *b << " ";
  cout << endl;
}
\end{lstlisting}
}


Teraz by malo byť jasné, ako sa dá posielať pole ako parameter do funkcie: stačí,
aby bol typu pointer. Treba si ale vždy poslať aj veľkosť poľa ako samostatný parameter:

\begin{lstlisting}[] 
void napln_pole(int n, int *a) {
  int i;
  for (i = 0; i < n; i++)
    a[i] = i * i;
}

int main() {
  int a[10];
  napln_pole(10, a);
  int i;
  for (i = 0; i < 10; i++)
    cout << a[i] << " ";
  cout << endl;
}
\end{lstlisting}

Tento program vypíše \vb{0 1 4 9 16 25 36 49 64 81}. Len pripomeniem: ak by si v tomto
programe zmenil riadok \prg!int a[10];! na \prg!int a[3];!, program by fungoval rovnako,
akurát, že vo funkcii \vb{napln\_pole} by sa zapisovalo aj do pamäte, ktorá nebola
rezervovaná poľom \vb{a}. Ak sú tam nejaké premenné tvojho programu (napr. \vb{i}), 
prepíšu sa. Ak tá
pamäť tvojmu programu nepatrí, príde náš priateľ \vb{segfault}.



Dvojrozmerné pole je pole, ktorého prvkami sú polia. Preto ak máš \prg!int a[2][3]!,
tak \vb{a} je (konštantný) pointer na trojprvkové pole typu \prg!int!. Preto 
\prg!*(a+1)! je to isté, čo \prg!a[1]!, čiže trojprvkové pole typu \prg!int!
(t.j. môžeš napísať \prg!int *q = *(a+1);! a potom \prg!q! bude ukazovať na
prvý prvok poľa \prg!a[1]!, čiže na druhý riadok poľa \prg!a!).
Z toho celého vyplýva, že dvojrozmerné pole je v pamäti vlastne jednorozmerné
pole uložené rovnako, ako sme to mali v kapitole~\ref{sect:2dpolia}.
Na to, aby si k nemu tak mohol pristupovať, treba ho pretypovať, takže môžeš
napísať \prg!int *p = (int *)a;!. To je zároveň aj najjednoduchší spôsob, ako
posielať dvojrozmerné pole ako parameter do funkcie.\\

\vbox{
\begin{lstlisting}[] 
int main() {
  int a[2][3];
  int i, j, k = 1;

  for (i = 0; i < 2; i++)      // naplníme pole po riadkoch 
    for (j = 0; j < 3; j++) {
      a[i][j] = k;
      k++;
    }

  int *q = *(a + 1);           // q ukazuje na druhý riadok
  for (int i = 0; i < 3; i++) 
    cout << q[i] << endl;

  int *p = (int *)a;          // pohľad na a ako na 1D pole 
  for (i = 0; i < 6; i++) 
    cout << p[i] << " ";
  cout << endl;
} 
\end{lstlisting}
}


 Malá odbočka: mechanizmus posielania parametrov pointrami sa používa aj v iných 
programovacích jazykoch, ale častokrát ``pod kapotou''. Napríklad Python posiela
základné typy ako parametre (tzv. {\em volanie hodnotou}), ale na zoznamy a iné zložité
typy posiela pointer (tzv. {\em volanie referenciou}). Pythonovský program\\

\vbox{
\begin{lstlisting}[language=python] 
def zelena(x):
    x = 42

def modra(x):
    x[0] = 42

a = 1
zelena(a)
print(a)

b = [1, 2, 3]
modra(b)
print(b)
\end{lstlisting}
}

vypíše

\begin{outputBox}
1
[42, 2, 3]
\end{outputBox}

Do funkcie \vb{zelena} sa poslalo číslo, preto sa menila hodnota lokálnej premennej
vo svete funkcie \vb{zelena} (ktorá spolu s jej svetom zanikla), do funkcie \vb{modra}
sa poslal pointer na zoznam \vb{b}. Tento spôsob volania dáva zmysel, lebo väčšinou je to
presne to, čo chceš, ale pre začínajúcich programátorov to niekedy môže byť mätúce.
Koniec malej odbočky.

 Vyriešili sme, ako posielať polia ako parametre, ale stále nám ostáva problém 
s globálnymi poľami. Pamätáš sa, že keď sme pri kreslení obrázkov chceli mať globálne
pole, museli sme ho vyrobiť tak, aby už v čase kompilácie bola známa jeho veľkosť. 
Riešili sme to tak, že sme pole spravili dostatočne veľké a používali z neho iba 
tak veľký kus, ako sme potrebovali. Toto evidentne nie je najlepší prístup.\indexItem{Prg}{dynamická alokácia: \vb{new[]} a \vb{delete[]}} 
Lepšie riešenie je tzv. {\em dynamická alokácia}: príkaz \prg!new int[100];! 
vyhradí v odľahlom kúte pamäte miesto na 100 premenných typu \prg!int! a vráti pointer
naň. Toto miesto je vyhradené pre tvoj program, ale vrátený pointer je jediný spôsob,
ako sa k tej pamäti dostať: ak si ho neuložíš do premennej (alebo ju neskôr zabudneš),
pamäť stále ostane vyhradená pre tvoj program, ale ten sa k nej už nikdy nebude môcť dostať.
Ak už alokovanú pamäť nepotrebuješ, treba ju vrátiť systému príkazom \prg!delete[]!. Treba
byť opatrný, lebo to, či je pamäť vyhradená, alebo nie, nijak nevidno. Dobrý zvyk 
je hneď po volaní \prg!delete[]! priradiť do odalokovanej premennej \prg!nullptr!.\\

\vbox{
\begin{lstlisting}[] 
#include <iostream>
using namespace std;

int *a = (int *)12345;

void rob_nieco() {
  int i;
  for (i = 0; i < 100; i++) a[i] = 42;
}   

void vypis(int x) { cout << x << ": " << (unsigned long)a << endl; }
  
int main() {
  vypis(1);
  a = new int[100];
  vypis(2);     // a ukazuje na pridelenú pamäť
  rob_nieco();  // v poriadku, píšem do svojej pamäte
  delete[] a;   // pridelená pamäť sa označila ako voľná
  vypis(3);     // hodnota a sa nezmenila, stále ukazuje na to isté miesto,
                // ale tá pamäť mi už nepatrí
  rob_nieco();  // <-- !! toto by bol problém - systém medzitým mohol 
                // tú pamäť dať niekomu inému
}
\end{lstlisting}
}

Tento program vypíše napr.

\begin{outputBox}
1: 12345
2: 94751025169088
3: 94751025169088
\end{outputBox}

 Pri priraďovaní si treba dávať pozor, 
hlavne ak je pointer na dynamicky alokovanú pamäť súčasť zloženého typu, napr.

\vskip 2ex
\vbox{
\begin{lstlisting}[] 
struct Kvak {
  int *a, b, c[2];
};
\end{lstlisting}
}

Ak napíšeš

\begin{lstlisting}[] 
Kvak x, y;
x.a = new int[100];
y.a = new int[100];
\end{lstlisting}

a ešte ponastavuješ ďalšie hodnoty, môže to v pamäti vyzerať takto:


\def\var(#1)#2#3{%
  \draw (0.2,-#1) rectangle node[align=center](v#1) {\vb{#3}} (1.2,1-#1);
  \draw (0,0.5-#1) node[anchor=east]{\textcolor{magenta}{\vb{#2}}};
}

\def\ptr(#1,#2)[#3] {
  \draw[blue,-{>[length=4pt,width=2.5pt]}] (1.7,0.5-#1) to[out=#3,in=-#3] (1.7,0.5-#2);
}

\def\world(#1)#2#3{
  \draw[thick,#3](-1,-#1) -- (2,-#1) node[anchor=west]{\vb{#2}};
}

\def\kon(#1){ 
  \draw[draw=none] (0.2,-#1) rectangle node[align=center]{\vb{\ldots}} (1.2,1-#1);
}
  
\def\godot#1{\filldraw[blue,scale=0.6](#1) circle (0.4ex and 1.4ex);}

\begin{tikzpicture}[yscale=0.4,xscale=1.4]

  \var(0){x.a}{}
  \var(1){x.b}{1}
  \var(2){x.c[0]}{2}
  \var(3){x.c[1]}{3}
  \var(4){y.a}{}
  \var(5){y.b}{4}
  \var(6){y.c[0]}{5}
  \var(7){y.c[1]}{6}
  \world(-1){x}{orange}
  \world(3){y}{olive}
  \kon(8)

  \coordinate (a1) at (3,-0.5);
  \coordinate (a2) at (3,-4.5);

  \foreach \f/\t/\l in {v0/a1/{},v4/a2/{inde\ }}{
  \godot{\f}
  \draw[blue,-{>[length=8pt,width=5pt]}] (\f) to [out=0,in=180] (\t);
  \draw (\t)++(0,-0.5) rectangle node[align=center]{400 bytov niekde \l v pamäti} ++(5,1);
  }
\end{tikzpicture}

keď potom napíšeš \prg!y=x!, stane sa toto:

\begin{tikzpicture}[yscale=0.4,xscale=1.4]

  \var(0){x.a}{}
  \var(1){x.b}{1}
  \var(2){x.c[0]}{2}
  \var(3){x.c[1]}{3}
  \var(4){y.a}{}
  \var(5){y.b}{1}
  \var(6){y.c[0]}{2}
  \var(7){y.c[1]}{3}
  \world(-1){x}{orange}
  \world(3){y}{olive}
  \kon(8)

  \coordinate (a1) at (3,-0.5);
  \coordinate (a2) at (3,-4.5);

  \foreach \f/\t/\l in {v0/a1/{},v4/a2/{inde\ }}{
  \godot{\f}
  \draw (\t)++(0,-0.5) rectangle node[align=center]{400 bytov niekde \l v pamäti} ++(5,1);
  }

  \draw[blue,-{>[length=8pt,width=5pt]}] (v0) to [out=0,in=180] (3,-0.35);
  \draw[blue,-{>[length=8pt,width=5pt]}] (v4) to [out=0,in=180] (3,-0.65);
\end{tikzpicture}

Navždy si stratil 400 bytov pamäte a navyše ak napíšeš \prg!delete[] x.a; delete[] y.a;!
program zhavaruje s chybou

\begin{outputBox}
free(): double free detected in tcache 2
Aborted
\end{outputBox}

lebo si dvakrát odalokovával tú istú pamäť (aj keď si sa k nej dostal z iných pointrov).

\begin{uloha}
  Npaíš funkciu, kotrá ako parametre dostane číslo $n$ a pointer na  pole $n$
  čísel a toto pole v pamäti otočí. Napr. ak pred volaním bolo $n=5$
  a v poli boli čísla \vb{1 2 3 4 5} po volaní tam bude \vb{5 4 3 2 1}.
\end{uloha}

\begin{uloha}
  \label{uloha:arraycopy}
  Napíš funkciu, ktorá ako parametre dostane číslo $n$ a pointer na pole $n$
  čísel a toto pole skopíruje (t.j. vráti pointer na iné pole, v ktorom budú
  tie isté prvky).
\end{uloha}

\begin{uloha}
  Napíš funkciu, ktorá ako parametre dostane číslo $n$, pointer na  pole $n$
  čísel a číslo $k$ a pole v pamäti cyklicky posunie o $k$ miest.
  Napr. ak $n=5$, $k=2$ a pole bolo \vb{1 2 3 4 5}, tak po volaní funkcie
  pole bude \vb{3 4 5 1 2}.
\end{uloha}

\begin{uloha}
  \label{uloha:primetest}
  Napíš funkciu, ktorá dostane parameter $n$ a vráti pointer na dynamicky alokované
  pole, v ktorom bude prvých $n$ prvočísel\footnote{%
    \indexItem{Mat}{prvočíslo}
    Prvočíslo je číslo $p$, ktoré nemá iných deliteľov ako 1 a $p$. Inými slovami
    \prg~p%i != 0~ pre všetky $i<p$. Ak to chceš jemne vylepšiť, premysli si,
    že ak má číslo $a$ nejakého deliteľa, tak má aj deliteľa veľkosti 
    nanajvýš $\sqrt{a}$. Ak na začiatku napíšeš \prg!#include<cmath>!, môžeš
    použiť funkciu \prg!sqrt(x)! na výpočet $\sqrt{x}$ (vracia typ double).
    }.
\end{uloha}

\begin{uloha}
  \label{uloha:merge}
  Napíš funkciu, ktorá dostane ako parametre dve utriedené polia veľkostí $n$ a $m$
  (t.j. dostane $n$, $m$ a dva pointre) a vyrobí utriedené pole dĺžky $n+m$, 
  ktoré obsahuje prvky z oboch vstupných polí.
\end{uloha}

\begin{uloha}
  \label{uloha:mergesort}\indexItem{Alg}{MergeSort}
  Napíš funkciu, ktorá dostane ako parameter $n$ a dynamicky alokované pole
  dľžky $n$ a utriedi ho pomocou algoritmu {\em Merge Sort}. Algoritmus {\em Merge
  Sort} je rekurzívny algoritmus, ktorý funguje takto: pole dĺžky $1$ je utriedené,
  netreba nič robiť. Ak má pole dĺžku aspoň 2, rozdeľ ho na dve polovice (t.j.
  vyrob si nové polia správnej dĺžky, prekopíruj do nich prvky z pôvodného poľa)
  a rekurzívne obe utrieď. Potom použi algoritmus z úlohy~\ref{uloha:merge} na to,
  aby si vyrobil výsledné pole (nezabudni vrátiť pamäť z pomocných polí).
\end{uloha}

\begin{uloha}
  \label{uloha:sortbody}
  Majme typ \prg!struct Bod{float x,y;};! Napíš funkciu, ktorá ako parameter
  dostane pole bodov a utriedi ho\footnote{napr. pomocou algoritmu {\em Insertion Sort}
  z kapitoly~\ref{sect:zlozitost}} podľa $x$-ovej súradnice.
\end{uloha}

\begin{uloha}
  Na vstupe je číslo $n$, a pole $n$ čísel \vb{t[0]},\ldots,\vb{t[n-1]}, ktoré 
  predstavujú časované bomby: \vb{i}-ta bomba vybuchne po \vb{t[i]} sekundách.
  Pyrotechnik vie každú sekundu zneškodniť jednu bombu. Napíš program, ktorý
  zistí, či sa mu podarí zneškodniť všetky bomby. Napr. pre vstup \vb{9 3 2 5 2} 
  sa mu to podarí: v prvej sekunde zneškodní poslednú bombu, preto v druhej sekunde
  budú časovače na bombách ukazovať \vb{8 2 1 4 $\times$}. Pyrotechnik akurát včas zneškodní
  bombu číslo $2$, takže bude stav \vb{7 1 $\times$ 3 $\times$}, opäť v poslednej
  sekunde zneškodní bombu č.1: \vb{6 $\times$ $\times$ 2 $\times$} a posledné dve bomby
  zneškodní s prehľadom. Pre vstup \vb{3 4 3 7 1 3} sa mu to ale nepodarí.
\end{uloha}
