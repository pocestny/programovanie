\chapter{Meranie času, statické a privátne metódy}
\label{sect:clock}

Na meranie času slúži knižnica \vb{chrono}.\indexItem{Prg}{knižnica \vb{chrono}} 
Môže pôsobiť trochu komplikovane, ale my z
nej budeme používať iba niečo. Základom je, že sú rôzne druhy hodiniek (\vb{clock}),
napr. \prg!system_clock! alebo \prg!high_resolution_clock!. Každé hodinky majú
nejaký začiatočný čas a od neho si pamätajú počet tiknutí (každé hodinky môžu mať
rôznu rýchlosť).
Zvyšné triedy spracovávajú tiknutia hodiniek tak, aby sa
dali použiť na meranie času. Prvou triedou je \vb{duration}, v ktorej sa ukladá počet
tiknutí nejakej rýchlosti. Je to šablóna s dvoma parametrami -- prvý je typ, v ktorom
sa počet ukladá, a druhý typ, ktorý udáva, koľko sekúnd trvá jedno tiknutie.
Na dĺžku tiknutia sa používa typ \prg!ratio!, čo je šablóna reprezentujúca zlomok.
Napr. \prg!duration<long, ratio<1,10>>! 
uchováva počet tiknutí v type \prg!long!
a jedno tiknutie trvá desatinu sekundy. 
Treba zdôrazniť, že dĺžka tiknutia je parameter šablóny, 
pre každú dĺžku sa teda vyrobí separátny typ. 
Pre typické časy sú vyrobené pomocné typy, napr. \prg!milliseconds!
je \prg!duration<long, ratio<1,1000>>!, \prg!hours! je \prg!duration<int,ratio<3600,1>>!
a pod.


Počet tiknutí sa dá nastaviť v konštruktore a vrátiť funkciou \prg!count()!.
Premena jednotiek sa dá robiť pomocou šablóny \prg!duration_cast! takto:

\begin{lstlisting}
#include <chrono>
#include <iostream>

using namespace std;
using namespace chrono;
using Dvojsekundy = duration<double, ratio<2, 1>>;

int main() {
  milliseconds m(1000);
  Dvojsekundy ds = duration_cast<Dvojsekundy>(m);
  cout << m.count() << " " << ds.count() << endl;
} 

\end{lstlisting}

Tento program vypíše \vb{1000 0.5}. Premenná \vb{m}  ráta v milisekundách a
obsahuje $1000$ tikov. Premenná \vb{ds} ráta v dvojsekundových intervaloch:
$1000$ms je $1$s a to je $0.5$ dvojsekundového intervalu.


Druhou pomocnou triedou je \prg!time_point!. Je to šablóna, ktorej parametrom sú hodinky,
a pamätá si počet tiknutí od začiatku hodiniek. Každé hodinky majú funkciu \prg!now()!,
ktorá vráti aktuálny \prg!time_point!, napr.
\prg!time_point<system_clock> vcul = system_clock::now();!

Tento zápis s dvoma dvojbodkami je nový. V kapitole~\ref{sect:algstl}
sme hovorili, že dve dvojbodky znamenajú {\em ``patriaci do''}: vnorený typ, metóda triedy, funkcia v \vb{namespace} a pod.
V tomto
prípade to vyzerá, ako keby sme volali funkciu \vb{now()} z \vb{namespace system\_clock}, ale \vb{system\_clock} nie je \vb{namespace}, ale typ.
Takže \vb{now()} by mala byť metóda toho typu.
Lenže metódu z nejakého typu vieme zavolať iba na premennú toho typu (metóda má prvý neviditeľný parameter \vb{this}), niečo ako \vb{budik.now()}.
Takýto zápis označuje volanie tzv. {\em statickej metódy}. Poďme si to pozrieť detailnejšie. Doteraz sme hovorili, \indexItem{Prg}{statické metóody}
že vlastný typ vyrobený pomocou \prg!struct! má svoje premenné. 
Zoberme napr. takýto 
typ \vb{Mamut}

\vskip 2em
\vbox{
\begin{lstlisting}
struct Mamut {
  int ID;
  float chobot;
  vector<float> chlp;
};
\end{lstlisting}
}

v ktorom je o mamutovi uložený jeho identifikátor,  dĺžka chobota
a dĺžka každého chlpu. Každá premenná typu \vb{Mamut} má vyhradenú
pamäť na všetky tri položky. Povedzme, že keď chcem pridať nového mamuta,
chcem mu dať nový identifikátor tak, aby všetky mamuty mali rôzne
identifikátory. 
Na to si môžem urobiť globálnu premennú \vb{volne}, v ktorej si budem
pamätať prvý voľný identifikátor. To ale nie je veľmi pekné, lebo po čase (napr.
keď budem chcieť typ \vb{Mamut} použiť v inom programe) ľahko zabudnem, že 
k typu \vb{Mamut} patrí aj globálna premenná \vb{volne}. A navyše sa mi
ľahko môže stať, že si \vb{volne} kdesi v programe omylom niečím prepíšem.
Premennú \vb{volne} ale môžem definovať ako statickú premennú typu \vb{Mamut}:

\begin{lstlisting}
struct Mamut {
  int ID;
  float chobot;
  vector<float> chlp;

  Mamut() {
    ID = volne;
    volne++;
  }
  int pocet_chlpov() { return chlp.size(); }
  int get_ID() { return volne; }

  static int volne;
};

int Mamut::volne = 1;

int main() {
  Mamut m;
  cout << m.ID << endl;
  cout << Mamut::volne << endl;
}
\end{lstlisting}

a vyrobiť globálnu premennú \prg!Mamut::volne!, ktorá logicky patrí k
typu \vb{Mamut}, ale nepatrí žiadnej konkrétnej premennej:
každá premenná typu \vb{Mamut} má stále v pamäti iba tri položky
\vb{ID}, \vb{chobot} a \vb{chlp}.
Zápis 
\prg!static int volne;! iba hovorí \cmd{očakávam, že v programe bude
definovaná globálna premenná \prg!Mamut::volne!} t.j.
pred začiatkom programu musíš mať definíciu \prg!int Mamut::volne;!
Program by vypísal $1$ a $2$.


Podobne je to s metódami. Hovorili sme, že každá metóda nejakého typu má neviditeľný
prvý parameter \prg!this!. Ak metódu označím ako \prg!static!, tento parameter
mať nebude a bude to obyčajná funkcia, ktorá ale logicky patrí k typu.
Napr. môžem označiť \prg!static int get_ID(){...}! a bude to obyčajná funkcia, 
ktorá vráti \prg!Mamut::volne!,
ale  \prg!pocet_chlpov()! nemôže byť \vb{static}, lebo v nej používam 
\prg!this->chlp.size()!.


Stále mi ale ostáva problém, že \vb{Mamut::volne} môžem omylom kdekoľvek prepísať.\indexItem{Prg}{\vb{private}}
Ja by som pritom chcel zaručiť, aby sa používala iba v konštruktore typu 
\vb{Mamut}. Na to slúži privátna časť typu. Ak v definícii typu napíšem
\prg!private:!, všetky nasledujúce premenné a metódy budú privátne. Naspäť
do verejnej časti sa prepnem kľúčovým slovom \prg!public:!

\vskip 2ex
\vbox{
\begin{lstlisting}
struct A {
  int x; // verejné
  private:
  int y; // privátne
  public:
  int z; // verejné
  private:
  int u; // privátne
};
\end{lstlisting}
}


K privátnym premenným sa dá pristupovať iba z metód príslušnej triedy. V mojom
prípade môžem premennú \vb{volne} urobiť v type \vb{Mamut} privátnou:

\vbox{
\begin{lstlisting}
struct Mamut {
  int ID;
  float chobot;
  vector<float> chlp;

  Mamut() {
    ID = volne; // toto je v poriadku, lebo k premennej volne
    volne++;    // pristupujem z metódy (konštruktora) Mamut
  }

  int pocet_chlpov() { return chlp.size(); }
  static int get_ID() { return volne; }

  private:
  static int volne;
};

int Mamut::volne = 1; // toto je v poriadku, 
                      // inicializácia má výnimku

int main() {
  Mamut m;
  cout << m.ID << endl; 
  cout << Mamut::volne << endl;    // toto je chyba, Mamut::volne
                                   // je privátne v type Mamut
  cout << Mamut::get_ID() << endl; // toto je v poriadku                                 
}
\end{lstlisting}
}


Teraz nikde v programe nemôžem k premennej \vb{Mamut::volne} pristupovať,
takže mám zaručené, že si ju omylom neprepíšem.
Podobne to funguje aj s metódami: privátne metódy sa dajú volať iba z iných
metód typu, ktorému patria. Napr.

\vskip 2ex
\begin{lstlisting}
struct A {
  int x;

  private:
  int y;
  void hrr(int a) { y += a; }

  public:
  int frr(int a) {  // k privátnym premenným a metódam
    y = x;          // môžem pristupovať
    hrr(a);
    return y;
  }
};

int main() {
  A a;
  a.x = 1;                   // ok, x je public
  a.y = 2;                   // <- chyba  
  a.hrr()                    // <- chyba
  cout << a.frr(4) << endl;  // ok, vypíše 5
}
\end{lstlisting}

Privátne časti typov umožňujú ľahšie udržiavať poriadok vo väčších programoch: 
napr. typ \prg!set! v \vb{C++} má v \prg!public! časti všetky metódy ako 
\prg!insert()!, \prg!erase()!,
\prg!find()!, \prg!size()!, \ldots a v \prg!private! časti
implementáciu red-black stromu. Keby sa niekedy autori rozhodli vymeniť red-black
strom za AA-strom, môžu to urobiť bez strachu, že nejaké programy prestanú 
fungovať. 
Len pripomeniem, čo by malo byť jasné: privátne a statické premenné/metódy sú rôzne veci:
statický znamená, že je to normálna globálna premenná alebo funkcia, ale logicky patrí
k danému typu (a tým pádom môže byť privátna). Privátne premenné/metódy sú také, ku ktorým
sa dá pristupovať len z metód typu, ktorému patria.
Na záver tejto vsuvky posledná poznámka: okrem kľúčového slova 
\prg!struct! sa na definovanie nových typov dá použiť aj \prg!class!. Robia
presne to isté, iba \prg!struct! má default časť \prg!public! a \prg!class!
má default \prg!private!.


Poďme naspäť k meraniu času. 
Skončili sme pri tom, že každý typ hodiniek má statickú metódu
\prg!now()!, ktorá vráti aktuálny čas ako \prg!time_point!, napr. \indexItem{Prg}{\vb{tiem\_point::time\_since\_epoch()}}

\begin{lstlisting}
time_point<system_clock> vcul = system_clock::now();
\end{lstlisting}

Ešte treba vedieť, že \prg!time_point! má metódu \prg!time_since_epoch()!, ktorá
vráti \prg!duration! od začiatku príslušných hodiniek a že 
dve premenné typu \prg!time_point! môžem odčítať a dostanem 
príslušný \prg!duration!.


\IGNORE{
Keď chcem preto nastaviť seed do pseudonáhodného generátora, často sa používa čosti ako

\begin{lstlisting}
srandom(system_clock::now().time_since_epoch().count());
\end{lstlisting}
}


\begin{uloha}
  Naprogramuj funkciu \prg!template<typename F> int meraj(F f){...}!, ktorá
  má ako parameter funkciu /lambdu bez parametrov a odmeria, koľko milisekúnd trvá
  jej spustenie.
\end{uloha}


S riešením predchádzajúvej úlohy
môžeš skúsiť pozorovať, ako je kompilátor schopný veci optimalizovať. 
Zober si napr. program

\begin{lstlisting}
int main() {
  cout << meraj([]() {
    int x;
    for (int i = 0; i < 1000000000; i++) x++;
  }) << endl;
}
\end{lstlisting}

A skompiluj ho raz \vb{g++ test.cc -o test} a raz s prepínačom \vb{O3} napr.
\vb{g++ -O3 test.cc -o test\_opt}. Pri optimalizácii kompilátor pochopil, že v tom cykle
nič zaujímavé nerobíš a dá sa to zrátať priamo.

\begin{uloha}
S pomocou predchádzajúcej úlohy napíš program, ktorý odmeria rýchlosť triedenia takto:
pre každé $n$ z rozsahu $10000,20000,30000,\ldots,1000000$ sa $10\times$ vygeneruje
náhodný vektor s $n$ prvkami, odmeria sa čas, ktorý treba na jeho utriedenie
  funkicou \prg!sort! z knižnice \prg!algorithm! a vypíše sa priemer z 
  časov v milisekundách.
\end{uloha}

