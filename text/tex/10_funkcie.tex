\chapter{Funkcie}

Kapitolu~\ref{sec:CB obrazky} som začal tým, že som ti pripravil súbor, v ktorom
bol nový príkaz \prg!zapis_cb_png!. Teraz ti idem ukázať, ako si nové príkazy, ktorým
sa hovorí {\em funkcie}, môžeš vyrobiť aj ty. 
Možnosť vyrobiť si vlastné funkcie je veľmi dôležitá:
umožňuje ti rozdeliť si veľký program na malé časti, ktoré potom môžeš používať viackrát.


Najjednoduchšia funkcia vyzerá takto:\\

\vbox{
\begin{lstlisting}[] 
#include <iostream>
using namespace std;

int sedem() {
  return 7;
}

int main() {
  cout << sedem() << endl;
}
\end{lstlisting}
}

Všimni si, že sa veľmi podobá na zápis, ktorý používame na program. Nie je to náhoda.
Celý program je v skutočnosti len kopa funkcií, ibaže jedna z nich je špeciálna: tá, ktorá
sa volá \prg!main! sa začne vykonávať ako prvá. Každá funkcia je samostatný program,
ktorý má svoj vlastný svet.

 Funkcia má meno\footnote{Podobne
ako pri premenných, meno funkcie môže byť čokoľvek z veľkých a malých písmen, podčiarkovníkov
a čísel, ale nesmie začínať číslom.} a typ. V našom programe sme spravili funkciu, ktorá
sa volá \vb{sedem}. Typ \prg!int! pred jej menom hovorí, že táto funkcia predstavuje
výraz, ktorého výsledok je celé číslo.
Guľaté zátvorky za menom slúžia
na parametre. Zatiaľ žiadne nemáme, ale zátvorky aj tak treba písať, aby bolo jasné, že je to funkcia a nie premenná.
Potom nasleduje zložený príkaz, t.j. v kučeravých zátvorkách napísané príkazy. 

 Funkciu vieme v nejakej inej funkcii použiť ({\em zavolať}) tak, že napíšeme jej meno a parametre,
v našom prípade \prg!sedem()!. 
Pretože naša funkcia má typ \prg!int!, vieme, že každé jej zavolanie {\em vráti} celé číslo a
môžme ju použiť
vo výrazoch rovnako, ako by sme používali číslo. Môžeme napr. urobiť 
\prg!int x = 3 + sedem();! a pod. Kedykoľvek treba vyhodnotiť výraz, v ktorom
sa volá funkcia, všetko sa preruší a spustí sa tá funkcia. Príkaz
\prg!cout << sedem() << endl;! v našom programe hovorí \cmd{
  Začni vypisovať, mal by si vypísať číslo. Preruš robotu, spusti funkciu \prg!sedem()! 
  a počkaj, kým skončí.
Keď skončí, ako výsledok ti dá číslo. Pokračuj vo vypisovaní, vypíš  toto číslo a potom znak 
konca riadku.}

\indexItem{Prg}{\vb{return}}
 Príkazy, ktoré sa vo funkcii vykonávajú sa volajú {\em telo} funkcie. Špeciálny 
príkaz \prg!return vyraz;! urobí to, že skončí funkciu a výsledok funkcie 
(tzv. {\em návratová hodnota}) bude hodnota výrazu \prg!vyraz!. Typ výrazu musí byť rovnaký
ako návratový typ funkcie, ktorý sme zadali pri jej definícii. Napr. naša funkcia
\prg!sedem()! vracia typ \prg!int!, preto výraz v príkaze \prg!return! musí mať hodnotu typu
\prg!int!. Funkcie ale môžu mať ako návratovú hodnotu ľubovoľný typ. Ak by som mal
funkciu \prg!bool zle() { return false; }!, môžem ju použiť všade, kde môžem mať výraz typu
\prg!bool!, napr. \prg!if (zle()) {}!.


Každá funkcia musí skončiť príkazom
\prg!return!\footnote{Slušné by bolo, aby sme aj na konci hlavného programu písali 
\prg!return 0;!, ale keďže už potom program skončí, tak to nevadí, keď to vynecháme.}.
Funkcia je samostatný svet, takže môže mať vlastné premenné. Tie sú úplne nezávislé
od iných funkcií (hovoríme im\indexItem{Prg}{lokálne premenné}{\em lokálne premenné}). Čo urobí tento program?

%\vbox{
\begin{lstlisting}[] 
#include <iostream>
using namespace std;

int pridaj() {
  int x;
  cin >> x;
  return x + 7;
}

int main() {
  int x = pridaj();
  cout << x + pridaj() << endl;
}
\end{lstlisting}
%}

Hlavný program vyrobí premennú \prg!x!. Zavolá funkciu \prg!pridaj()! a jej
výsledok má uložiť do \prg!x!. Keď sa zavolá funkcia \prg!pridaj()!, vznikne 
nový svet, v ktorom sa vytvorí iná premenná \prg!x! a načíta sa do nej číslo zo vstupu.
Dajme tomu, že používateľ napíše 2.
Potom sa funkcia \prg!pridaj()! skončí, jej svet zanikne, a ostane len výsledok, číslo 9. 
Pracovať pokračuje hlavný program, do svojej premennej \prg!x! 
uloží výsledok volania funkcie, t.j. 9. Pokračuje príkazom \prg!cout <<!, 
na to potrebuje vyhodnotiť
výraz \prg!x + pridaj()!. V \prg!x! má 9, opäť preruší robotu a zavolá \prg!pridaj()!.
Zase vznikne nový svet s premennou \prg!x!, do ktorej sa prečíta hodnota. 
Dajme tomu, že teraz
je na vstupe 3. Funkcia skončí, jej svet zanikne a ostane číslo 10. Hlavný program
teraz dokončí rátanie výrazu \prg!9 + 10! a vypíše \prg!19!.


\indexItem{Prg}{parametre funkcie}
Doteraz sme nehovorili o parametroch. Parametre umožňujú používať tú istú funkciu na
rátanie rôznych vecí. Sú to premenné, ktoré sú lokálne premenné funkcie, ale
ich hodnota sa dá nastaviť zvonka pri volaní funkcie. Parametre a ich typy
sú uvedené v zátvorkách za menom funkcie, takže napr.\\

\vbox{
\begin{lstlisting}[] 
int scitaj(int x, int y) {
  int z = x + y;
  return z;
}
\end{lstlisting}
}

je funkcia, ktorá má dva parametre a oba sú celé čísla. Vo vnútri funkcie sa 
k nim dá správať ako k normálnym premenným. Majme hlavný program:\\

\vbox{
\begin{lstlisting}[] 
int main() {
  int x = 1;
  x = scitaj(x, scitaj(x, 3 * x));
  cout << x << endl;
}
\end{lstlisting}
}

Skús sa zamyslieť, čo vypíše. Máš to? Poďme to pozrieť detailne. Hlavný program
sa spustí, vyrobí si svoju premennú \vb{x} a dá do nej $1$. Potom ide
vyhodnocovať výraz, ktorého výsledok má uložiť do \vb{x} a zistí, že je to 
volanie funkcie.

\def\tw{2}
\def\th{3ex}
\def\r{1ex}
\def\vw{3.5ex}
\def\mimo{\color{black!50!white}}
\def\hi{\color{red!50!black}}
\def\dn{\color{blue!80!white}}
\def\fr#1{
  \filldraw[thick,draw=\clr,fill=white,rounded corners=\r] (0,\th) rectangle (\tw,-\r);
  \filldraw[draw=none, fill=white] (0,0) rectangle (\w,-\h);
  \draw[thick,draw=\clr] (0,-\h) -- (0,0) -- (\w,0) -- (\w,-\h);
  \draw[thick,draw=none] (0,\th) rectangle node[color=\clr,anchor=center]{\vb{#1}} (\tw,0);
}
\def\var(#1,#2)#3#4#5{
  \draw (#1,#2) rectangle node[anchor=center]{#4} (#1+\vw,#2-\vw) 
  coordinate[pos=0.5] (#5);
  \node [anchor=east] at (#1,#2-0.5*\vw) {\vb{#3:}};
}

\def\conn[#1][#2](#3,#4){
    \draw[draw=orange, thick, ->, shorten >= 3ex, shorten <= 3ex] (#1) to[in=#3,out=#4] 
    node[midway, above, color=orange]{}(#2);
}

\def\fsc{[yscale=0.78]}
\begin{tikzpicture}\fsc
  \def\h{1.8}
  \def\w{6.2}
  \def\clr{green!55!black}
  \fr{main}
  \var(4ex,-1ex){x}{1}{mX}
  \node [anchor=south west] at (0,-10ex) {\vb{ 
  {\mimo x = {\hi scitaj(}x, scitaj(x, 3 * x){\hi )}};}};
\end{tikzpicture}


Vyrobí sa teda nový svet funkcie \vb{scitaj} a hlavný program chce nastaviť
hodnoty parametrov. 
Prvý parameter nastaví podľa aktuálneho obsahu \vb{x}, 
ale pri druhom zistí, že je to výraz, v ktorom 
je opäť volanie funkcie.


\begin{tikzpicture}\fsc
  \def\h{1.8}
  \def\w{6.2}
  \def\clr{green!55!black}
  \fr{main}
  \var(4ex,-1ex){x}{1}{mX}
  \node [anchor=south west] at (0,-10ex) {\vb{ 
  {\mimo x = {\dn scitaj(x, }{\hi scitaj(}x, 3 * x{\hi )}{\dn )}};}};

  \begin{scope}[shift={(8,-1)}]
  \def\h{1}
  \def\w{4}
  \def\clr{blue!55!black}
  \fr{scitaj}
  \var(4ex,-1ex){x}{1}{vX}  
  \var(15ex,-1ex){y}{}{vY}  
  \draw[draw=\clr](0,-5ex) -- (\w,-5ex); 
  \conn[mX][vX](150,10);
  \end{scope}
\end{tikzpicture}

 Vytvorí sa teda ďalší, nezávislý, svet funkcie \vb{scitaj}, nastavia sa mu hodnoty
parametrov a vyráta sa výsledok, ktorý sa potom vráti tomu, kto funkciu zavolal.


\tikzexternaldisable
\begin{tikzpicture}[remember picture, yscale=0.78]%\fsc
  \def\h{1.8}
  \def\w{6.2}
  \def\clr{green!55!black}
  \fr{main}
  \var(4ex,-1ex){x}{1}{mX}
  \node [anchor=south west] at (0,-10ex) {\vb{ 
  {\mimo x = {\dn scitaj(x, }{\hi scitaj(}x, 3 * x{\hi )}{\dn )}};}};

  \begin{scope}[shift={(10,0.2)}]
  \def\h{1}
  \def\w{4}
  \def\clr{blue!55!black}
  \fr{scitaj}
  \var(4ex,-1ex){x}{1}{vX}  
  \var(15ex,-1ex){y}{}{vY}  
  \draw[draw=\clr](0,-5ex) -- (\w,-5ex); 
  \end{scope}
  
  \begin{scope}[shift={(7.5,-1)}]
  \def\h{2.6}
  \def\w{4}
  \def\clr{orange!55!black}
  \fr{scitaj}
  \var(4ex,-1ex){x}{1}{vvX}  
  \var(15ex,-1ex){y}{3}{vvY}  
  \draw[draw=\clr](0,-5ex) -- (\w,-5ex); 
  \conn[mX][vvX](140,20);
  \conn[mX][vvY](140,20);
  \var(4ex,-6ex){z}{4}{vvZ}  
    \node [anchor=south west] at (0,-13ex) {\vb{... }};
    \node [anchor=south west] at (0,-16ex) {\vb{return z;}};
  \draw[draw=\clr](0,-\h) -- (\w,-\h); 
  \end{scope}
\end{tikzpicture}

 Teraz druhý svet zanikne, ostane z neho iba výsledok $4$, a tým sa nastaví parameter
v prvom svete:


\begin{tikzpicture}[remember picture,  yscale=0.78]
  \def\h{1.8}
  \def\w{6.2}
  \def\clr{green!55!black}
  \fr{main}
  \var(4ex,-1ex){x}{1}{mX}
  \node [anchor=south west] at (0,-10ex) {\vb{ 
  {\mimo x = {\dn scitaj(x, }{\color{white} scitaj(x, 3 * x )}{\dn )}};}};
  \node[anchor=south west] (vys) at (24ex,-10ex) {\color{orange} 4};

  \begin{scope}[shift={(8,-0.5)}]
  \def\h{2.6}
  \def\w{4}
  \def\clr{blue!55!black}
  \fr{scitaj}
  \var(4ex,-1ex){x}{1}{vX}  
  \var(15ex,-1ex){y}{4}{vY}  
  \draw[draw=\clr](0,-5ex) -- (\w,-5ex); 
  \var(4ex,-6ex){z}{5}{vZ}  
    \node [anchor=south west] at (0,-13ex) {\vb{... }};
    \node [anchor=south west] at (0,-16ex) {\vb{return z;}};
  \draw[draw=\clr](0,-\h) -- (\w,-\h); 
  \end{scope}
 
  \conn[vys][vY](150,10);
  %\conn[vvZ][vys](0,0);
    \draw[overlay, draw=orange!80, thick, ->, opacity=0.4, shorten >= 1ex, shorten <= 3ex] (vvZ) to[in=90,out=180] 
    node[midway, above, color=orange]{}(vys);
\end{tikzpicture}
\tikzexternalenable


Napokon zanikne aj prvý svet, ostane hodnota 5 a tá sa zapíše do premennej \vb{x}
v hlavnom programe. Takže si to zhrňme: funkcia je samostatný program, ktorý má 
výstupnú hodnotu\footnote{Tu len pripomeniem: úplne na začiatku sme hovorili,
že najjednoduchší príkaz \vb{3*(2+5);} len vyráta hodnotu $21$ a výsledok zahodí. 
To je, samozrejme, zbytočné, ale niekedy môžeš chcieť zahodiť hodnotu funkcie. Napríklad
ak zoberiem funkciu \vb{pridaj} z predchádzajúceho príkladu , tak
volanie \prg!pridaj();! načíta číslo zo vstupu a 
ignoruje ho.}.
Dá sa použiť vo výrazoch. Pri volaní funkcie sa vytvorí
nový nezávislý svet, nastavia sa hodnoty parametrov\footnote{%
Zatiaľ vieme ako parametre funkciám posielať jednoduché premenné, ale nie polia. 
Polia sa tiež dajú ako parametre použiť, ale o tom až neskôr.
} a vypočíta sa výstupná hodnota.

\begin{uloha}
  Napíš funkciu \vb{max} s tromi parametrami, ktorá vráti najväčší z nich.
\end{uloha}

\begin{uloha}
  Napíš funkciu \prg! int dlzka(int n) {...}!, ktorá pre zadané číslo $n$
  vráti počet cifier (napr. \vb{dlzka(546)} je 3). Pomocou nej napíš program,
  ktorý prečíta zo vstupu číslo \vb{n} a vypíše zarovnanú tabuľku s 
  mocninami čísel $2$ až $n$. Napr. pre $n=7$ by výstup vyzeral

\begin{outputBox}
      1      2      4      8     16     32     64
      1      3      9     27     81    243    729
      1      4     16     64    256   1024   4096
      1      5     25    125    625   3125  15625
      1      6     36    216   1296   7776  46656
      1      7     49    343   2401  16807 117649
\end{outputBox}
\end{uloha}


\indexItem{Prg}{globálne premenné}
Doposiaľ sme mali každú premennú vyrobenú v nejakej funkcii (či už \vb{main} alebo 
v nejakej inej). Dajú sa však vyrobiť aj premenné, ktoré žiadnej funkcii nepatria
a sú viditeľné z každého sveta: hovorí sa im {\em globálne premenné}. Najľahšie
sa to ukáže na píklade:

%\vbox{
\begin{lstlisting}[label=funkcie.0] 
#include <iostream>
using namespace std;

int a; @\ll1@

void pridaj(int kolko) { a = a + kolko; } @\ll2@

int main() {
  int i;
  a = 0;
  for (i = 0; i < 10; i++) pridaj(i);
  cout << a << endl;
}
\end{lstlisting}
%}


Tu sme vyrobili globálnu premennú \vb{a}. Keďže je vyrobená skôr, ako
definujeme funkciu \vb{pridaj}, telo funkcie môže k premennej \vb{a} pristupovať
\footnote{Keby sme vymenili riadky \ref{funkcie.0-1} a \ref{funkcie.0-2}, 
globálna premenná \vb{a} by vo funkcii \vb{pridaj} nebola viditeľná; premenná
musí byť vždy vyrobená v programe skôr, ako sa použije.}.
\indexItem{Prg}{typ \vb{void}}
Tu som použil ako návratový typ nový typ: \prg!void!. Je to prázdny typ, ktorým hovoríme,
že funkciu chcem vždy použiť iba v situácii, keď výslednú hodnotu zahodím. Takéto funkcie
nemusia končiť príkazom \prg!return!. Po spustení program vypíše $0+1+2+\cdots+9=45$.

\indexItem{Prg}{viditeľnosť}
Poznámka o viditeľnosti: rovnako ako pri funkciách sa samostatný svet
vyrába aj pri zložených príkazoch. To znamená, že svety môžu zapadať jeden do druhého:
svet globálnych premenných, v ňom svet funckie, v ňom svet zloženého príkazu,\ldots.
Vždy, keď program potrebuje použiť nejakú premennú,,
vyberie sa premenná s požadovaným menom z najbližšieho sveta, v ktorom taká je. 
Napr. program\\

\vbox{
\begin{lstlisting}[label=funkcie.1] 
#include <iostream>
using namespace std;

int x = 5;

void p() { cout << x; }

int main() {
  int x = 7;
  int a = 9;
  cout << x;
  p();
  { 
    int x = 6; @\ll2@
    cout << x << a;
    p(); @\ll1@
  }
  cout << x << endl;
}
\end{lstlisting}
}

vypíše \footnote{Všimni si, že C++ má tzv. {\em static scope}: ktorý je najbližší 
svet sa rozhoduje podľa toho, ako je to v programe zapísané. Takže vo volaní \prg!p()!
z riadku \ref{funkcie.1-1} sa použije globálna premenná \vb{x} a nie premenná \vb{x}
zo sveta zloženého príkazu na riadku \ref{funkcie.1-2}.}
\vb{756957}: 
funkcia \vb{main} má svoju vlastnú \vb{x}, preto sa použije tá.
Funkcia \vb{p} vo svojom svete nemá premennú \vb{x}, preto sa použije
globálna \vb{x}. 
Zložený príkaz má svoj vlastný svet, v ktorom má svoju vlastnú \vb{x}, ale \vb{a}
sa použije zo sveta \vb{main}.


\begin{column}{0.5}
Globálne premenné môžu byť rovnako dobre aj  polia. Tu je ale problém, akú veľkosť má
pole mať. Pretože ho vyrábam skôr, ako sa spustí hlavný program, nemôžem čakať, kým
zo vstupu dostanem veľkosť poľa. Najjednoduchší spôsob je povedať si dopredu ohraničenie
na veľkosť poľa, vyrobiť dostatočne veľké pole a potom z neho použiť len potrebný kus.
Toto nie je ideálny prístup (okrem iného mrhá pamäťou), ale nateraz nám postačí.
Napr. program vpravo:
\end{column}\hfill\begin{column}{0.4}
\vbox{
\begin{lstlisting}[] 
#include <iostream>
using namespace std;

int a[1000];

int main() {
  int n = -1;
  do {
    n++;
    cin >> a[n];
  } while (a[n] > 0);
}
\end{lstlisting}
}
\end{column}

číta do poľa \vb{a} vstupné kladné čísla a skončí, keď sa prvýkrát prečíta niečo 
$\le 0$. V premennej \vb{n} (ktorá je lokálna funkcii \vb{main}) je uložená ''skutočná''
dĺžka poľa. Ak by bolo na vstupe viac ako 1000 čísel, stanú sa hrozné veci.

\begin{uloha}
  \label{uloha:sort2}
  Napíš funkciu s parametromm \prg!int n!, ktorá v globálnom poli \vb{a} 
  (o ktorom predpokladáme, že obsahuje kladné čísla) nájde najväčší
  prvok spomedzi prvých \vb{n}, prepíše ho v poli na \vb{-1} a vráti jeho hodnotu
  (ešte predtým, ako ho prepísala). Potom s pomocou nej
  napíš program, ktorý do poľa prečíta kladné
  čísla zo vstupu a vypíše ich v utriedenom poradí.
\end{uloha}

Na ľahšiu prácu s globálnymi premennými je v súbore \vb{obrazok.h}
vyrobená funkcia \hbox{\vb{zapis\_cb\_png\_vyrez}}, ktorá má dodatočné
štyri parametre. Tie hovoria, ktoré riadky a ktoré stĺpce sa majú do obrázka zapísať.
Program

\vskip 2ex
\vbox{
\begin{lstlisting}[] 
#include <iostream>
#include "obrazok.h"
using namespace std;

const int N = 2000;
int a[N][N];

int main() {
  // tuto niečo nakresli
  zapis_cb_png_vyrez(N, N, a, "vystup.png", 50, 80, 300, 200);
}
\end{lstlisting}
}

vypíše do súboru \vb{vystup.png} výrez z poľa \vb{a}  tvorený riadkami
$50$ až $299$ a stĺpcami $80$ až $199$. Pretože \vb{a} je teraz vyrobené ako globálna
premenná, jeho rozmery musia byť známe už v čase kompilácie. \indexItem{Prg}{modifikátor \vb{const}}Na to je špeciálny
modifikátor \prg!const!, ktorý sme použili pri vytváraní premennej \vb{N} a ktorý
hovorí, že \vb{N} sa v programe nebude meniť (môžeš si vyskúšať, že to nejde).

\begin{column}{0.7}
\begin{uloha}
  Napíš funkciu \prg!void stvorec(int r, int s, int d)!, ktorá v globálnom poli
  vyplní jednotkami štvorec $d\times d$ začínajúci na riadku $r$ a stĺpci $s$.
  S jej pomocou napíš program, ktorý načíta tri čísla $a, b, c$ také, že
  $a+b+c<100$ a vyrobí obrázok rozmerov $100\times 100$, v ktorom budú tri štvorce
  rozmerov $a\times a$, $b\times b$ a $c\times c$ položené jeden na druhom 
  v strede obrázka. Napr. pre vstup $40, 30, 20$ bude obrázok vyzerať takto:
\end{uloha}
\end{column}
\begin{column}{0.3}
 {
\setlength{\fboxsep}{0pt}
\centerline{\fbox{\includegraphics[width=3.5cm]{data/tristvorce.png}}}
  }
\end{column}  
