\chapter{Polia}

Chceme napísať takýto program: najprv napíšeš číslo $n$ a potom $n$ prirodzených 
čísel. Nakoniec napíšeš ešte jedno číslo $a$. Program má vypísať najmenšie zo zadaných čísel,
ktoré je väčšie ako $a$ (alebo vypíše $a$, ak žiadne z čísel nie je väčšie). 
Napríklad:

\vspace*{-2ex}
\begin{column}{0.45}
Vstup\\
\begin{outputBox}
5
1 10 100 1000 10000
427
\end{outputBox}
\end{column}
\hfill
\begin{column}{0.45}
Výstup\\
\begin{outputBox}
1000
\end{outputBox}
\end{column}


Problém je v tom, že keď načítavame čísla zo vstupu, ešte nevieme, ako ich bude treba 
spracovať, takže by sme si ich potrebovali uložiť do premenných. Keby sme počet čísel
poznali v čase písania programu (napr. by bolo povedané, že vždy bude presne päť čísel),
mohli by sme si vyrobiť premennú pre každé číslo. Počet čísel je ale známy až počas behu 
programu. 
\indexItem{Prg}{pole}
Použijeme preto {\em pole} premenných. Pole je vlastne veľa premenných, ktoré
majú spoločné meno a rozlišujú sa číslom (tzv. {\em indexom}). Pole sa vyrobí podobne,
ako jedna premenná, ale v hranatých zátvorkách je uvedený počet premenných.
Tak napr. \prg!int x[10];! vyrobí pole premenných \prg!x[0]!, \prg!x[1]!, až
\prg!x[9]!, z ktorých každá vie mať uložené celé číslo\footnote{Všimni si, že prvá premenná
má číslo 0, preto ak vyrobíme 10 premenných, posledná má číslo 9.}.
Výhoda je, že keď chceme použiť nejakú premennú z poľa, nemusíme písať jej meno priamo,
napr. \prg!x[5]!, ale na mieste indexu môžeme použiť hocijaký výraz. Takže môžeme
napísať \prg!x[2+3]! a tiež dostaneme hodnotu z premennej \prg!x[5]!. A samozrejme,
výraz môže obsahovať aj mená premenných, takže \prg!x[i+3]! znamená \cmd{Pozri sa,
čo je práve teraz uložené v premennej \prg!i!, pripočítaj k tomu 3, výsledok zober ako
číslo premennej z poľa a vráť hodnotu, ktorá je v nej uložená.} Ak by v \prg!i! bola uložená dvojka, tak \prg!x[i+3]! vráti hodnotu
uloženú v premennej \prg!x[5]!. 
Pole nie je premenná, ale veľa premenných. Preto aj príkaz priradenia funguje
na jednotlivých premenných, ale nie na celom poli. Ak by sme mali premenné
\prg!int a[10], b, c[10], d;! môžeme napísať \prg!b=d;! aj \prg!a[d+3]=b;! aj
\prg!c[b+a[d]]=a[d]+c[3];! ale nemôžeme napísať \prg!a=c;! a už vôbec nie
\prg!a=b;! Rovnako, keďže pole nie je premenná, tak nemôžeme používať ani
operácie ako \prg!a+c! alebo \prg!a==c!, ale musíme ich robiť na jednotlivých premenných 
poľa.

Predpokladajme, že máme pole \prg!int a[10];! a premennú \prg!int x!. Čo spraví príkaz
\prg!x=a[22]!? Kompilátor by mal hádam vyhlásiť chybu, ale on nie. 
Premenné sú totiž krabičky, ktoré sú v pamäti uložené jedna za druhou.
Príkazom \prg!int a[10];! kompilátor
vyhradí v pamäti miesto na 10 premenných typu \prg!int! a zapamätá si, kde je prvá z nich,
\prg!a[0]!. Ich počet sa nikde nezapamätá.
Keď napíšeš \prg!a[5]!, znamená to \cmd{piata premenná typu \vb{int} za premennou \prg!a[0]!}. Ak napíšeš
\prg!a[22]!, program sa pozrie do pamäte, kde by bola 23. premenná z poľa, keby to 
pole bolo dosť dlhé. Ak je kratšie, môže na tom mieste byť hocičo (napríklad nejaká
iná premenná) a program sa
začne správať veľmi podivne. Chyby tohto typu sa v programe veľmi ťažko hľadajú.
Preto je dobré dávať veľký pozor, na aké miesta poľa pristupuješ.

\indexItem{Prg}{lenivé vyhodnocovanie podmienok}
Pri pristupovaní do poľa je príjemné, že výrazy typu \prg!bool! sa vyhodnocujú len dovtedy, kým
nie je jasný výsledok, potom sa s počítaním prestane. Napr. pri výraze
\prg!false  && ( zlozita_vec )! je hneď po vyhodnotení prvého \prg!false! jasné, že výsledok bude
\prg!false!, preto sa \prg!zlozita_vec! ani nevykoná. To umožňuje napísať do jednej podmienky 
kontrolu, či je index do poľa správny
napr. napíšem
\prg!if (i>=0 && i<10 && a[i]==5)! a viem, že z poľa sa bude čítať iba vtedy, ak je premenná
\prg!i! zo správneho rozsahu.

Vráťme sa k nášmu príkladu. S pomocou poľa a cyklu si vieme zapamätať všetky zadané čísla:

\vskip 2ex
\vbox{
\begin{lstlisting} 
int i, n;
cin >> n;       // prečítam počet čísel 
int x[n];       // vyrobíme pole potrebnej veľkosti
i = 0;         
while (i < n) { // v cykle prečítame všetky prvky poľa
  cin >> a[i];
  i = i + 1;
}
\end{lstlisting}
}

Keď už budeme mať všetky čísla uložené v poli, prečítame zo vstupu číslo $a$. Ak chceme
zistiť najmenšie číslo, ktoré je väčšie ako $a$, urobíme si ďalšiu premennú \prg!min!,
v ktorej si budeme pamätať doteraz nájdené číslo. V cykle sa pozrieme na každú premennú z 
poľa a zistíme, či je jej hodnota väčšia ako $a$. Ak áno a je zároveň menšia ako
\prg!min!, aktualizujeme \prg!min!. Posledná vec: na začiatku potrebujeme do \prg!min!
dať vhodnú zarážku. Môžeme použiť najväčšie číslo, ktoré si môžeme zistiť priebežne
pri ukladaní vstupného poľa. Celý program by vyzeral takto:\\

\vbox{
\begin{lstlisting}
#include <iostream>
using namespace std;
int main() {
  int n, i, a, min;
  cin >> n;
  int x[n];       // vyrobíme pole správnej dĺžky        
  i = 0;
  min = 0;
  while (i < n) { // načítavame vstup a zároveň počítame maximum
    cin >> x[i];
    if (x[i] > min) min = x[i];
    i = i + 1;
  }
  cin >> a;       // prečítame a
  i = 0;
  while (i < n) { // v cykle nájdeme hľadané číslo
    if (x[i] > a && x[i] < min) min = x[i];
    i = i + 1;
  }
  cout << min << endl;
}
\end{lstlisting}
}

\begin{uloha}
  Používateľ zadá číslo $n$, potom $n$ čísel a nakoniec číslo $a$. Napíš
  program, ktorý zistí, koľkokrát bolo zadané číslo $a$. Napríklad

\begin{column}{0.45}
Vstup\\
\begin{outputBox}
9 3 9 3 1 3 2 3 4 5 3
\end{outputBox}
\end{column}
\hfill
\begin{column}{0.45}
Výstup\\
\begin{outputBox}
4
\end{outputBox}
\end{column}
\end{uloha}

\begin{uloha}
  Používateľ zadá číslo $n$, potom $n$ čísel a nakoniec číslo $a$. Napíš
  program, ktorý vypíše všetkých $n$ čísel, ale zväčšených o $a$.
\end{uloha}

\begin{uloha}
   Používateľ zadá číslo $n$ a potom zoznam $n$ kariet s číslami. 
   Dvaja hráči, $A$ a $B$ hrajú takúto hru. V každom kole sa $A$ pozrie
   na prvú kartu a $B$ na poslednú kartu (ak je v zozname iba jedna karta,
   je zároveň prvá aj posledná; ak je zoznam prázdny, hra sa skončila).
   Ak $A$ vidí väčšie číslo, výsledok kola je $1$ a $B$ zahodí kartu z konca, 
   ak $A$ vidí menšie číslo, výsledok
   je $-1$ a $A$ zahodí kartu zo začiatku, a ak vidia rovnaké, výsledok je $0$
   a zahodí sa karta z konca aj zo začiatku.
   Napíš program, ktorý prečíta začiatočný zoznam a vypíše postupne výsledky
   všetkých kôl. Napr. pre $n=3$ a vstup \vb{1 2 3} sú výsledky
   kôl \vb{-1 -1 0}.
\end{uloha}

\begin{uloha}
  \label{uloha:sort1}
  Používateľ zadá číslo $n$ a potom $n$ rôznych čísel. Napíš program, ktorý
  najprv vypíše najmenšie číslo a potom vždy najmenšie číslo, ktoré
  je väčšie, ako posledné vypísané. (Všimni si, že program vlastne
  vypíše všetky zadané čísla v utriedenom poradí.)
\end{uloha}

\begin{uloha}
  Používateľ zadá číslo $n$ a potom pole $n$ prirodzených čísel. Napíš program,
  ktorý zistí najväčšie také číslo $k$, že sa v poli nachádza aspoň $k$
  čísel veľkosti aspoň $k$. Napr. pre vstup \prg!9 3 9 1 1 2 2 3 4 5! 
  %(prvá deviatka znamená 9 čísel, potom nasleduje pole) 
  je odpoveď \prg!3!, lebo
  sú tam aspoň tri čísla väčšie alebo rovné $3$ (je tam $5$ takých: $3, 9, 3, 4, 5$),
  ale nie sú tam štyri čísla väčšie alebo rovné $4$ (sú tam len tri: $9, 4, 5$).
\end{uloha}

