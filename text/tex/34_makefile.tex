\chapter{Ako udržať poriadok vo veľkých projektoch}
\label{sect:makefile}

V tejto časti budem hovoriť o usporiadaní väčšieho projektu.
Prostredia ako
\link{https://www.codeblocks.org/}{Code::Blocks}, či \link{https://visualstudio.microsoft.com}{MS Visual Studio}
majú spravidla možnosť vytvoriť projekt, ktorý sa skladá z viacerých súborov. 
Ja tu budem ukazovať príklady v prostredí linuxového terminálu: jednak dávajú väčšiu flexibilitu, nie sú závislé od konkrétneho 
prostredia a dá sa na nich najlepšie pochopiť, ako veci interne fungujú.
Pre Windows sa dá nainštalovať napr. \link{https://www.msys2.org/}{MSYS2}, ktorý poskytuje
viac-menej rovnaké prostredie (terminál, príkazy, \ldots). 


Čím väčší program píšeš (a niektoré programy môžu byť veľmi veľké), 
začína byť čoraz dôležitejšie udržovať si v programe
poriadok. Ak dopredu vieš, že program, ktorý ideš písať, môže byť časom dlhší, je 
dobré si v ňom začať dodržovať poriadok už hneď od začiatku. V kapitole
\ref{sect:zasobnik} sme hovorili o tom, že časti programu, ktoré spolu súvisia (napr. 
definíciu triedy a jej metód) je dobré dať do samostatného súboru a do hlavného programu
ho pridať pomocou direktívy \prg!#include!. 
Dobré je aj dať si na začiatok takého súboru
dlhší komentár, kde si napíšeš názvy metód, čo robia ako ako sa používajú: keď budeš po čase potrebovať
zistiť, ako sa tá trieda vlastne používa, nebudeš musieť hľadať hlboko v programe.

Toto ale nerieši iný problém, a tým je čas kompilácie. 
Isto si si všimol, že hlavne ak použiješ optimalizáciu (napr. prepínač \vb{-O3}), program sa
kompiluje pomerne dlho. Ak máš aj program rozdelený do viacerých súborov, preprocesor ich pred
začiatkom kompilácie zlepí do jedného, takže sa vždy kompiluje všetko. Aby sme si ukázali,
ako sa to dá zrýchliť, treba vedieť, že kompilátor vyrába program v dvoch fázach: najprv vyrobí\indexItem{Prg}{object code, linker}
tzv. {\em object code}, kde sú pre všetky funkcie 
vyrobené príkazy pre procesor, ale ešte nemajú presne
určené adresy v pamäti. Object code potom spracuje tzv. {\em linker}, ktorý zoberie kusy z object code
a zlepí ich do výsledného spustiteľného súboru. To, čo trvá dlho, je vyrobiť object code, následné 
linkovanie je krátke. Za normálnych okolností object code ani nevidíš, ale kompilátoru sa dá
(pre \vb{g++} je to prepínač \vb{-c}) povedať, aby ho napísal do súboru a nelinkoval. 
Dajme tomu, že v súbore \vb{main.cc} máš nasledovný program:

\vskip 2ex
\begin{lstlisting}
int cibulka = 4;
int lopata = 5;

int tulipany() {
  int res = 1;
  while (cibulka-- > 0) res <<= 1;
  return res;
}

int jama(int x) { return x * lopata; }

int main() { int x = jama(tulipany()); }
\end{lstlisting}

Kvôli jednoduchosti v ňom nepoužívam \prg!cout!, takže po spustení len do premennej \prg!x!
priradí\footnote{Funkcia \prg!tulipany! nastaví premennú \prg!cibulka! na nulu a pritom ráta hodnotu 
$2^\text{\vb{cibulka}}$. 
Keďže \prg!cibulka!
je na začiatku $4$, tak \prg!tulipany! vráti $16$, čo následne \prg!jama! vynásobí piatimi.} 
$80$ a skončí.
Keď v termináli napíšeš \vb{g++ main.cc -o main} kompilátor vyrobí súbor \vb{main},
ktorý môžeš spustiť \vb{./main} (a nič sa nestane). Ak ale napíšeš \vb{g++ -c main.cc -o main.o} kompilátor vyrobí súbor \vb{main.o}, ktorý obsahuje object code a spustiť nedá.  
Potom môžeš napísať \vb{g++ main.o -o main} a object code z \vb{main.o} sa zlinkuje do súboru
\vb{main}, ktorý sa už spustiť dá. Na súbor \vb{main.o} sa dá pozrieť napr. príkazom
\vb{objdump -Ct main.o}, ktorý vypíše napr. niečo takéto:

\vbox{
\begin{Verbatim}[frame=single]
main.o:     file format elf64-x86-64

SYMBOL TABLE:
0000000000000000 l    df *ABS*	0000000000000000 main.cc
0000000000000000 l    d  .text	0000000000000000 .text
0000000000000000 g     O .data	0000000000000004 cibulka
0000000000000004 g     O .data	0000000000000004 lopata
0000000000000000 g     F .text	000000000000002d tulipany()
000000000000002d g     F .text	0000000000000013 jama(int)
0000000000000040 g     F .text	000000000000001e main
\end{Verbatim}
}

Bez toho, aby sme to rozoberali do detailov vidno, že súbor má dva segmenty:
\vb{.data} a \vb{.text} a v nich obsahuje všetky globálne premenné a
skompilované funkcie uložené za sebou (číslo v prvom stĺpci je adresa). Náš
nápad na zrýchlenie kompilácie by preto bol rozdeliť program do viacerých
súborov, každý z nich prekompilovať do object code a zlinkovať dokopy. Keď sa
potom niečo zmení, bude stačiť prekompilovať ten zmenený súbor a zlinkovať
dokopy s ostatnými, nezmenenými, object súbormi (linkovanie je rýchle).


Spravme si preto súbor \vb{tulipany.cc}, v ktorom bude:

\begin{lstlisting}
int cibulka = 4;

int tulipany() {
  int res = 1;
  while (cibulka-- > 0) res <<= 1;
  return res;
}
\end{lstlisting}

a súbor \vb{jama.cc}, v ktorom bude

\begin{lstlisting}
int lopata = 5;

int jama(int x) {
  return x * lopata;
}
\end{lstlisting}

Teraz môžem napísať \vb{g++ -c tulipany.cc -o tulipany.o} a vyrobí sa súbor \vb{tulipany.o}, ktorý si viem skontrolovať
\vb{objdump -Ct tulipany.o} :

\begin{Verbatim}[frame=single]
tulipany.o:     file format elf64-x86-64

SYMBOL TABLE:
0000000000000000 l    df *ABS*	0000000000000000 tulipany.cc
0000000000000000 l    d  .text	0000000000000000 .text
0000000000000000 g     O .data	0000000000000004 cibulka
0000000000000000 g     F .text	000000000000002d tulipany()
\end{Verbatim}

Rovnako to viem urobiť aj so súborom \vb{jama.cc}. V súbore \vb{main.cc} ale nechcem dávať \prg!#include "tulipany.cc"!, 
to by som bol tam, kde na začiatku, takže mi v ňom ostane

\begin{lstlisting}
int main() { 
  int x = jama(tulipany()); 
}
\end{lstlisting}

Keď teraz ale napíšem \vb{g++ -c main.cc -o main.o}, kompilátor nevyrobí \vb{main.o}, ale začne sa, celkom oprávnene, sťažovať, že jamu ani tulipány nepozná. Toto by bol inak dosť neprekonateľný problém, našťastie C++ má mechanizmus 
tzv. {\em deklarácií}. Čo to je? Ak napíšeš súbor \vb{slon.cc}\indexItem{Prg}{definícia vs. deklarácia}

\begin{lstlisting} 
int slon(int x) {
  return x * 5;
}

int main() {
  int z = slon(3);
}
\end{lstlisting}

je v ňom {\em definícia} funkcie \prg!slon!. Do programu ale môžeš napísať hlavičku funkcie bez tela 
(s bodkočiarkou na konci), čím hovoríš kompilátoru
\cmd{Časom ti definujem takúto funkciu. Ak ju niekde treba použiť, buď v pohode, je to pod kontrolou.} 
Tomuto sľubu sa hovorí {\em deklarácia}.
Samozrejme,
sľub treba dodržať a funkciu časom definovať, ale to sa prejaví až pri linkovaní. Zoberme súbor
\vb{slon.cc}

\begin{lstlisting}
int slon(int x);

int main() {
  int z = slon(3);
}
\end{lstlisting}

Ak napíšem \vb{g++ -c slon.cc -o slon.o}, kompilátor vyrobí súbor \vb{slon.o}, v ktorom je 

\begin{Verbatim}[frame=single]
slon.o:     file format elf64-x86-64

SYMBOL TABLE:
0000000000000000 l    df *ABS*	0000000000000000 slon.cc
0000000000000000 l    d  .text	0000000000000000 .text
0000000000000000 g     F .text	000000000000001c main
0000000000000000         *UND*	0000000000000000 slon(int)
\end{Verbatim}

Všimni si \vb{*UND*} pri symbole \vb{slon(int)}: kompilátor ti uveril, že \vb{slon} nakoniec niekde bude, a do object code
si zaznačil, že zatiaľ ho nenašiel, ale nerobí kvôli tomu paniku. Keby si chcel súbor (aj) zlinkovať (napr.
\vb{g++ slon.o -o slon}), vypíše sa chyba:

\begin{Verbatim}
/usr/bin/ld: /tmp/cc4uVwt3.o: in function `main':
slon.cc:(.text+0xe): undefined reference to `slon(int)'
collect2: error: ld returned 1 exit status
\end{Verbatim}

Teraz linker hovorí, že chcel zlepiť dokopy výsledný súbor, ale slona v žiadnom vstupnom súbore nenašiel. Definíciu
slona môžeš dodať kamkoľvek, napr. ak spravíš súbor \vb{slon.cc} takto:

\begin{lstlisting}
int slon(int x); // ok, deklarácia
int slon(int x); // druhá deklarácia, nie je problém ak je rovnaká

int main() {
  int z = slon(3);
}

int slon(int x) { // definícia
  return x * 2;
}
\end{lstlisting}

Všetko bude v poriadku. Záver z tohoto pozorovania je teda takýto: 

\vskip 2ex 

\centerline{\fcolorbox{magenta}{magenta!3!white}{\parbox{0.9\textwidth}{
\begin{enumerate}\itemsep=-1mm
    \item funkcia môže mať veľa deklarácií (musia byť rovnaké)
    \item funkcia musí mať práve jednu definíciu (inak sa vyrobí object code, ale nezlinkuje sa)
    \item použitiu funkcie v programe musí predchádzať definícia alebo deklarácia
\end{enumerate}
}}}


Naspäť k príkladu s jamou a tulipánom (zmaž všetky súbory \vb{.o}). 
Máme súbory \vb{jama.cc}, \vb{tulipany.cc} a \vb{main.cc}.
Ten zmeníme tak, že pridáme deklarácie príslušných funkcií:

\begin{lstlisting}
int jama(int);
int tulipany();

int main() {
  int x = jama(tulipany());
}
\end{lstlisting}

Teraz treba napísať príkazy:

\begin{Verbatim}[frame=single]
g++ -c jama.cc -o jama.o
g++ -c tulipany.cc -o tulipany.o
g++ -c main.cc -o main.o
g++ main.o jama.o tulipany.o -o main
\end{Verbatim}

a všetko prejde, ako má. Čo ak by som ale chcel v hlavnom programe modifikovať premennú \vb{lopata} (ktorá je v súbore
\vb{jama.cc})? Nemôžem napísať:

\begin{lstlisting}
int jama(int);
int tulipany();

int main() {
  lopata = 10;
  int x = jama(tulipany());
}
\end{lstlisting}

lebo pri kompilovaní \vb{main.cc} a vyhlási chyba, že nepozná premennú \vb{lopata}. Nemôžem napísať ani

\begin{lstlisting}
int jama(int);
int tulipany();

int lopata;

int main() {
  lopata = 10;
  int x = jama(tulipany());
}
\end{lstlisting}

Tým totiž vyrobím (inú) premennú \vb{lopata} v súbore \vb{main.o}. 
Súbor \vb{main.cc} sa skompiluje, ale linker sa bude sťažovať na \vb{multiple definition of `lopata'; }.\indexItem{Prg}{\vb{extern}}
Analógiou deklarácie funkcie je slovíčko \prg!extern! pre premennú: hovorí kompilátoru aby pre ňu nerobil
novú adresu, ale spoľahol sa na to, že časom taká premenná bude definovaná. Výsledný súbor \vb{main.cc}
by teda mohol vyzerať
takto:

\vbox{
\begin{lstlisting}
// odkazy na súbor jama.cc
extern int lopata;
int jama(int);

// odkazy na súbor tulipany.cc
extern int cibulka;
int tulipany();

// hlavný program
int main() {
  lopata = 10;
  cibulka = 7;
  int x = jama(tulipany());
}
\end{lstlisting}
}

Aby sa udržal poriadok, je zvykom deklarácie z nejakého súboru (napr. \vb{jama.cc}) dať do rovnako sa volajúceho 
súboru s príponou \vb{.h} ako {\em header} (napr. \vb{jama.h}) a ten potom vložiť pomocou \prg!#include! všade, kde ho je treba.
Takže do súboru \vb{jama.h} napíšem

\vskip 2ex
\vbox{\begin{lstlisting}
extern int lopata;
int jama(int);
\end{lstlisting}}
do súboru \vb{tulipany.h} napíšem

\vskip 2ex
\vbox{\begin{lstlisting}
extern int cibulka;
int tulipany();
\end{lstlisting}}
a výsledný program bude

\vskip 2ex
\vbox{\begin{lstlisting}
#include "jama.h"
#include "tulipany.h"

int main() {
  lopata = 10;
  cibulka = 7;
  int x = jama(tulipany());
}
\end{lstlisting}}


Metódy tried sú funkcie ako každé iné, a preto pre ne platia rovnaké pravidlá o
deklaráciách a definíciách, ako pre normálne funkcie. Treba len myslieť na to,
že ak sa na metódu odvolávame mimo definície jej triedy, treba ju volať jej
celým menom, t.j. pred menom metódy treba vždy uviesť (s dvomi dvojbodkami)
triedu, ktorej patrí. Porovnaj si tieto dva zápisy

\begin{column}{0.45}
\begin{lstlisting}
struct Bod {
  int x, y;
  int sucet() { return x + y; }
};
\end{lstlisting}
\end{column}
\hfill
\begin{column}{0.45}
\begin{lstlisting}
struct Bod {
  int x, y;
  int sucet();
};

int Bod::sucet() { return x + y; }
\end{lstlisting}
\end{column}

Doteraz sme všetky metódy tried písali tak, ako na príklade vľavo. Oba zápisy robia to isté, ale
je tam drobný rozdiel, ktorý som ti doteraz nespomenul: 
zápis vľavo považuje metódu \vb{sucet} za tzv. {\em inline}: kompilátor ju \indexItem{Prg}{inline metóda}
neskompikuje priamo, ale vždy, keď sa v programe vyskytne jej volanie, kompilátor ho nahradí
programom z definície\footnote{%
  v niektorých prípadoch je to rýchlejšie, ale rastie veľkosť skompilovaného programu}.
Preto zápis vľavo môže byť v header súbore, ale zápis vpravo musíme riešiť rovnako,
ako pri ostatných funkciách.

Zvykom je v definícii typu (triedy) napísať inline iba krátke malé metódy,
pri ktorých nevadí, že sa v skompilovanom programe objavia veľakrát.
Pri ostatných metódach napísať v headri iba ich deklarácie a ich definície písať neskôr. 


Skúsme si to ukázať na trochu zložitejšom príklade.  Vyrob si tri súbory. Prvý bude \vb{vec.h} a bude v ňom

\begin{lstlisting}
#ifndef __vec_h__
#define __vec_h__
struct Vec {
  struct Proxy {          // tento typ sa celým menom volá Vec::Proxy
    Proxy(Vec *a);        // konštruktor Vec::Proxy (deklarácia)
    Proxy &operator++();  // zvýši hodnotu x a vráti referenciu na seba (deklarácia)
    private:
    int *x;               // x je privátne pre Vec::Proxy
  };
  Vec(int x);          // konštruktor Vec (deklarácia)
  int &operator()();   // vráti referenciu na mojaVec (deklarácia)
  Proxy vyrobProxy();  // vyrobí Vec::Proxy, ktorého x ukazuje na mojaVec (deklarácia)
  private:
  int mojaVec;         // privátna hodnota pre Vec
};
#endif
\end{lstlisting}

Je v ňom definovaný typ \vb{Vec}, ktorý má privátnu premennú \vb{mojaVec}, má konštruktor s parametrom \prg!int!,
\prg!operator()()! (t.j. operátor \vb{()} bez parametrov) a metódu \vb{vyrobProxy}. Všetky tri sú iba deklarované,
definície budú v inom súbore. Okrem toho má typ \vb{Vec} svoj vlastný typ \vb{Proxy} (podobne ako iterátory v STL),
ktorý sa celým menom volá \vb{Vec::Proxy}. Ten má svoj vlastný konštruktor a (prefixový) operátor \vb{++}. Tieto
sú tiež v súbore \vb{vec.h} iba deklarované a bude treba ich definovať. Definície budú v súbore \vb{vec.cc}:

\vskip 2ex
\begin{lstlisting}
#include "vec.h" // include treba, aby kompilátor vedel o typoch Vec a Proxy
                 // keď vyrába object code pre vec.cc
Vec::Proxy::Proxy(Vec *a) { x = &a->mojaVec; } // konštruktor pre Vec::Proxy
                                               // x je pointer na a-čkovu mojaVec
Vec::Vec(int x) { mojaVec = x; } // konštruktor pre Vec
int &Vec::operator()() { return mojaVec; }

// zavolaj konštruktor Proxy a daj mu ako parameter pointer na seba
Vec::Proxy Vec::vyrobProxy() { return Proxy(this); }

Vec::Proxy &Vec::Proxy::operator++() {
  (*x)++;
  return *this;
}
\end{lstlisting}

Zápis \vb{Vec::Proxy::Proxy(Vec *a)} treba chápať takto: \vb{Vec} je typ, v ňom je typ \vb{Vec::Proxy}. Hocijaká funkcia \vb{fun}, ktorá
by patrila typu \vb{Vec::Proxy} sa preto celým menom volá \vb{Vec::Proxy::fun}. Konštruktor je metóda, ktorá sa volá rovnako, ako typ
(v našom prípade \vb{Proxy}). Preto celé meno konštruktora je \vb{Vec::Proxy::Proxy}.
A napokon hlavný súbor \vb{main.cc}:

\begin{lstlisting}
#include "vec.h"
#include <iostream>
using namespace std;
int main() {
  Vec v(42);                // konštruktor spraví Vec v, kde v.mojaVec bude 42
  cout << v() << endl;      // v() vráti referenciu na v.mojaVec, vypíše sa 42
  v() += 5;                 // v() je referencia, takže viem v.mojaVec meniť, 
                            // aj keď mojaVec je private
  cout << v() << endl;      // vypíše sa 47
  auto p = v.vyrobProxy();  // p je Vec::Proxy, jeho x je pointer na v.mojaVec
  ++++p;                    //++p inkrementuje v.mojaVec a vráti referenciu na seba
                            // preto ++++p je ++(++p)
  cout << v() << endl;      // vypíše sa 49
}
\end{lstlisting}

Teraz môžeš napísať

\begin{Verbatim}[frame=single]
g++ -c vec.cc -o vec.o
g++ main.cc vec.o -o main
\end{Verbatim}

a po spustení \vb{./main} by sa mali vypísať čísla $42, 47$ a $49$ (všimni si, že pri jednom volaní kompilátora môžem 
zadať viacero vstupných súborov, ktoré môžu byť zmes programov a object code súborov).


Máme hotový mechanizmus, ako udržovať v programe poriadok a zároveň obmedziť čas kompilácie: časti programu, ktoré
logicky patria k sebe (napr. niekoľko tried, ktoré robia jadnu vec a je šanca, že ich použijeme aj inokedy), dáme 
dokopy: typy a deklarácie idú do súboru \vb{.h} a príslušné definície do súboru \vb{.cc}. Súbor \vb{.cc} skompilujeme
do object code, súbor \vb{.h} vložíme pomocou \prg!#include! kam treba. Ak sa niečo zmení, prekompilujeme iba príslušnú časť
a všetko zlinkujeme dokopy. Jediná nepraktická vec je, že na vyrobenie programu musíme veľakrát spúšťať kompilátor s rôznymi parametrami,
a musíme si pamätať, čo sa zmenilo a čo treba prekompilovať. Ako toto celé riešiť automaticky si ukážeme o chvíľu, 
predtým jedno dôležité upozornenie.


Dôležité upozornenie: celý tento mechanizmus, ktorý sme si ukázali, nefunguje pre šablóny. \indexItem{Prg}{šablóny a header súbory}
Opäť si to ukážeme na príklade. Zober si nasledovný súbor, v ktorom je trieda \vb{Dvojica}, ktorá má 
dve premenné nejakého typu a vie ich vymeniť:

\begin{column}{0.45}
\vbox{\begin{lstlisting}
template <typename T>
struct Dvojica {
  T x, y;
  void vymen();
};

template <typename T>
void Dvojica<T>::vymen() {
  T tmp = x;
  x = y;
  y = tmp;
}
\end{lstlisting}}
\end{column}\hfill\begin{column}{0.45}
\vbox{\begin{lstlisting}
#include <iostream>
using namespace std;
int main() {
  Dvojica<int> d;
  d.x = 3;
  d.y = 4;
  d.vymen();
  cout << d.x << " " << d.y << endl;
}
\end{lstlisting}}\end{column}

V prvom rade si všimni, akým spôsobom zapisujeme metódu \vb{vymen}: nestačí písať \prg!void Dvojica::vymen()!, lebo \vb{Dvojica} sama osebe nie je typ.
Aj pred definíciou treba pridať príslušnú šablónu, aby sme mohli napísať celý typ, ktorý je \prg!Dvojica<T>!. Kým je celý program v jednom súbore, všetko funguje.
Skús ho teraz rozdeliť na tri súbory, ako sme to robili pred chvíľou: šablóna typu \vb{Dvojica} pôjde do súboru \vb{dvojica.h}, definícia metódy \vb{vymen} do súboru
\vb{dvojica.cc} a zvyšok ostane v \vb{main.cc} (s tým, že \vb{main.cc} aj \vb{dvojica.cc} budú mať \prg!#include "dvojica.h"!).
Keď skúsiš súbory skompilovať tak, ako doteraz, t.j

\begin{Verbatim}[frame=single]
g++ -c dvojica.cc -o dvojica.o
g++ -c main.cc -o main.o
g++ dvojica.o main.o -o main
\end{Verbatim}


linker sa zrazu bude sťažovať, že 

\begin{Verbatim}
/usr/bin/ld: main.o: in function `main':
main.cc:(.text+0x1e): undefined reference to `Dvojica<int>::vymen()'
collect2: error: ld returned 1 exit status
\end{Verbatim}

Čo sa stalo? Odpoveď dá \vb{objdump -Ct dvojica.o}:

\vbox{
\begin{Verbatim}[frame=single]
dvojica.o:     file format elf64-x86-64

SYMBOL TABLE:
0000000000000000 l    df *ABS*	0000000000000000 dvojica.cc
\end{Verbatim}
}


Súbor \vb{dvojica.o} neobsahuje žiadne funkcie. Prečo? V súbore \vb{dvojica.cc} nie je zapísaná žiadna funkcia, iba šablóna, ako príslušnú funkciu vyrobiť, keď ju bude treba.
Ale nie je tam žiadne miesto, kde by bolo treba vytvoriť funkciu \prg!Dvojica<int>::vymen()!, takže sa nevytvorí. Bolo by ju treba až v súbore \vb{main.o}, ale to zistí
až linker, keď je už neskoro. Keby si do súboru \vb{dvojica.cc} pridal funkciu, ktorá prinúti kompilátor šablónu použiť, napr.:

\begin{lstlisting}
#include "dvojica.h"
template <typename T>
void Dvojica<T>::vymen() {
  T tmp = x;
  x = y;
  y = tmp;
}

void hlupaFunkcia() {
  Dvojica<int> d;
  d.vymen();
}
\end{lstlisting}

všetko by už fungovalo (skontroluj si  \vb{objdump -Ct dvojica.o}). Samozrejme, takéto riešenie je zlé, 
preto pri šablónach je zvykom aj ich definície\footnote{
  Treba si ale dať pozor pri špecializácii. Keď máš napr. šablónu 

\begin{lstlisting}
template <typename T>
struct Vec {  void daj(); };
\end{lstlisting}

tak definícia jej metódy

\begin{lstlisting}
template <typename T>
void Vec<T>::daj() { ...  }
\end{lstlisting}

  je šablóna a vyrobí sa, až keď je jasné, s akým typom \vb{T} je treba.
  Preto patrí do súboru \vb{.h}. Ale keď tú šablónu plne špecializuješ 

\begin{lstlisting}
template <>
struct Vec<int> { void daj(); };
\end{lstlisting}
  
tak jej metóda (všimni si, že nepíšem prázdne \prg!template<>! : toto nie je 
špecializácia metódy, ale metóda špecializovanej triedy)

\begin{lstlisting}
void Vec<int>::daj() { ...  }
\end{lstlisting}

je už normálna funkcia ako každá iná a kompilátor ju vyrobí, len čo ju zbadá. Ak ju dáš do súboru
\vb{.h}, tak sa kompilátor môže sťažovať na viacnásobnú definíciu.
}písať do súboru \vb{.h}.


 Tak a teraz sa môžeme dostať k druhej časti tejto kapitoly, a to ako
automatizovať, ktoré súbory sa kedy majú prekompilovať. To sa veľmi dobre dá
pomocou programu \vb{make}.  \vb{make} je program, ktorý vie vykonať nejaké
akcie (napr. zavolať kompilátor), ak sú splnené nejaké podmienky (napr. súbor
treba prekompilovať, lebo sa zmenil). Väčšinou sa \vb{make} používa na kompilovanie programov, ale dá
akcie môžu byť ľubovoľné\footnote{\vb{make} napríklad používam aj pri
generovaní tohoto textu}. Je to veľmi flexibilný a silný nástroj, 
tu
si z neho ukážeme iba zopár vecí, ktoré budeme používať.


Ak spustíš \vb{make} bez parametrov, bude hľadať v aktuálnom\indexItem{Alg}{\vb{Makefile}}
adresári súbor \vb{Makefile} a z neho bude čítať pravidlá. Pravidlo má tvar

\begin{Verbatim}
cieľ: prerekvizity
  akcia
  ...
\end{Verbatim}

V najjednoduchšej forme je \vb{ciel} názov súboru, ktorý treba vyrobiť, ak sa
niektorý zo súborov uvedených v časti \vb{prerekvizity} zmení. \vb{akcia} hovorí,
čo treba urobiť na vyrobenie cieľa. Akcia môže mať viacero riadkov, ale každý
musí začínať tabulátorom. Zoberme si opäť príklad so súbormi \vb{main.cc},
\vb{jama.cc}, \vb{jama.h}, \vb{tulipany.cc}, \vb{tulipany.h}. Najjednoduchší
\vb{Makefile} by vyzeral takto:


\vbox{
\begin{Verbatim}[frame=single]
main: main.o jama.o tulipany.o
  g++ main.o jama.o tulipany.o -o main

jama.o: jama.cc jama.h
  g++ -c jama.cc -o jama.o

tulipany.o: tulipany.cc tulipany.h
  g++ -c tulipany.cc -o tulipany.o

main.o: main.cc tulipany.h jama.h
  g++ -c main.cc -o main.o
\end{Verbatim}
}

Ak spustíš \vb{make} bez parametrov, zoberie si za hlavný cieľ prvý v poradí.
Konkrétny cieľ môžeš dať ako parameter pri spustení, takže v našom prípade je
\vb{make} a \vb{make main} to isté. \vb{make} zistí, že na vyrobenie súboru
\vb{main} treba mať aktuálne súbory \vb{main.o} \vb{jama.o} a \vb{tulipany.o}.
Najprv ide skontrolovať prerekvizity. Pozrie sa, či je aktuálny súbor
\vb{main.o}. Zistí, že taký súbor neexistuje, ale existuje cieľ, ktorým sa dá
vyrobiť. Prerekvizity cieľa \vb{main.o} sú \vb{main.cc}, \vb{tulipany.h} a
\vb{jama.h}, ktoré všetky existujú, takže sa vykoná akcia \vb{g++ -c main.cc -o
main.o} a súbor \vb{main.o} je v poriadku. Podobne sa prejdú ciele \vb{jama.o}
a \vb{tulipany.o} a vyrobia sa príslušné súbory. Keď sú prerekvizity cieľa
\vb{main} v poriadku, vykoná sa akcia \vb{g++ main.o jama.o tulipany.o -o main}
a hotovo (vyskúšaj si to).  Ak napíšeš \vb{make} znovu, znovu sa prečíta
\vb{Makefile} a skontrolujú sa ciele. Teraz všetky súbory existujú, preto
\vb{make} iba skontroluje čas modifikácie\footnote{Operačný systém si pri
každom súbore vedie rôzne štatistiky, okrem iného aj čas poslednej modifikácie.
\vb{make} podľa toho vie rozpoznať, ktorý súbor sa zmenil.}. Postupne zistí, že
\vb{main.o} bol modifikovaný neskôr ako \vb{main.cc}, \vb{tulipany.h} aj
\vb{jama.h}, takže akciu pre aktualizáciu cieľa \vb{main.o} netreba robiť. 
Rovnako to dopadne aj pre ostatné ciele, nakoniec sa zistí, že aj súbor \vb{main}
je aktuálny, a netreba nič kompilovať. Ak treaz zmeníš napr. súbor \vb{jama.cc} a spustíš \vb{make},
ciele \vb{tulipany.o} aj \vb{main.o} budú aktuálne, takže sa iba prekompiluje \vb{jama.o} 
a všetko sa zlinkuje dokopy. Ak to zhrnieme: pravidlo v \vb{Makefile} hovorí, že ak
sa zmení niektorá z prerekvizít, treba aktualizovať cieľ pomocou príslučnej akcie.


Ciele sa chápu ako názvy súborov, ale dá sa to využiť aj inak. Na koniec \vb{Makefile} si dopíš

\begin{Verbatim}[frame=single]
clean:
  rm -f *.o
  rm -f main
\end{Verbatim}

Čo sa stane, ak zavoláš \vb{make clean}? Skontroluje sa cieľ \vb{clean} a keďže taký súbor neexistuje, \vb{make} sa ho pokúsi
vyrobiť príslušnou akciou (prerekvizity žiadne nie sú, takže sa nič nekontroluje). Lenže tá akcia nevytvorí súbor \vb{clean}, iba 
zmaže všetky vygenerované súbory. \vb{make clean} sa teraz dá použiť na upratovanie: kedykoľvek ho napíšeš, upracú sa zbytočné súbory.
Toto je fajn, ale ak by sa v adresári nejak ocitol súbor, ktorý by sa volal \vb{clean}, prestane to fungovať. \vb{make} má pre tento účel
špeciálny cieľ \vb{PHONY.}, ktorý hovorí, že všetky jeho prerekvizity treba aktualizovať, aj keby existovali. Preto sa zvykne písať

\begin{Verbatim}[frame=single]
.PHONY: clean
clean:
  rm -f *.o
  rm -f main
\end{Verbatim}


Ak máš projekt, v ktorom je veľa súborov, začne byť ručné písanie pravidiel pre každý súbor neprehľadné. \vb{make} má preto možnosi\footnote{%
  Tých možností je ozaj veľa, odporúčam pozrieť 
pozrieť manuál: \href[pdfnewwindow=true]{https://www.gnu.org/software/make/manual/make.html}{www.gnu.org/software/make/manual/make.html}} ako veci automatizovať.
Jendou z nich sú premenné: premenná v \vb{Makefile} môže obsahovať postupnosť reťazcov. Premenná sa vyrobí tak, že sa do nej priradí \vb{meno = hodnota}
a pristupuje sa k nej cez znak \vb{\$}, t.j. \vb{\$(meno)}. Napr.

\begin{Verbatim}[frame=single]
comp = g++ -O3
sources = $(wildcard *.cc)
objects = $(sources:.cc=.o)

main: $(objects)
  $(comp) $(objects) -o main
\end{Verbatim}

vyrobí premennú \vb{comp}, v ktorej bude \vb{``g++ -O3''}, premennú \vb{sources}, v ktorej bude výsledok príkazu \hbox{\vb{wildcard *.cc}}\footnote{To je vstavaný príkaz v \vb{make},
ktorý vráti zoznam všetkých súborov v aktuálnom adresári, ktorých mená sa končia na \vb{.cc} (hviezdička tradične znanemá \cmd{hocičo})}, t.j. \vb{``main.cc tulipany.cc jama.cc''}
a premennú \vb{objects}, ktorá obsahuje\footnote{opäť špeciálny zápis, ktorý \vb{make} poskytuje} všetky reťazce z \vb{\$(sources)}, ale koncovka \vb{``.cc''} sa nahradí \vb{``.o''},
t.j \vb{``main.o tulipany.o jama.o''}. Cieľ pre \vb{main} teda funguje rovnako, ale keď sa pridá ďalší súbor, netreba meniť \vb{Makefile}. 
Ostatné pravidlá vyzerajú všetky podobne: na vyrobenie súboru \vb{.o} treba skompilovať súbor \vb{.c}. V \vb{Makefile} sa na to dajú použiť tzv. {\em pattern rules}, t.j. šablóny 
pravidiel, v ktorých vystupuje znak \vb{\%}: pri hľadaní vhodného pravidla sa zaňho môže dosadiť čokoľvek. Napr. \vb{Makefile} z predchádzajúceho rámčeka by sme mohli doplniť

\begin{Verbatim}[frame=single]
comp = g++ -O3
sources = $(wildcard *.cc)
objects = $(sources:.cc=.o)

main: $(objects)
  $(comp) $(objects) -o main

%.o:  %.cc
  $(comp) -c $< -o $@
\end{Verbatim}

Posledné pravidlo hovorí, že ak treba vyrobiť hocijaký súbor \vb{kvik.o}, treba skontrolovať prerekvizitu \vb{kvik.cc} a ak treba, skompilovať ho. Premenná \vb{\$<} je
špeciálna premenná, ktorá označuje prvú prerekviciztu (napr. \vb{kvik.cc}) a \vb{\$@} označuje cieľ, t.j. \vb{kvik.o}. 
Toto bude fungovať, až na to, že sa neberú do úvahy súbory \vb{.h}. Ak napr. zmeníš \vb{jama.cc}, správne sa prekompiluje \vb{jama.o} a zlinkuje sa výsledok, ale
ak zmeníš \vb{jama.h}, nestane sa nič. Toto je hlbší problém, pretože by som chcel pre každý súbor \vb{.cc} zistiť, ktoré súbory si vkladá pomocou \prg!#include!, a dať príslušné
prerekvizity. Práve kvôli tomuto má \vb{g++} prepínač \vb{-MM}, ktorý vypíše závislé súbory. Ak napíšeš \vb{g++ -MM main.cc}, vypíše sa 

\begin{Verbatim}
main.o: main.cc jama.h tulipany.h
\end{Verbatim}

teda presne to, čo by som chcel mať v hlavičke pravidla v \vb{Makefile}. To viem použiť spolu s mechanizmom \vb{include} v \vb{Makefile} na to, aby som spravil niečo takéto

\begin{Verbatim}[frame=single]
sources  = $(wildcard *.cc)
objects  = $(sources:.cc=.o)
depends  = $(sources:.cc=.d)

main: $(objects)
  g++ $(objects) -o $@

%.d: %.cc
  g++ -MM $< -o $@

include $(depends)

%.o: %.cc
  g++ -c $< -o $@
\end{Verbatim}

Poďme sa pozrieť, čo tieto pravidlá hovoria. Okrem toho, čo sme tu už mali, si pre každý súbor chcem vyrobiť súbor \vb{.d} (ich názvy mám v premennej \vb{depends}).
Mám pravidlo, ktoré mi hovorí, ako vyrobiť súbor \vb{.d} zo súboru \vb{.cc}. Riadok \vb{include \$(depends)} znamená, že na toto miesto treba vložiť obsah súborov
\vb{main.d}, \vb{jama.d} a \vb{tulipany.d}. Ak tie súbory neexistujú, vyrobia sa podľa dostupého pravidla. Keď existujú, vložia sa tam a budú dávať dodatočné prerekvizity
pre jednotlivé \vb{.o} súbory\footnote{Po vložení bude mať napr. súbor \vb{main.o} dve pravidlá: jedno, ktoré vzniklo vložením súboru \vb{main.d}, a
kde má prerekvizity \vb{main.cc jama.h tulipany.h}, ale žiadnu akciu
a druhé, kde sa použije \vb{\%.o: \%.cc}. Pri spracovaní \vb{make} funguje tak, že pri viacerých pravidlách pre jeden cieľ vyžaduje, aby práve jedno malo nastavenú akciu.
Potom pozbiera všetky prerekvizity do jednej a ak treba, akciu urobí.}
Skús si tento \vb{Makefile} vyskúšať tak, že budeš meniť rôzne veci v rôznych súboroch a sledovať, kedy sa čo prekompiluje. Je tam ešte jeden drobný problém. Nájdeš ho?


Ak si ho našiel, gratulujem. Problém je v tom, že súbor \vb{.d} závisí iba od príslušného súboru \vb{.cc}. Dá sa to vidieť na príklade. Vyrob si nejaký súbor \vb{krtko.h}.
Po tom, čo spustíš \vb{make}, pridaj do \vb{jama.h} riadok \prg!#include "krtko.h"!. Zavoláš \vb{make} a výsledný súbor sa prekompiluje, ale súbor \vb{jama.d} sa nezmení.
Keď teraz zmeníš súbor \vb{krtko.h}, nič sa neprekompiluje\footnote{Všimni si, že keby si riadok \prg!#include "krtko.h"! pridal do \vb{jama.cc}, prerobí sa aj súbor
\vb{jama.d} a všetko bude v poriadku.}. Aby sme to opravili, pravidlo pre generovanie súborov \vb{.d} upravíme tak, aby pridalo závislosti aj pre samotný súbor \vb{.d},
takže napr. namiesto riadku \vb{jama.o: jama.cc jama.h} bude \vb{jama.d jama.o: jama.cc jama.h}. A úplne posledná vec: je prehľadnejšie, ak sa generované súbory nemiešajú s tými, ktoré
píšeme. Takže si spravíme adresár (napr. \vb{build}) a všetky dočasné súbory budeme umiestňovať tam. Výsledný \vb{Makefile} by preto mohol vyzerať takto:

\vbox{
\begin{Verbatim}[frame=single]
sources  = $(wildcard *.cc)
builddir = ./build
objects  = $(addprefix $(builddir)/,$(sources:.cc=.o))
depends  = $(objects:.o=.d)

main: $(objects)
  mkdir -p $(builddir)
  g++ $(objects) -o $@

$(builddir)/%.d: %.cc
  mkdir -p $(builddir)
  echo -n $@" "$(builddir)/> $@
  g++ -MM $< >> $@

include $(depends)

$(builddir)/%.o: %.cc
  mkdir -p $(builddir)
  g++ -c $< -o $@

PHONY.: clean
clean:
  rm -rf $(builddir)
\end{Verbatim}
}


Príkaz \vb{mkdir -p} urobí nový adresár, ale iba ak ešte neexistuje. Príkaz \vb{echo -m oznam} vypíše \vb{oznam} (\vb{-n} hovorí, aby na koniec nedal \prg!endl!).
Presmerovanie \prg~>>~ znamená \cmd{Pripoj výstup z programu na koniec súboru.}


Po odbočke k organizácii projektu sa vráťme k hre \btr a urobme si prípravné práce. Políčko budeme mať zapamätané ako dvojicu {\em stĺpec} ({\em file}, \vb{f})
a {\em riadok} ({\em rank}, \vb{r}). A ťah si budeme pamätať ako dvojicu políčok {\em odkiaľ}, {\em kam}. To nám dáva prirodzene triedy

\begin{lstlisting}
struct Square {
  uint8_t f, r;
  bool legal();
};

struct Move {
  Square from, to;
};
\end{lstlisting}

s tým, že do \vb{Square} som pridal metódu \vb{legal}, ktorá skontroluje rozsah (od 0 do 7). 
Premenné \vb{f,r} som definoval s typom\indexItem{Prg}{číselné typy \vb{int8\_t}, \vb{uint32\_t} a spol.}
\prg!uint8_t!. S takým typom sme sa ešte nestretli, ale je to jeden z číselných
typov. Okrem typov ako \prg!int!, \prg!unsigned long! a pod., ktorých veľkosť
závisí od systému, sú aj názvy typov s fixnou dĺžkou. Takže \prg!uint8_t!
znamená 8-bitový \vb{unsigned int}, t.j. typ, ktorý vždy zaberá 1 byte a dajú
sa v ňom pamätať čísla $0\ldots255$. Podobne \prg!int32_t! je 32-bitový
\vb{int} so znamienkom, \prg!uint64_t! 64-bitový  \vb{unsigned int} a pod.
Tieto typy sú definované v knižnici \prg!<cstdint>!, takže keď ich chceš používať,
treba zavolať \prg!#include <cstdint>!.

Šachovnicu si najjednoduchšie môžeme pamätať ako pole $8\times 8$ políčok, kde 0 znamená biely,
1 čierny a 2 prázdne políčko. Ale zhodou okolností šachovnica má presne 64 políčok, tak sa poďme pocvičiť v bitových operáciách:
budeme mať bitovú mapu pre biele a čierne kamene, \prg!uint64_t pmap[2]! (\prg!pmap[0]! pre biele a \prg!pmap[1]! pre čierne). 
Obidve sú 64-bitové čísla, pričom príslušný bit je nastavený na 1, ak je na danom políčku kameň tej správnej farby.
Ak máme políčko na stĺpci ({\em file}) \vb{f} a riadkku ({\em rank}) \vb{r} (obidve v rozsahu 0 až 7), tak číslo bitu môže byť 
\prg!(f<<3)|r!\footnote{hodnotu \vb{f} posuniem o tri bity doľava a na uvoľnené tri bity dám hodnotu \vb{r}, čo je to isté, ako
$8\cdot f+r$, pretože \vb{r} je nanajvýš 7}.
Môžem si spraviť funkciu \vb{idx(f,r)}, ktorá vráti 64-bitové číslo, ktoré má nastavený iba bit zodpovedajúci políčku \vb{(f,r)},
t.j. $2^{8\cdot f + r}$, resp. \prg!(uint64_t)1 << ((uint8_t)((f << 3) | r))!.



Od triedy \vb{board} by sme nateraz čakali niečo takéto:

\begin{lstlisting}
const uint8_t White = 0;
const uint8_t Black = 1;
const uint8_t Empty = 2;

struct Board {
  Board();  // konštruktor - nastaví začiatočnú pozíciu

  uint64_t idx(int f, int r) const;     // bit v mape zodpovedajúci políčku
  uint8_t get(int f, int r) const;      // kto má kameň na políčku
  void set(int f, int r, uint8_t val);  // nastav políčko

  int winner() const;                    // má už pozícia víťaza?
  bool legal(Move m) const;              // je m prípustný ťah?
  std::vector<Move> legalMoves() const;  // vráť všetky prípustné ťahy
  Board &operator+=(Move m);             // urob ťah m

  int ply;           // číslo polťahu
  uint64_t pmap[2];  // mapa kameňov bieleho a čierneho
};
\end{lstlisting}


 

Okrem mapy šachovnice (\vb{pmap}) si chceme pamätať {\em polťah} ({\em
ply}): po každom potiahnutí sa zvýši, takže ak je na ťahu biely, je \vb{ply}
párne, ak je na ťahu čierny, je nepárne.  Pravidlá hry sú schované v metódach
\vb{winner}, a \vb{legal}: \vb{winner} má vrátiť \vb{White}, \vb{Black}, alebo
\vb{Empty}, podľa toho, či je víťaz alebo nie (vtedy vráti \vb{Empty}; všimni
si, že remíza nemôže nastať). Metóda \vb{legal} má povedať, či je daný ťah v danej pozícii
prípustný. Na začiatok sa budú hodiť ešte dve metódy: \vb{legalMoves} vráti zoznam všetkých prípustných 
ťahov z danej pozície a \prg!operator+=! urobí ťah.


Konštruktor má nastaviť hrací plán na začiatok hry, t.j.\\

\centerline{
%\tikzset{external/force remake}
\begin{tikzpicture}[scale=0.65]
  \boardBg
  \boardGrid
  \wPs{00,10,20,30,40,50,60,70,01,11,21,31,41,51,61,71}
  \bPs{07,17,27,37,47,57,67,77,06,16,26,36,46,56,66,76}
\end{tikzpicture}
}

To môžem urobiť buď tak, že veľakrát zavolám \vb{set}, ale môžem napísať aj
\prg!pmap[0]=0x303030303030303ULL;! \prg!pmap[1]=0xc0c0c0c0c0c0c0c0ULL;!.  Čo
je toto za zápis? \vb{0x} na začiatku čísla znamená, že je to číslo v
šestnástkovej sústave\footnote{O $16$-kovej sústave pozri poznámku na
str.~\pageref{foot:hexa}; za chýbajúce cifry používame \vb{a = 10}, \vb{b =
11}, \vb{c = 12}, \vb{d =13}, \vb{e = 14} a \vb{f = 15}, na veľkosti nezáleží,
takže \vb{e} aj \vb{E} je $14$}. Prípona \vb{ULL} na konci znamená, že je to
\prg!unsigned long long!.  Ako sme vraveli v poznámke pri bitových operáciách
na str.~\pageref{foot:hexa2}, prevod šestnástkovej sústavy do dvojkovej je
príjemný v tom, že jedna šestnástková cifra sú presne 4 dvojkové, takže viem
prevádzať cifru po cifre: \vb{0x03} je v dvojkovej \vb{00000011}, \vb{0x0303}
je \vb{0000001100000011} atď.  Podobne \vb{0xc0} je \vb{11000000}, \vb{0xc0c0}
je \vb{1100000011000000} atď.


Teraz ešte môžeme pridať funckie

\begin{lstlisting}
std::ostream &operator<<(std::ostream &, const Square &);
std::ostream &operator<<(std::ostream &, const Move &);
std::ostream &operator<<(std::ostream &, const Board &);
\end{lstlisting}

aby sa nám veci pekne vypisovali. Všimni si, že v deklaráciách stačí pri parametroch napísať typ,\indexItem{Prg}{deklarácia funkcie bez názvu parametra}
netreba písať aj meno parametra (to treba až vtedy, keď sa ten parameter má použiť v tele funkcie).

\def\ulohaTarget{Dir}
\begin{uloha}
  \label{uloha:btr1}
  Priprav si projekt s \vb{Makefile}, súborom \vb{board.h}, v ktorom budú definície, čo sme tu napísali a súborom \vb{board.cc}, v ktorom budú
  príslušné definície funkcií. Urob jednoduchý \vb{main.cc}, ktorý vypíše pozíciu zo strany \pageref{pg:breaktru:def} a pod ňu všetky možné ťahy bieleho:
\end{uloha}

\begin{outputBox}
    a   b   c   d   e   f   g   h
  +---+---+---+---+---+---+---+---+
8 |   |   |   |   |   |   |   |   | 8
  +---+---+---+---+---+---+---+---+
7 |   |   |   |   |   | O |   | O | 7
  +---+---+---+---+---+---+---+---+
6 |   | X |   |   |   |   | X |   | 6
  +---+---+---+---+---+---+---+---+
5 |   |   |   |   |   |   |   |   | 5
  +---+---+---+---+---+---+---+---+
4 |   |   |   |   |   |   |   |   | 4
  +---+---+---+---+---+---+---+---+
3 |   | X | X |   |   | O | O |   | 3
  +---+---+---+---+---+---+---+---+
2 |   | X |   |   |   |   | X |   | 2
  +---+---+---+---+---+---+---+---+
1 |   |   |   |   |   |   |   |   | 1
  +---+---+---+---+---+---+---+---+
    a   b   c   d   e   f   g   h

b2-a3  b3-a4  b3-b4  b3-c4  b6-a7  
b6-b7  b6-c7  c3-b4  c3-c4  c3-d4  
g2-f3  g2-h3  g6-f7  g6-g7  g6-h7
\end{outputBox}



