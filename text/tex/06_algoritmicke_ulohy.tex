\chapter{Algoritmické úlohy a príkazový riadok}

Všetky úlohy, ktoré si doteraz robil, vyzerali podobne: používateľ niečo napíše na vstupe
(napríklad 7 čísel väčších ako 3),
program s tým má urobiť presne popísanú činnosť (napríklad nájsť najväčšie číslo) a výsledok 
vypísať. Takýmto úlohám, ktoré zo vstupu robia výstup podľa presne stanovených pravidiel
sa hovorí {\em algoritmické úlohy}. Sú dôležité preto, lebo pri nich vieme presne
povedať, kedy je program správny: musí vypísať správny výstup {\em pre všetky} možné vstupy.
Keby sme napr. v úlohe o hľadaní najväčšieho čísla mali program, ktorý skoro vždy
napíše skutočne najväčšie číslo, ale ak je náhodou jedno z čísel na vstupe 12343, 
tak program sa zacyklí,
tak takýto program by bol zlý (aj keby väčšinou fungoval správne).
Aj keď programuješ väčší program, napr. hru, je dobré si ho rozdeliť na menšie, algoritmické,
časti, ktoré sa dajú samostatne skontrolovať. 

Ako sa dajú algoritmické úlohy kontrolovať? Väčšinou nie je možné spustiť program
na všetky možné vstupy, lebo ich je priveľa. Keďže majú presne definované, ako má vyzerať
výstup, je možné o nejakom programe aj matematicky dokázať, že je správny. To býva ale
ťažké, preto sa väčšinou programy {\em testujú}. Snažíme sa vybrať rôzne vstupy a na nich
program spustíme. Keď na všetkých vstupoch dáva správne výsledky, veríme, že je dobrý.
Nájsť dobrú sadu testovacích vstupov býva samo osebe dosť náročné. Je veľa miest,
kde sa dajú algoritmické úlohy trénovať (napr. 
\link{https://testovac.ksp.sk}{testovac.ksp.sk}, \link{https://www.spoj.com}{SPOJ},
%\link{https://codechef.com}{CodeChef}, 
\link{https://codeforces.com}{CodeForces}
a ďalšie). Tieto stránky majú úlohy s možnosťou testovania: odošleš svoj program, na
stránke sa skompiluje, spustí na veľa rôznych vstupoch a oznámi sa ti výsledok. Je aj
veľa súťaží v riešení algoritmických úloh (napr. 
\link{https://www.ksp.sk/}{KSP}, \link{http://oi.sk/}{slovenská} a 
\link{https://ioinformatics.org/}{medzinárodná} olympiáda, \ldots).

\indexItem{Alg}{štandardný vstup/výstup}
Ako si môžeš testovať programy aj sám? Keď program číta vstup pomocou \prg!cin>>! a 
vypisuje na výstup pomocou \prg!cout<<!, používa pri tom tzv. {\em štandardný vstup a výstup}.
Za normálnych okolností je to klávesnica a obrazovka, ale je možné ich 
{\em presmerovať}. Keď v linuxovom termináli spustíš 
\vb{./program <vstup >vystup}, program bude čítať vstup namiesto z klávesnice 
zo súboru \prg!vstup! a vypisovať namiesto obrazovky do súboru \prg!vystup!. Môžeš
si teda napísať rôzne vstupné súbory, ktoré si nazveš napr. \prg!1.in!, \prg!2.in!
atď. a k nim si pripravíš výstupné súbory \prg!1.out!, \prg!2.out! atď so správnymi 
odpoveďami.
Potom môžeš spustiť \vb{./program <1.in >1.prg} a v súbore \vb{1.prg} bude výstup 
programu zo vstupu \vb{1.in}. Teraz stačí porovnať, či sú \vb{1.prg} a \vb{1.out}
rovnaké. To sa dá aj automaticky, pomocou programu \vb{diff}: napíšeš 
\prg!diff 1.prg 1.out!. 
Ak sú rovnaké, 
nevypíše sa nič, inak sa vypíše, kde sa líšia.

Ďalšia možnosť spájania vstupu a výstupu je tzv. {\em pipe}. Ak napíšeš
\prg!./program1 | ./program2! tak sa spustí \vb{program1} a jeho výstup sa použije
ako vstup pre \vb{program2}.  Z príkazového riadku môžeš automaticky spustiť veľa testov.
Príkaz

\vb{for f in *.in; do ./program <\$f | diff - `basename \$f .in`.out; done}

spustí \vb{program} pre všetky súbory s príponou \vb{.in} a pre každý porovná výstup so
súborom \vb{.out} s rovnakým menom. 

Ak zistíš, že v programe je chyba, treba ju hľadať. Prostredia ako \vb{Code::Blocks} umožňujú program
spúšťať príkaz po príkaze a sledovať, ako sa pritom menia hodnoty premenných. To isté vie robiť
aj program, ktorý sa volá {\em debugger}, napr. \vb{gdb}. Pre naše účely ale budú celkom stačiť kontrolné výpisy: na dôležité miesta v programe pridáš vypisovanie premenných a keď spustíš program, vidíš, 
kde sa čo mení. Kontrolné výpisy nakoniec nezabudni vypnúť. Je dobré ich nemazať ale dať do komentárov, 
možno ich ešte v budúcnosti použiješ.
