\chapter{Memoizácia a dynamické programovanie}
\label{sect:dynamika}

Na začiatku kapitoly \ref{sect:zlozitost} sme mali príklad počítania Fibonacciho 
čísel, na ktorom sme ukazovali, že niekedy je rekurzívna formulácia veľmi prirodzená,
ale trvá dlho. Pozrime sa teraz na podobný príklad. 


Na vstupe je číslo $n$ a potom $n$ nezáporných 
čísel. Naším cieľom je vybrať niekoľko z nich tak, aby mali
čo najväčší súčet. Je tam ale navyše podmienka, že nesmieme vybrať dve po sebe idúce čísla.
Napr. pre vstup \vb{6 20 1 2 5} je výsledok \vb{25}, čo získame, ak vyberieme \vb{20} a potom
\vb{5}. Viac vybrať nemôžeme, lebo ak by sme nezobrali \vb{20}, tak celkový súčet všetkých ostávajúcich
čísel je \vb{14}. Ale ak zoberieme \vb{20}, tak nemôžeme zobrať \vb{6} ani \vb{1}. Z dvojice 
\vb{2} a \vb{5} môžeme vybrať len jedno číslo, takže \vb{25} je najviac, ako sa dá. 


Túto úlohu môžeme ľahko vyriešiť rekurziou, ktorá bude skúšať všetky možnosti. 
Ako môže vyzerať výber, ktorý spĺňa naše pravidlá? Buď prvé číslo \vb{a[0]}
nevyberieme, a potom
najlepšie, čo môžeme urobiť, je vybrať z čísel \vb{a[1], a[2], \ldots, a[n-1]} čo najviac
podľa pravidiel.
Alebo prvé číslo \vb{a[0]} vyberieme, potom ale nesmieme vybrať \vb{a[1]} a opäť najlepšie,
čo môžeme urobiť, je dovyberať podľa pravidiel z čísel \vb{a[2], \ldots, a[n-1]}. 
V oboch prípadoch vieme urobiť nejakú akciu (vybrať, či nevybrať prvé číslo) a potom musíme
vyriešiť tú istú úlohu (vyberať čísla podľa pravidiel), len na menšom vstupe. 
Toto sa prirodzene zapíše rekurziou. Naša funkcia \vb{najdi} bude hľadať najlepší
výber počnúc číslom na pozícii \vb{i}.

\begin{lstlisting}
#include <iostream>
using namespace std;

int a[10000];
int n;

int najdi(int i) {
  if (i >= n) return 0;
  int x = a[i] + najdi(i + 2), y = najdi(i + 1);
  if (x > y)
    return x;
  else
    return y;
}

int main() {
  cin >> n;
  for (int i = 0; i < n; i++) cin >> a[i];
  cout << najdi(0) << endl;
}

\end{lstlisting}


Na to, aby sme vyrátali hodnotu \vb{najdi(i)}, dvakrát sa zavolá rekurzívna funkcia:
raz pre \vb{najdi(i+1)} a raz pre \vb{najdi(i+2)}. Keď si začneme kresliť, aké svety
funkcií sa budú vytvárať nre vstup s \vb{n=5}, 
dostaneme obrázok, ktorý sme už videli pri Fibonacciho číslach:

\centerline{
\begin{tikzpicture}[scale=0.88]
  \foreach \y/\lst[count=\j] in {
    0/{0.5,1.5},
    0.7/{1,2,3,4,5,6,9,10},
    1.5/{1.5,3.5,5.5,7.5,9.5,11.5,13.5,15.5},
    2.6/{2.5,6.5,10.5,14.5},
    3.5/{4.5,12.5},
    4.5/{8.5}
  }{
    \foreach \x[count=\i] in \lst {
       \coordinate (p\j\i) at (\x,\y);
       %\pgfmathtruncatemacro{\tmp}{20*\j}
       %\filldraw[fill=black!\tmp] (p\j\i) circle (0.1);
    }
  }
  \foreach \i/\j/\k in {2/1/1,2/1/2,3/1/1,3/1/2,3/2/3,3/2/4,3/3/5,3/3/6,3/5/7,3/5/8,
  4/1/1,4/1/2,4/2/3,4/2/4,4/3/5,4/3/6,4/4/7,4/4/8,5/1/1,5/1/2,5/2/3,5/2/4,6/1/1,6/1/2}{
    \pgfmathtruncatemacro{\tmp}{\i-1}
    \draw (p\i\j)--(p\tmp\k);
  }
  \foreach \lst[count=\i] in {{2,1},{3,2,2,1,2,1,2,1},{4,3,3,2,3,2,2,1},{5,4,4,3},{6,5},{7}} {
    \foreach \v[count=\j] in \lst {
      \pgfmathtruncatemacro{\tmp}{7-\v}
      \node[circle,draw=black,fill=blue!10, inner sep=1pt] at (p\i\j) {\vb{\tmp}};
    }
  }
\end{tikzpicture}}


Treba si všimnúť dve veci: po prvé, rekurzívne volania funkcie \vb{najdi} síce narobia veľa
rôznych svetov, ale parameter \vb{i} je vždy z
rozsahu $0$ až $n-1$, takže tých veľa volaní vždy opakuje iba málo rôznych
parametrov. Navyše, zavolanie funkcie nijak neovplyvní svet
naokolo\footnote{hovoríme, že \vb{najdi} nemá vedľajšie efekty\indexItem{Alg}{vedľajší efekt funkcie}}: nemení
globálne premenné, nečíta vstup, ani nevypisuje, takže ak dvakrát zavolám
\vb{najdi} s rovnakým parametrom \vb{i} (a medzitým nezmením pole \vb{a}),
dostanem rovnaké výsledky. To znamená, že ak už raz vyrátam hodnotu pre nejaké
\vb{i}, môžem si ju zapamätať a pri nasledujúcom volaní s rovnakým parametrom
ju môžem použiť:

\begin{lstlisting}
#include <iostream>
using namespace std;

int a[10000], m[10000];
int n;

int najdi(int i) {
  if (i >= n) return 0;
  if (m[i] >= 0) return m[i];  // ak som už tento výsledok vyrátal, nemusím znovu
  int x = a[i] + najdi(i + 2), y = najdi(i + 1);
  if (x > y) {
    m[i] = x;  // zapamätám si, keby som to znovu potreboval
    return x;
  } else {
    m[i] = y;
    return y;
  }
}

int main() {
  cin >> n;
  for (int i = 0; i < n; i++) {
    cin >> a[i];
    m[i] = -1;    // na začiatku si sem dám zarážku
  }
  cout << najdi(0) << endl;
}

\end{lstlisting}


Teraz sa pre každé \vb{i} vyráta hodnota \vb{najdi(i)} iba raz, pri ďalšom použítí sa iba zoberie zapamätaná hodnota. 
Preto výsledná zložitosť bude lineárna.

\centerline{
\begin{tikzpicture}[scale=0.88]
  \foreach \y/\lst[count=\j] in {
    0/{0.5,1.5},
    0.7/{1,2,3,4,5,6,9,10},
    1.5/{1.5,3.5,5.5,7.5,9.5,11.5,13.5,15.5},
    2.6/{2.5,6.5,10.5,14.5},
    3.5/{4.5,12.5},
    4.5/{8.5}
  }{
    \foreach \x[count=\i] in \lst {
       \coordinate (p\j\i) at (\x,\y);
    }
  }
  \foreach \i/\j/\k in {2/1/1,2/1/2,3/1/1,3/1/2,3/2/3,3/2/4,3/3/5,3/3/6,3/5/7,3/5/8,
  4/1/1,4/1/2,4/2/3,4/2/4,4/3/5,4/3/6,4/4/7,4/4/8,5/1/1,5/1/2,5/2/3,5/2/4,6/1/1,6/1/2}{
    \pgfmathtruncatemacro{\tmp}{\i-1}
    \draw[gray!60] (p\i\j)--(p\tmp\k);
    \filldraw[gray!60, fill=white] (p\tmp\k) circle (0.2);
  }
  \foreach \i/\j/\k in {2/1/1,2/1/2,3/1/1,3/1/2,
  4/1/1,4/1/2,5/1/1,5/1/2,6/1/1,6/1/2}{
    \pgfmathtruncatemacro{\tmp}{\i-1}
    \draw (p\i\j)--(p\tmp\k);
  }
  \foreach \lst[count=\i] in {{2,1},{3,2},{4,3},{5,4},{6,5},{7}} {
    \foreach \v[count=\j] in \lst {
      \pgfmathtruncatemacro{\tmp}{7-\v}
      \node[circle,draw=black,fill=blue!10, inner sep=1pt] at (p\i\j) {\vb{\tmp}};
    }
  }
  \foreach \i in {2,...,5} {
      \pgfmathtruncatemacro{\tmp}{7-\i}
      \node[circle,draw=black,fill=green!10, inner sep=1pt] at (p\i2) {\vb{\tmp}};
  }
\end{tikzpicture}}


Keď si porovnám časy, pôvodná verzia už pre vstup dĺžky $50$ trvá vyše  minúty, vylepšená verzia
aj pre vstup dĺžky $1000$ zbehne za zhruba milisekundu. Technika, ktorú sme práve použili, sa volá\indexItem{Alg}{memoizácia}% 
{\em memoizácia}\footnote{Nevypadlo mi ``r'', nie je to {\em memorizácia}, je to {\em memoizácia}.}.
Pri nej máme rekurzívnu funkciu (dôležité je, aby nemala vedľajšie efekty), ktorá má síce
veľa volaní, ale málo rôznych parametrov a vylepšíme ju tak, že si zapamätávame už raz vyrátané
výsledky, aby sme ich nemuseli rátať znova.


Skúsme pokračovať v tom, aké veľké vstupy dokáže memoizovaná verzia vyriešiť. 
Aj pre pomerne veľké vstupy memoizovaná funkcia beží rýchlo, ale pri istej (nie príliš veľkej) veľkosti vstupu
program zrazu zhavaruje. Čo sa stalo?  Ako si videl v časti \ref{sec:rekurzia}, pri volaní každom rekurzívnej funkcie 
sa vyrobí nový svet a ostatné svety čakajú. Pre každý svet musí byť v pamäti vyhradená časť s jeho premennými.
Z dôvodov, ktoré tu teraz nebudem rozoberať, 
sa svety ukladajú do zásobníka, ktorý má fixnú veľkosť. Keď je svetov priveľa, zásobníku dojde pamäť a program zhavaruje.


Našťastie, v našom príklade sa pretečeniu zásobníka vieme ľahko vyhnúť. Všimni si, že v memoizovanom programe
používame pole \vb{m} na ukladanie už raz vypočítaných výsledkov; na konci výpočtu pole \vb{m} obsahuje všetky priebežné výsledky,
t.j. v \prg!b[i]! je hodnota \prg!najdi(i)!. V prístupe, ktorý sa volá {\em dynamické programovanie}\indexItem{Alg}{dynamické programovanie}
sa budem snažiť vyrobiť si rovnaké pole \vb{m}, ale bez rekurzívnych volaní. 
Hodnota \prg!m[i]! sa nastaví ako maximum z \prg!a[i] + najdi(i + 2)! a \prg!najdi(i + 1)!. Ak budem postupovať od konca poľa, bude platiť,
že keď počítam \prg!m[i]!, tak hodnoty \prg!m[i + 2]! aj \prg!m[i + 1]! už mám vypočítané. Preto (všimni si, že som si na koniec poľa pridal zarážku)
môžem napísať


\begin{lstlisting}
int najdi() {
  m[n] = 0;
  m[n - 1] = a[n - 1];
  for (int i = n - 2; i >= 0; i--) {
    int x = a[i] + m[i + 2], y = m[i + 1];
    if (x > y) m[i] = x;
    else m[i] = y;
  }
  return m[0];
}

\end{lstlisting}



Dynamické programovanie je prístup, v ktorom vypĺňam celú memoizačnú tabuľku bez toho, aby som robil rekurzívne volania. Na to potrebujem násjť správne poradie,
aby som si bol istý, že hodnoty, ktoré potrebujem na výpočet, už mám vyrobené. Okrem toho, že nepretečie zásobník, získam aj na rýchlosti. Obidve verzie 
majú lineárnu zložitosť (t.j. časy behu na všetkých vstupoch sa zmestia pod priamku), ale keďže dynamické programovanie nemusí vytvárať a rušiť 
svety na zásobníku, ušetrí dosť veľa času.\\

\centerline{
\begin{tikzpicture}
\begin{axis}[
  width=\textwidth, 
  height=10cm,
  xlabel=$n$,
  ylabel=čas (ms),
  scaled x ticks=false,
  scaled y ticks=false,
  y tick label style={
        /pgf/number format/.cd,
            fixed,
            fixed zerofill,
            precision=1,
        /tikz/.cd
    },
  legend cell align={left},
  legend pos = north west,
  %scale only axis,
  %separate axis lines,
  %domain=0:50000,
  %xmin=0,
  %xmax=50000,
  ymin=0,
  %ymax=20,
  /pgf/number format/.cd,
        1000 sep={}
]
  \addplot+[no markers,orange!80!white] table 
  [y expr=\thisrow{memo}/1000, x=n ]{data/flase.dat};
  \addlegendentry{memoizácia}
  \addplot+[no markers,blue!80!cyan] table 
  [y expr=\thisrow{dp}/1000, x=n ]{data/flase.dat};
  \addlegendentry{dynamické programovanie}
\end{axis}
\end{tikzpicture}
}

\begin{uloha}
  Uprav riešenie s dynamickým programovaním tak, aby okrem výsledného súčtu vypísalo aj vybraté čísla.
\end{uloha}


Skúsme urobiť ďalší príklad. Na vstupe máme dve postupnosti čísel
$a_0, a_1, \ldots, a_{n-1}$ a $b_0, b_1, \ldots, b_{m-1}$. Našim cieľom je 
čo najmenej z nich vyhodiť tak, aby obidve postupnosti ostali rovnaké. Napr. v nasledovnom vstupe musíme vyhodiť
7 čísel, tri z prvej postupnosti a štyri z druhej.


\begin{tikzpicture}
  
  \foreach \list[count=\y] in { {1/5,0/2,1/1,0/9,0/7,1/2,0/1,1/3,0/5,1/4},{1/5,1/1,0/6,1/2,0/7,1/3,1/4}}{
    \foreach \s/\v[count=\x] in \list {
    \if\s1 
    \node[draw=orange,circle, fill=orange!20]  at (\x,\y) {$\v$};
    \else 
    \node[draw=teal!50, circle, cross out] at (\x,\y) {$\v$};
    \fi
  }}

\end{tikzpicture}




Prvá vec, ktorú si môžeš všimnúť je, že ak obidve postupnosti začínajú rovnakým
číslom, toto číslo sa neoplatí ani z jednej postupnosti vyhodiť\footnote{ Skús
si to premyslieť, nie je to až také zrejmé, ako sa na prvý pohľad zdá.  Môže
byť totiž viacero optimálnych riešení a v niektorých z nich sa to prvé číslo
vyhodiť môže. Napr. pre \vb{5 4 5 5} a \vb{5 6 5} treba vyhodiť všetko tak, aby
v prvej aj druhej postupnosti ostalo \vb{5 5}. Takže z druhej postupnosti treba
ponechať obidve päťky a z prvej si môžem vybrať, ktoré dve päťky si nechám.
Ale ak obidve postupnosti začínajú rovnako, tak určite existuje aj také
optimálne riešenie, ktoré prvé číslo v obidvoch ponechá.  }. Ak teda $a_0=b_0$,
stačí nám vyriešiť úlohu pre postupnosti $a_1,\ldots, a_{n-1}$ a
$b_1,\ldots,b_{n-1}$. Čo ak $a_0\not=b_0$? Tu sa to nedá tak ľahko povedať: 
sú situácie, keď treba vyhodiť $a_0$, inokedy treba vyhodiť $b_0$
a niekedy aj obidve\footnote{skús nájsť príklady pre všetky tri situácie}.
Najjednoduchší spôsob, ako zaručiť, že nájdeme správne riešenie, je vyskúšať
všetky možnosti, to znamená, že najprv skúsime vyriešiť úlohu pre
dvojicu postupností $a_1,\ldots,a_{n-1}$ a $b_0,\ldots,b_{n-1}$, potom
pre $a_0,\ldots,a_{n-1}$ a $b_1,\ldots,b_{n-1}$ a napokon pre $a_1,\ldots,b_{n-1}$
a $b_1,\ldots,b_{n-1}$ a vyberieme tú najlepšiu možnosť. 
Pretože vždy sa opakuje tá istá úloha, napíšeme si rekurzívnu funkciu 
\vb{diff} s parametrami \vb{i} a \vb{j} tak, že \vb{diff(i,j)} nájde riešenie
úlohy pre vstup $a_i,a_{i+1},\ldots,a_{n-1}$ a $b_j,b_{j+1},\ldots,b_{n-1}$.
Mohlo by to vyzerať napríklad takto:\\


\vbox{
\begin{lstlisting}
#include <iostream>
using namespace std;

int n[2], a[2][10000];

int diff(int i, int j) {
  if (i >= n[0]) return (n[1] - j); // ak je prvá postupnosť prázdna, 
                                    // z druhej treba vyhodiť všetko
  if (j >= n[1]) return (n[0] - i);
  if (a[0][i] == a[1][j]) return diff(i + 1, j + 1);
  int x = 1 + diff(i + 1, j), y = 1 + diff(i, j + 1);
  if (y < x) x = y;
  y = 2 + diff(i + 1, j + 1);
  if (y < x) x = y;
  return x;
}

int main() {
  cin >> n[0] >> n[1];
  for (int i = 0; i < 2; i++)
    for (int j = 0; j < n[i]; j++) cin >> a[i][j];
  cout << diff(0, 0) << endl;
}
\end{lstlisting}
}


Keďže funkcia \vb{diff} nemá vedľajšie efekty, môžeme použiť memoizáciu a 
už raz vypočítané výsledky si zapamätať: budeme mať dvojrozmerné pole 
\vb{m} tak, že \vb{m[i][j]} bude mať zapamätaný výsledok volania
\vb{diff(i,j)}. Memoizovanú verziu vieme prerobiť na dynamické programovanie
tak, že budeme pole \vb{m} vytvárať  zaradom celé.
Zase si musíme dať pozor na to, aby sme pole prechádzali  v takom poradí, že keď
počítame \vb{m[i][j]}, všetky potrebné hodnoty, t.j. \vb{m[i][j+1]},
\vb{m[i+1][j]} a \vb{m[i+1][j+1]} už máme vypočítané.\\

\vbox{
\begin{lstlisting}
for (int i = n[0]; i >= 0; i--)
  for (int j = n[1]; j >= 0; j--)
    if (i == n[0])
      m[i][j] = n[1] - j;
    else if (j == n[1])
      m[i][j] = n[0] - i;
    else if (a[0][i] == a[1][j])
      m[i][j] = m[i + 1][j + 1];
    else {
      int x = 1 + m[i + 1][j], y = 1 + m[i][j + 1];
      if (y < x) x = y;
      y = 2 + m[i + 1][j + 1];
      if (y < x) x = y;
      m[i][j] = x;
    }

cout << m[0][0] << endl;
\end{lstlisting}
}

 
Do tretice skúsme takýto príklad. Budeme mať známu hru {\sffamily minesweeper}, ale vo veľmi zjednodušenej podobe: na jednorozmernom poli. Budeme mať $n$
políčok, pričom na niektorých môžu byť bomby. Na políčkach, ktoré susedia s bombou je napísaný počet bômb, s ktorými susedia. Takže herný plán môže vyzerať napr. takto

  \def\bop{\usymH{1F4A3}{1.8ex}}
  \def\qp{\usymH{2754}{2ex}}
  \def\jp{\vb{\textcolor{green!60!black}{1}}}
  \def\dvp{\vb{\textcolor{red!60!black}{2}}}
  \def\bb{red!10}
  \def\qb{teal!30}
  \def\nb{white}


\begin{tikzpicture}[scale=0.5]
  \foreach \v/\bg[count=\x] in {\bop/\bb,\jp/\nb,~/\nb,\jp/\nb,\bop/\bb,\dvp/\nb,\bop/\bb,\bop/\bb,\jp/\nb} {
    \draw[fill=\bg](\x,0) rectangle +(1,1) node[pos=0.5]{\v};
  }
\end{tikzpicture}


Na vstupe budeme mať danú rozohratú pozíciu, v ktorej niektoré políčka sú zakryté. 
Vstup je daný ako reťazec znakov, kde \vb{'*'} znamená bombu,  napr.


\begin{tikzpicture}[scale=0.5]
  \foreach \v/\bg[count=\x] in {\qp/\qb,\qp/\qb,\qp/\qb,\jp/\nb,\bop/\bb,\dvp/\nb,\qp/\qb,\qp/\qb,\jp/\nb} {
    \draw[fill=\bg](\x,0) rectangle +(1,1) node[pos=0.5]{\v};
  }
\end{tikzpicture}


je \vb{???1*2??1}. Našou úlohou je zistiť, koľkými spôsobmi je možné doplniť zakryté políčka tak, aby sme mali korektný hrací plán. V našom príklade
na predposlednom políčku musí byť bomba (lebo na poslednom je napísaná jednotka), a na siedmom políčku tiež (lebo šieste políčko susedí s dvoma bombami).
Na treťom políčku bomba nesmie byť, preto sú štyri možné hracie plány:


\begin{tikzpicture}[scale=0.5]
  \foreach \v/\bg[count=\x] in {~/\nb,~/\nb,~/\nb,\jp/\nb,\bop/\bb,\dvp/\nb,\bop/\bb,\bop/\bb,\jp/\nb} {
    \draw[fill=\bg](\x,0) rectangle +(1,1) node[pos=0.5]{\v};
  }
  \foreach \v/\bg[count=\x] in {\bop/\bb,\jp/\nb,~/\nb,\jp/\nb,\bop/\bb,\dvp/\nb,\bop/\bb,\bop/\bb,\jp/\nb} {
    \draw[fill=\bg](\x,-1.5) rectangle +(1,1) node[pos=0.5]{\v};
  }
  \foreach \v/\bg[count=\x] in {\jp/\nb,\bop/\bb,\jp/\nb,\jp/\nb,\bop/\bb,\dvp/\nb,\bop/\bb,\bop/\bb,\jp/\nb} {
    \draw[fill=\bg](\x,-3) rectangle +(1,1) node[pos=0.5]{\v};
  }
  \foreach \v/\bg[count=\x] in {\bop/\bb,\bop/\bb,\jp/\nb,\jp/\nb,\bop/\bb,\dvp/\nb,\bop/\bb,\bop/\bb,\jp/\nb} {
    \draw[fill=\bg](\x,-4.5) rectangle +(1,1) node[pos=0.5]{\v};
  }
\end{tikzpicture}



Priamočiary spôsob, ako to naprogramovať, je rekurzívne skúšať všetky možnosti. Ak je na nejakom políčku \vb{i} znak \prg!'?'!, postupne skúsime dosadiť
znaky \prg!'*','0','1','2'! a rekurzívne vyskúšame všetky možnosti pre políčka \vb{i+1} a ďalšie. Vždy, keď prídeme na koniec, skontrolujeme, 
či máme prípustný hrací plán, a ak áno zarátame ho ako jednu z možností:\\

\begin{lstlisting}
#include <iostream>
using namespace std;

char s[10000];
char znaky[] = "*012";
int n;

int pocet;  // tu si pamätáme celkový počet možností

void skus(int i) {
  if (i == n) { // ak sme prišli na koniec vstupu, zistíme, či máme správny hrací plán
    for (int j = 0; j < n; j++)
      if (s[j] != '*') { // pre každé políčko, ktoré nie je bomba
        int val = s[j] - '0'; // hodnota 0,1,2
        int sus = 0;          // počet bômb, s ktorými susedí
        if (j > 0 && s[j - 1] == '*') sus++;
        if (j < n - 1 && s[j + 1] == '*') sus++;
        if (val != sus) return; // ak je to nesprávne, nebola to dobrá možnosť, 
                                // skončíme celú funkciu
      }
    pocet++; // ak sme prišli až sem, všetky políčka sú korektné, máme novú možnosť
    return;
  }

  // v opačnom prípade, kým nie sme na konci vstupu
  if (s[i] == '?') { // ak tam bol otáznik, skúsime dosadiť všetky možnosti
    for (int j = 0; j < 4; j++) {
      s[i] = znaky[j];
      skus(i + 1); // zakaždým sa rekurzívne zavoláme
    }
    s[i] = '?'; // nakoniec upraceme - rozmysli si, prečo je toto dôležité !
  } else
    skus(i + 1); // ak na políčku nebol otáznik, len sa zavoláme ďalej
} // koniec funkcie skus

int main() {
  cin >> s;
  n = 0;
  while (s[n] != 0) n++;
  cout << n << endl;

  pocet = 0;
  skus(0);
  cout << pocet << endl;
}
\end{lstlisting}


Pre malé vstupy to funguje, ale už napr. vstup, v ktorom je 20 otáznikov, trvá pridlho. Kontrolná 
otázka: môžeme na túto funkciu použiť memoizáciu?



Samozrejme, že nie, pretože modifikuje globálne premenné. Prvý problém je, že výsledok nevracia
pri volaní, ale modifikuje globálnu premennú \vb{pocet}. To sa dá napraviť jednoducho:
\vb{skus} bude vracať počet možností, ako sa dá doplniť zvyšok od $i$-tej pozície. V podmienke
\prg!if (i == n)! sa vráti \vb{0} alebo \vb{1} podľa toho, či je pozícia korektná.
Pre opačnú podmienku si zrátam dokopy výsledky všetkých rekurzívnych volaní a tento súčet vrátim.
Druhý problém je, že počas volania sa modifikuje pole \vb{s}. Preto výsledok volania \vb{skus(i)}
nemôžem memoizovať, lebo závisí od toho, aké je práve pole \vb{s}. Na memoizáciu potrebujem spraviť takú 
rekurzívnu funkciu, ktorá by si všetky zmeny posielala v parametroch. Keby som ale ako parameter
posielal celé pole \vb{s}, memoizačná tabuľka by bola priveľká. Našťastie si stačí uvedomiť, že na to, aby som 
skontroloval, či je hrací plán v poriadku, nepotrebujem čakať do konca poľa, ale viem robiť kontrolu
priebežne. Na to, aby som skontroloval políčko $i$ mi stačí vedieť hodnoty políčok $i-1$ a $i+1$.
Lenže keď volám \vb{skus(i)}, tak hodnotu políčka $i+1$ nepoznám (možno je tam \prg!'?'!). 
Vyriešim to napr. takto: budem si ako parametre posielať hodnoty predchádzajúcich dvoch políčok.
Volanie \vb{skus(i,a,b)} bude teda znamenať \cmd{Zrátaj, koľkými možnosťami môžeme doplniť
hrací plán od pozície $i$ do konca, ak na políčkach $i-2$ a $i-1$ sú znaky \vb{a} a \vb{b}}.
Teraz pri volaní \vb{skus(i,a,b)} viem skontrolovať, či je políčko $i-1$ v poriadku.
Celé to môže vyzerať napr. takto:

\begin{lstlisting}
// Mám tri za sebou idúce znaky a,b,c. Je znak b v poriadku?
// parameter i je iba na to, aby som nekontroloval zarážku pred vstupom
int kontrola(int i, char a, char b, char c) {
  if (i == 0 || b == '*') return 1; // zarážka alebo bomba je vždy v poriadku
  int val = b - '0';                // hodnota podľa čísla  
  int sus = 0;                      // počet susedov
  if (a == '*') sus++;
  if (c == '*') sus++;
  if (val != sus) return 0;
  return 1;
}

char znaky[] = "*012";

int skus(int i, char a, char b) {
  if (i == n) return kontrola(i, a, b, '0');
  if (s[i] == '?') {
    int sum = 0;
    for (int j = 0; j < 4; j++)
      sum += kontrola(i, a, b, znaky[j]) * skus(i + 1, b, znaky[j]);
    return sum;
  } else 
    return kontrola(i, a, b, s[i]) * skus(i + 1, b, s[i]);
}
\end{lstlisting}


Táto funkcia už nemodifikuje žiadne globálne premenné, takže ju ľahko memoizujem: pridám
si pole \prg!m[10000][4][4]!, ktoré si inicializujem zarážkou (napr. \vb{-1}).
Vždy, keď zavolám \vb{skus(i,a,b)} najprv skontrolujem, či \vb{m[i][a][b]>-1}. Ak áno,
iba vrátim uloženú hodnotu. Ak nie, urobím všetko ako doteraz, iba pred skončením aktualizujem
hodnotu \vb{m[i][a][b]}, aby som ju mal uloženú pre ďalšie volania.
Posledný krok je dynamické programovanie: budem postupne prechádzať poľom \vb{m} (odzadu)
a vypĺňať všetky možnosti.

\begin{lstlisting}
for (int i = n; i >= 0; i--)
  for (int a = 0; a < 4; a++)
    for (int b = 0; b < 4; b++) {
      if (i == n) {
        if (znaky[b] == s[n - 1] || s[n - 1] == '?')
          m[i][a][b] = kontrola(i, znaky[a], znaky[b], '0');
      } else if (s[i] == '?') {
        int sum = 0;
        for (int j = 0; j < 4; j++)
          sum += kontrola(i, znaky[a], znaky[b], znaky[j]) * m[i + 1][b][j];
        m[i][a][b] = sum;
      } else {
        int j;
        for (j = 0; j < 4; j++)
          if (s[i] == znaky[j]) break;
        m[i][a][b] = kontrola(i, znaky[a], znaky[b], s[i]) * m[i + 1][b][j];
      }
    }
\end{lstlisting}



\begin{uloha}
  Na vstupe je dané \vb{n} a potom pole $n$ celých čísel.
  Napíš program, ktorý vypíše najdlhšiu (nie nutne súvislú) rastúcu 
  postupnosť, ktorá sa vyskytuje vo vstupnom poli. 
  Napr. pre pole \vb{10 6 9 7 1 8 2 4 3 4} 
  treba vypísať \vb{1 2 3 4}.
\end{uloha}


\begin{uloha}
  Na vstupe je reťazec znakov. Napíš program, ktorý vypíše 
  najdlhší (nie nutne súvislý) palindróm (reťazec, ktorý je rovnaký spredu aj zozadu),
  ktorý sa v ňom vyskytuje. 

  Napr. pre vstup
  \hbox{\vb{zjedzlesneplodyvsipilivokneglej}}
  je najdlhší palindróm \vb{jelenovipivonelej}.
\end{uloha}

\begin{uloha}
  Na vstupe je čislo $n$ a potom $n$ čísel. Každé číslo je buď $-1$ alebo celé
  číslo z rozsahu $1,\ldots,200$. Našim cieľom je nahradiť všetky $-1$
  číslami z rozsahu $1,\ldots,200$ tak, aby vo výslednej postupnosti neboli
  lokálne maximá, t.j. aby každé číslo susedilo aspoň s jedným číslom, ktoré nie
  menšie. Napíš program, ktorý zistí, koľkými spôsobmi sa to dá urobiť.
  Napr. pre vstup \vb{-1 -1} je $200$ možností, lebo obidve čísla musia byť
  rovnaké. Pre vstup \vb{100 -1 -1} je tiež $200$ možností: druhé číslo musí byť
  $100$ alebo viac, lebo inak by prvé číslo bolo lokálne maximum. Ak je druhé
  číslo $100$, tretie číslo môže byť $100$ alebo čokoľvek menšie. Ak je druhé
  číslo väčšie ako $100$, tretie číslo musí byť rovnaké, lebo inak by druhé
  číslo bolo lokálne maximum.
\end{uloha}

\begin{uloha}
  Máme danú mapu lesa rozmerov $m\times n$, pričom každé políčko je buď voľné,
  alebo je na ňom strom. Začíname v ľavom hornom rohu a v každom kroku sa môžeme
  pohnúť o jedno políčko doprava, dole, alebo šikmo. Napíš program, ktorý zistí,
  koľkými možnosťami to vieme urobiť. Napríklad pre vstup vľavo existuje $21$ rôznych ciest,
  dve z nich sú naznačené na obrázku vpravo.
  
\begin{column}{0.45}
\begin{outputBox}
4 5
.##.#
...#.
.#..#
#....
\end{outputBox}
\end{column}\hfill\begin{column}{0.45}
  \begin{tikzpicture}[scale=0.6]
    \def\row(#1)#2{
      \gdef\tmp{#2}
      \foreach\x in {0,...,4} {
        \isnextbyte[q]{H}{\tmp}
        \if T\theresult{
          \draw (\x,-#1) rectangle +(1,1) node [pos=0.5] {\textcolor{green!50!black}{\usymH{1F332}{2ex}}};
        }\else {
          \draw (\x,-#1) rectangle +(1,1);
        }\fi
        \gobblechar{\tmp}
        \xdef\tmp{\thestring}
      }
    }

    \def\cesta[#1]#2{
      \filldraw[#1] (0.5,0.5)
      \foreach \x/\y in {#2} {
        -- (\y+0.5,-\x+0.5) circle (0.04)
      }
      ;
    }

    \row(0){.HH.H}
    \row(1){...H.}
    \row(2){.H..H}
    \row(3){H....}
     
     \cesta[thick,teal]{1/1,1/2,2/3,3/4}
     \cesta[thick,orange]{1/0,1/1,2/2,3/2,3/3,3/4}

  \end{tikzpicture}
\end{column}
\end{uloha}

\begin{uloha}
  Máme batoh s objemom $4200$ litrov. Na vstupe je $n$ vecí, každá z nich má objem
  a cenu (obidve sú celé čísla, objem každej je nanajvýš $4200$). 
  Chceme vybrať niektoré veci tak, aby sa nám zmestili
  do batoha (t.j. súčet ich objemov bol nanajvýš $4200$) a mali čo najväčšiu cenu.
  Napíš program, ktorý zistí, akú najväčšiu cenu môžeme dosiahnuť.
  Napríklad pre veci s cenami \vb{100 100 201} a objemami \vb{2000 2000 3000}
  najlepšie, čo môžeme urobiť, je zobrať tretiu vec a získať \vb{201}.
\end{uloha}
