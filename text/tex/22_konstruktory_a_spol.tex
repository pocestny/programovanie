\chapter{Konštruktory, referencie, operátory a iný cukor}
\label{sect:cukor}

V predchádzajúcej kapitole si videl, že typ \vb{vector} mal niektoré
divné vlastnosti (napr. že \prg!a[x]! vráti \vb{x}-tú premennú podobne ako 
pole). Teraz ti chcem ukázať, že to nie je žiadna mágia. Poďme poporiadku.
Povedzme, že chceme urobiť typ na prácu s tabuľkami. Tabuľka má mať $m$ riadkov
a $n$ stĺpcov a majú v nej byť celé čísla. Aby sme zabránili možnosti, že
pole \vb{data} nebude inicializované, spravíme aj konštruktor:

\vskip 2ex
\vbox{
\begin{lstlisting}[] 
struct Tabulka {
  int *data;
  int m,n;

  Tabulka() {
    m = 0;
    n = 0;
    data = nullptr;
  }

};
\end{lstlisting}
}

Konštruktor je funkcia takmer ako každá iná, a preto môže mať aj parametre. 
Konštruktor pre typ \vb{Tabulka} preto môžem zmeniť takto:

\begin{lstlisting}[] 
Tabulka(int _m, int _n) {
    m = _m;
    n = _n;
    data = new int[m * n];
    for (int i = 0; i < m; i++)
      for (int j = 0; j < n; j++) 
        data[i * n + j] = 0;
}
\end{lstlisting}

A pridáme aj deštruktor:

\begin{lstlisting}[] 
~Tabulka() {
  if (data!=nullptr) delete[] data;
}
\end{lstlisting}

Teraz sa premenná typu \vb{Tabulka} dá vyrobiť, iba ak dostane dva parametre:

\vskip 2ex
\vbox{
\begin{lstlisting}[] 
int main() {
  Tabulka t(3,3);  // v poriadku
  Tabulka q;       // kompilátor vyhlási chybu
}
\end{lstlisting}}

\indexItem{Prg}{implicitný (defualt) konštruktor}
Kým si nemal vlastný konštruktor, kompilátor používal na vytváranie premenných 
{\em implicitný} (default) konštruktor. Keď si si ale zadefinoval svoj, default
sa už nepoužíva. Stav, keď sa premenná bez potrebných parametrov nedá vyrobiť, je
niekedy žiadúci. Len si treba uvedomiť, že teraz nevieš alokovať dynamické pole
\prg!Tabulka *x = new Tabulka[10];!, lebo nemáš ako dodať parametre. Môžeš ale\indexItem{Prg}{dynamická alokácia: \vb{new} a \vb{delete}}
dynamicky alokovať jednu premennú, a to pomocou verzie príkazov \prg!new!, 
\prg!delete! bez hranatých zátvoriek:

\begin{lstlisting}[] 
int main() {
  Tabulka *t = new Tabulka(3,3);
  // niečo urob
  delete t;
}
\end{lstlisting}

\prg!new T! bez hranatých zátvoriek urobí veľmi podobnú vec ako \prg!new T[1]!, t.j.
alokuje jednu premennú, ale máš možnosť dodať parametre. Len si treba dať pozor na to,
že premenné alokované cez \prg!new! treba odalokovať cez \prg!delete! a premenné alokované 
cez \prg!new[]! treba odalokovať cez \prg!delete[]!. 


Okrem alokovania polí môžu parametre v konštruktoroch spôsobovať problémy aj pri vnorených 
typoch. Predstav si, že chceme mať typ, ktorý obsahuje tabuľku a pole dvoch čísel, napr.

\begin{lstlisting}
 struct Tukan {
  int a[2];
  Tabulka t;
};
\end{lstlisting}

Teraz sa zamysli, čo sa stane, keď chceš vyrobiť premennú \vb{Tukan x}: v pamäti sa vyhradí 
miesto pre jednu premennú typu \prg!int[2]! a jednu premennú typu \vb{Tabulka}. V čase, keď
sa zavolá konštruktor \vb{Tukan} (ktorý je zatiaľ iba implicitný), musia sa nastaviť premenné
\vb{a} a \vb{t} ich vlastnými konštruktormi. Ale \vb{Tabulka} nemá konštruktor bez 
parametrov, takže sa vyhlási chyba. Ešte raz zopakujem: keď sa vytvára premenná nejakého typu,
najprv sa vyhradí miesto, potom sa (rekurzívne) zavolajú konštruktory všetkých zložiek
a nakoniec sa zavolá hlavný konštruktor\footnote{toto je jediná postupnosť, ktorá dáva zmysel:
v konštruktore typu \vb{Tukan} asi chceš pristupovať k premennej \vb{t},
takže chceš, aby \vb{Tabulka t} už bola inicializovaná}. Ako môžeš dodať premennej \vb{t}
parametre? Mohlo by ťa napadnúť napísať 

\begin{lstlisting}
 struct Tukan {
  int a[2];
  Tabulka t(3,3);
};
\end{lstlisting}

ale tým jednak stratíš možnosť inicializovať tabuľku inými parametrami ako $3$, ale hlavne
nie je jasné, kedy sa má konštruktor \vb{Tabulka} vlastne volať. Takýto zápis sa
ti kompilátor ani nepokúsi skompilovať. Riešenie je špeciálny zápis konštruktora s
dvojbodkou:\indexItem{Prg}{poradie konštruktorov, \vb{:}}

\begin{lstlisting}
 struct Tukan {
  int a[2];
  Tabulka t;

  Tukan() : t(3,3) {}
};
\end{lstlisting}

Týmto zápisom hovoríš, že \vb{Tukan} má konštruktor bez parametrov, ktorý predtým,
ako sa začne vykonávať, zavolá konštruktor \vb{Tabulka} s parametrami $3,3$.
Za dvojbodkou môže byť aj viacero konštruktorov oddelených čiarkami, napr.

\begin{lstlisting}
  Tukan(int x) : t(x,x), a{x,x} {a[1] += a[0];}
\end{lstlisting}

Keď teraz vyrobíš premennú \vb{Tukan x(3);} tak najprv sa zavolá konštruktor \vb{Tabulka t(3,3)}
a inicializuje sa pole \vb{a} hodnotami \vb{3,3} a potom sa začne vykonávať
hlavný konštruktor, v ktorom sa nastaví \vb{a[1]=6}.

 Nielen pri konštruktoroch je niekedy\indexItem{Prg}{implicitné (default) parametre}
užitočný mechanizmus {\em default parametrov}. Ak pri definovaní funkcie
za posledných\footnote{
  Musia to byť posledné, lebo inak by nebolo jasné, ktorý parameter 
  sa má nahradiť. Keby si mal \prg!int f(int a=3, int b, int c=1)!,
  volanie \prg!f(2,2)! môže znamenať \prg!a=2!, \prg!b=2! a \prg!c=1! (default),
  ale aj \prg!a=3! (default), \prg!b=2! a \prg!c=2!. Kvôli tejto nejednoznačnosti
  je takéto umiestnenie default parametrov zakázané.
}niekoľko parametrov dopíšeš \vb{=}, nastavíš im implicitné hodnoty.
Ak sa potom pri volaní príslušný parameter vynechá, doplní sa
implicitnou hdontou. Napr.

\vbox{
\begin{lstlisting}[] 
void chlp(int x = 4, int y = 5) { cout << x << " " << y << endl; }

int main() {
  chlp();     // vypíše 4 5
  chlp(3);    // vypíše 3 5
  chlp(1,2);  // vypíše 1 3
}
\end{lstlisting}
}

Preto, keď sa vrátime k našim tabuľkám, 
keď do konštruktora typu \vb{Tabulka} pridáš default paramatre napr. takto:
\prg!Tabulka(int _m = 3, int _n = 3)!, tak
zápis \prg!Tabulka q;! vyrobí tabuľku $3\times 3$. 

Podobne
\prg!Tabulka *a = new Tabulka[10];! vyrobí pole $10$ tabuliek, každú rozmerov
$3\times 3$.

 Ďalej si môžeme napísať funkciu, ktorá ku každému prvku tabuľky priráta číslo.
Naša trieda bude vyzerať takto:

\begin{lstlisting}[] 
struct Tabulka {
  int* data;
  int m, n;

  Tabulka(int _m = 3, int _n = 3) {
    m = _m;
    n = _n;
    data = new int[m * n];
    for (int i = 0; i < m; i++)
      for (int j = 0; j < n; j++)
        data[i * n + j] = 0;
  }

  void pridaj(int x) {
    for (int i = 0; i < m; i++)
      for (int j = 0; j < n; j++)
        data[i * n + j] += x;
  }

  ~Tabulka() { 
    if (data!=nullptr) delete[] data; 
  }
};
\end{lstlisting}

Toto nijak neprekvapí, môžeš napísať napr.
\begin{lstlisting}[] 
 Tabulka t, r;
 t.pridaj(5);
\end{lstlisting}

 Teraz chcem napísať funkciu, ktorá priráta nie jedno číslo, ale tabuľku. Možno 
je trochu prekvapivé, že ju môžem nazvať rovnako: ak majú dve funkcie rôzne parametre,
môžu sa volať rovnako, lebo kompilátor aj tak vie, ktorú má použiť. Preto môžem do
definície triedy \vb{Tabulka} pridať funkciu \prg!void pridaj(Tabulka x)!. Musím
si rozmyslieť, čo robiť, ak majú tabuľky rôznu veľkosť. Rozhodol som sa, že budem 
prirátavať len v rozsahu, ktorý je dosť malý na to, aby sa zmestil do oboch:

\begin{lstlisting}[] 
  void pridaj(Tabulka x) {
    int nn = n, mm = m;
    if (x.n < n) nn = x.n;
    if (x.m < m) mm = x.m;
    for (int i = 0; i < mm; i++)
      for (int j = 0; j < nn; j++)
        data[i * n + j] += x.data[i * x.n + j];
  }
\end{lstlisting}

Teraz sa program skompiluje\footnote{%
  zápis \prg!x += y;! je skratka za \prg!x = x + y;!
}(kompilátor sa nesťažuje na dve funkcie s rovnakým menom),
ale keď spustím

\begin{lstlisting}[] 
int main() {
  Tabulka t, r(4,4);
  t.pridaj(5);
  r.pridaj(t);
}
\end{lstlisting}

vypíše sa mi 

\begin{outputBox}
free(): double free detected in tcache 2
Aborted
\end{outputBox}

Skús zistiť, kde je problém. Prišiel si na to?
Pred volaním \prg!r.pridaj(t);! to vyzeralo takto:


\def\var(#1)#2#3{%
  \draw (0.2,-#1) rectangle node[align=center](v#1) {\vb{#3}} (1.2,1-#1);
  \draw (0,0.5-#1) node[anchor=east]{\textcolor{magenta}{\vb{#2}}};
}

\def\ptr(#1,#2)[#3] {
  \draw[blue,-{>[length=4pt,width=2.5pt]}] (1.7,0.5-#1) to[out=#3,in=-#3] (1.7,0.5-#2);
}

\def\world(#1)#2#3{
  \draw[thick,#3](-1,-#1) -- (2,-#1) node[anchor=west]{\vb{#2}};
}

\def\kon(#1){ 
  \draw[draw=none] (0.2,-#1) rectangle node[align=center]{\vb{\ldots}} (1.2,1-#1);
}
  
\def\godot#1{\filldraw[blue,scale=0.6](#1) circle (0.4ex and 1.4ex);}

\begin{tikzpicture}[yscale=0.4,xscale=1.4]

  \var(0){t.data}{}
  \var(1){t.m}{3}
  \var(2){t.n}{3}
  \var(3){r.data}{}
  \var(4){r.m}{4}
  \var(5){r.n}{4}
  \world(-1){t}{orange}
  \world(2){r}{olive}
  \kon(6)

  \coordinate (a1) at (3,-0.5);
  \coordinate (a2) at (3,-4.5);

  \foreach \f/\t in {v0/a1,v3/a2}{
  \godot{\f}
  \draw[blue,-{>[length=8pt,width=5pt]}] (\f) to [out=0,in=180] (\t);
  \draw (\t)++(0,-0.5) rectangle node[align=center]{alokovaná pamäť} ++(5,1);
  }
\end{tikzpicture}

 Pri volaní sa vyrobil nový svet pre funkciu \vb{r.pridaj}. V ňom je premenná
\vb{x} typu \vb{Tabulka}, do ktorej sa nakopíruje premenná \vb{t}.


\begin{tikzpicture}[yscale=0.4,xscale=1.4]

  \var(0){t.data}{}
  \var(1){t.m}{3}
  \var(2){t.n}{3}
  \var(3){r.data}{}
  \var(4){r.m}{4}
  \var(5){r.n}{4}
  \var(6){x.data}{}
  \var(7){x.m}{3}
  \var(8){x.n}{3}
  \var(9){nn}{}
  \var(10){mm}{}
  \world(-1){t}{orange}
  \world(2){r}{olive}
  \world(5){x}{cyan!50!gray}
  \world(8){}{brown}
  \kon(11)

  \coordinate (a1) at (3,-0.5);
  \coordinate (a2) at (3,-4.5);

  \foreach \f/\t in {v0/a1,v3/a2,v6/a1}{
  \godot{\f}
  \draw[blue,-{>[length=8pt,width=5pt]}] (\f) to [out=0,in=180] (\t);
  }
  \foreach \t in {a1,a2}{
  \draw (\t)++(0,-0.5) rectangle node[align=center]{alokovaná pamäť} ++(5,1);
  }
  \draw[red,thick,->](-0.5,-4.9)-- (-1,-4.9)--node[midway,anchor=south,rotate=90]{\vb{r.pridaj}}(-1,-11);
\end{tikzpicture}

Keď sa funkcia \vb{r.pridaj} skončí, jej svet zanikne a s ním aj premenné v ňom.
Keď ide zaniknúť premenná \vb{x}, zavolá sa jej deštruktor, ktorý odalokuje 
pamäť.

\begin{tikzpicture}[yscale=0.4,xscale=1.4]

  \var(0){t.data}{}
  \var(1){t.m}{3}
  \var(2){t.n}{3}
  \var(3){r.data}{}
  \var(4){r.m}{4}
  \var(5){r.n}{4}
  \world(-1){t}{orange}
  \world(2){r}{olive}
  \kon(6)

  \coordinate (a1) at (3,-0.5);
  \coordinate (a2) at (3,-4.5);

  \foreach \f/\t in {v0/a1,v3/a2}{
  \godot{\f}
  \draw[blue,-{>[length=8pt,width=5pt]}] (\f) to [out=0,in=180] (\t);
  }
  \draw[dotted] (a1)++(0,-0.5) rectangle node[align=center]{odalokovaná pamäť} ++(5,1);
  \draw (a2)++(0,-0.5) rectangle node[align=center]{alokovaná pamäť} ++(5,1);
\end{tikzpicture}

Pointer \vb{t.data} sa nemenil a ukazuje na to isté miesto v pamäti, ale tá
už bola odalokovaná. Keď sa potom končí hlavný program, volá sa deštruktor
\vb{t}, ktorý sa pokúša druhýkrát odalokovať tú istú pamäť, a preto program skončí s chybou.


Aby sme sa týmto efektom vyhli, je lepšie ako parameter funkcie pridaj neposielať
hodnotu \vb{Tabulka}, ale iba pointer \vb{Tabulka *} takto:

\vbox{
\begin{lstlisting}[] 
  void pridaj(Tabulka *x) {
    int nn = n, mm = m;
    if (x->n < n) nn = x->n;
    if (x->m < m) mm = x->m;
    for (int i = 0; i < mm; i++)
      for (int j = 0; j < nn; j++)
        data[i * n + j] += x->data[i * x->n + j];
  }
\end{lstlisting}
}

Keď potom zavoláme \prg!r.pridaj(&t);! do sveta funkcie sa pridá iba pointer
na \vb{t}.


\begin{tikzpicture}[yscale=0.4,xscale=1.4]

  \var(0){t.data}{}
  \var(1){t.m}{3}
  \var(2){t.n}{3}
  \var(3){r.data}{}
  \var(4){r.m}{4}
  \var(5){r.n}{4}
  \var(6){x}{}
  \var(7){nn}{}
  \var(8){mm}{}
  \world(-1){t}{orange}
  \world(2){r}{olive}
  \world(5){}{cyan!50!gray}
  \kon(9)

  \coordinate (a1) at (3,-0.5);
  \coordinate (a2) at (3,-4.5);

  \foreach \f/\t in {v0/a1,v3/a2}{
  \godot{\f}
  \draw[blue,-{>[length=8pt,width=5pt]}] (\f) to [out=10,in=180] (\t);
  }
  \godot{v6}
  \draw[blue,-{>[length=8pt,width=5pt]}] (v6) to [out=20,in=-20] (1.2,0.3);

  \foreach \t in {a1,a2}{
  \draw (\t)++(0,-0.5) rectangle node[align=center]{alokovaná pamäť} ++(5,1);
  }
  \draw[red,thick,->](-0.5,-4.9)-- 
  (-1,-4.9)--node[midway,anchor=south,rotate=90]{\vb{r.pridaj}}(-1,-11);
\end{tikzpicture}

Preto keď zaniká svet \vb{r.pridaj}, zanikne pointer a nie je dôvod volať
deštruktor \vb{t}.

 Posielať ako parameter do funkcie pointer je veľmi užitočné, ale niekedy, ako uvidíme
o chvíľu, to môže byť nepraktické. V C++ je možnosť ''zamaskovať'' pointer \indexItem{Prg}{referencia}
pomocou tzv. {\em referencie}. Referencia sa v programe tvári ako premenná,
ale v skutočnosti je to pointer na jej adresu\footnote{Už z tohto vidno, že je to tak trochu
ako cirkulárka: užitočná vec, ale chceš si pri nej dávať pozor, inak sa neprestaneš čudovať,
čo sa to deje.}. Zapisuje podobne ako pointer, ale namiesto hviezdičky sa použije
ampersand (\prg!&!)\footnote{%
  \vb{\&} a \vb{*} môžu byť zo začiatku mätúce. Pri vyrábaní premennej je \vb{int *x} pointer na \vb{int} a
  \vb{int \&x} referencia. Vo výrazoch je \vb{\&x} adresa (t.j. pointer) a \vb{*x} je hodnota pointra.
}. Napríklad obidva programy

\begin{column}{0.45}
\begin{lstlisting}[] 
void zblnk(int &x) { x++; }

int main() {
  int x = 3;
  zblnk(x);
  cout << x << endl;
}
\end{lstlisting}
\end{column}
\hfill
\begin{column}{0.45}
\begin{lstlisting}[] 
void brnk(int *x) { (*x)++; }

int main() {
  int x = 3;
  brnk(&x);
  cout << x << endl;
}
\end{lstlisting}
\end{column}

vypíšu $4$, lebo parameter \prg!int &x! hovorí \cmd{zober referenciu (t.j. adresu) na premennú \vb{x}}. Takže
v skutočnosti sa  do funkcie \vb{zblnk} aj \vb{brnk} posiela 
pointer na premennú \vb{x}. Pri volaní \prg!brnk(&x)! je jasné, že parameter je pointer,
ale pri volaní \prg!zblnk(x)! nevieš rozlíšiť, či je funkcia \vb{zblnk} definovaná
s parametrom \prg!int! alebo \prg!int &!. Keďže je to ten druhý prípad, tak
\vb{zblnk(x)} zavolať môžeš, ale
\vb{zblnk(3)} vyhlási chybu v tom zmysle, že sa nedá urobiť pointer na trojku.

 Referencie sa hodia napríklad keď chceme upraviť, akým spôsobom sa majú tabuľky 
kopírovať. Podobne ako pri vyrábaní premenných sa volá konštruktor, pri priradení\indexItem{Prg}{operátor priradenia}
sa volá {\em operátor} priradenia (\vb{=}). Default operátor priradenia sa správa tak, 
že skopíruje premenné, ale môžeme si to upraviť: je to funkcia triedy ako každá iná, len
má trochu divný zápis. V našom prípade môžeme triedu \vb{Tabulka} definovať
napr. takto:

\begin{lstlisting}[] 
struct Tabulka {
  int* data;
  int m, n;
  
  Tabulka(int _m = 3, int _n = 3) { alokuj(_m, _n); }
  
  void alokuj(int _m, int _n) {
    m = _m; 
    n = _n;
    data = new int[m * n];
    for (int i = 0; i < m; i++)
      for (int j = 0; j < n; j++) 
        data[i * n + j] = 0;
  }

  void pridaj(int x) {
    for (int i = 0; i < m; i++)
      for (int j = 0; j < n; j++) 
        data[i * n + j] += x;
  }

  void pridaj(Tabulka& x) {
    int nn = n, mm = m;
    if (x.n < n) nn = x.n;
    if (x.m < m) mm = x.m;
    for (int i = 0; i < mm; i++)
      for (int j = 0; j < nn; j++) 
        data[i * n + j] += x.data[i * x.n + j];
  }

  Tabulka& operator=(Tabulka& x) {
    if (data!=nullptr) delete[] data;
    alokuj(x.m, x.n);
    pridaj(x);
    return *this;
  }

  ~Tabulka() { 
    if (data!=nullptr) delete[] data; 
  }
};
\end{lstlisting}

Pridal som funkciu \prg!operator=!, čo je špeciálne meno vyhradené na úpravu operátora 
priradenia.  Ako parameter sa zoberie referencia na tabuľku. Vracia sa tiež referencia,
a to referencia na seba (\prg!return *this;!, lebo \prg!this! je pointer na seba
a \prg!*this! je jeho hodnota). To ti umožňuje písať skrátene
\prg!t = r = q;! V tomto prípade sa najprv zavolá funkcia \prg!r.operator=(q)!, ako 
parameter sa zoberie referencia na \vb{q}, na základe nej sa upravia hodnoty v \vb{r}
a vráti sa referencia na \vb{r}. Tá sa hneď zoberie ako parameter pri volaní
\prg!t.operator=()!, takže hodnoty \vb{t} sa upravia podľa čerstvo upravených hodnôt \vb{r}.

 
 \indexItem{Prg}{copy constructor}Podobný ako operátor priradenia je tzv. {\em copy constructor}, ktorý sa volá
vtedy, ak sa má vyrobiť nová premenná z inej. Konkrétne napr. pri posielaní parametrov do funkcie.
V našom príklade máme operátor priradenia, ale ak by si mal napr. v rámci \vb{Tabulka}
metódu (bez referencie)
\prg!void quak(Tabulka x) { pridaj(x); }!, tak volanie \prg!r.quak(t)! bude 
mať rovnaký problém ako predtým: pri volaní sa do lokálnej premennej \vb{x}
skopírujú hodnoty
z \vb{r} a pri skončení fukcie sa zavolá deštruktor, ktorý odalokuje pointer.
Keby si okrem konštruktora \prg!Tabulka(int _m, int _n)! pridal ešte
konštruktor, ktorý má ako parameter referenciu na už existujúcu premennú typu
\vb{Tabulka}:

\begin{lstlisting}
  Tabulka(Tabulka& x) {
    alokuj(x.m, x.n);
    pridaj(x);
  }
\end{lstlisting}

tak pri volaní \vb{r.quak(t)} sa použije tento konštruktor, ktorý namiesto toho,
aby kopíroval pointer, v lokálnej premennej
\vb{x} naalokuje pamäť, takže volanie bude namiesto:


\begin{tikzpicture}[yscale=0.4,xscale=1.4]

  \var(0){t.data}{}
  \var(1){t.m}{3}
  \var(2){t.n}{3}
  \var(3){r.data}{}
  \var(4){r.m}{4}
  \var(5){r.n}{4}
  \var(6){x.data}{}
  \var(7){x.m}{3}
  \var(8){x.n}{3}
  \world(-1){t}{orange}
  \world(2){r}{olive}
  \world(5){x}{cyan!50!gray}
  \kon(9)

  \coordinate (a1) at (3,-0.5);
  \coordinate (a2) at (3,-4.5);

  \foreach \f/\t in {v0/a1,v3/a2,v6/a1}{
  \godot{\f}
  \draw[blue,-{>[length=8pt,width=5pt]}] (\f) to [out=0,in=180] (\t);
  }
  \foreach \t in {a1,a2}{
  \draw (\t)++(0,-0.5) rectangle node[align=center]{alokovaná pamäť} ++(5,1);
  }
  \draw[red,thick,->](-0.5,-4.9)-- 
  (-1,-4.9)--node[midway,anchor=south,rotate=90]{\vb{r.quak(t)}}(-1,-11);
\end{tikzpicture}

vyzerať takto:


\begin{tikzpicture}[yscale=0.4,xscale=1.4]

  \var(0){t.data}{}
  \var(1){t.m}{3}
  \var(2){t.n}{3}
  \var(3){r.data}{}
  \var(4){r.m}{4}
  \var(5){r.n}{4}
  \var(6){x.data}{}
  \var(7){x.m}{3}
  \var(8){x.n}{3}
  \world(-1){t}{orange}
  \world(2){r}{olive}
  \world(5){x}{cyan!50!gray}
  \kon(9)

  \coordinate (a1) at (3,-0.5);
  \coordinate (a2) at (3,-3.5);
  \coordinate (a3) at (3,-6.5);

  \foreach \f/\t in {v0/a1,v3/a2,v6/a3}{
  \godot{\f}
  \draw[blue,-{>[length=8pt,width=5pt]}] (\f) to [out=0,in=180] (\t);
  }
  \foreach \t in {a1,a2,a3}{
  \draw (\t)++(0,-0.5) rectangle node[align=center]{alokovaná pamäť} ++(5,1);
  }
  \draw[red,thick,->](-0.5,-4.9)-- 
  (-1,-4.9)--node[midway,anchor=south,rotate=90]{\vb{r.quak(t)}}(-1,-11);
\end{tikzpicture}

Preto deštruktor na konci volania správne odalokuje pamäť v lokálnej premennej \vb{x}.
Tento spôsob funguje, ale pri väčších tabuľkách je neefektívny: dáta sa najprv kopírujú
z \vb{t} do lokálnej \vb{x}, potom z lokálnej \vb{x} do \vb{r}, a nakoniec
sa lokálna \vb{x} zahodí. Je lepšie ako parameter do funkcie posielať pointer alebo referenciu.

\indexItem{Prg}{predefinovanie operátorov}
 Operátor priradenia nie je jediný operátor, ktorý sa dá predefinovať, dajú sa takmer
všetky\footnote{%
  detaily sú napr. na 
  \href{https://en.cppreference.com/w/cpp/language/operators}{\nolinkurl{%
en.cppreference.com/w/cpp/language/operators}}}.
Ak chceme napríklad vedieť sčitovať tabuľky, môžeme spraviť niečo takéto:

%\vbox{
\begin{lstlisting}
struct Tabulka {
  // celá definícia
};

Tabulka operator+(Tabulka &x, Tabulka &y) {
  Tabulka res(x);
  res.pridaj(y);
  return res;
}

int main() {
  Tabulka r, s, t;
  t = r + s;
}
\end{lstlisting}
%}

\begin{lrbox}{\TmpBox}
  \begin{lstlisting}[basicstyle=\scriptsize\roboto]
swap(T& a, T& b) {
    T tmp(move(a));  // z a urob xvalue, použije sa move constructor
    a = move(b);     // z b urob xvalue, použije sa move assignment
    b = move(tmp);   // z tmp urob xvalue, použije sa move assignment
}
\end{lstlisting}
\end{lrbox}

Definoval som (mimo triedy \vb{Tabulka}) operátor \vb{+}, ktorý zoberie 
referencie na dve tabuľky a vyrobí výslednú, ktorá bude ich súčtom. Všimni si,
že keby sa vracala hodnota referenciou ako
\begin{lstlisting}
Tabulka& operator+(Tabulka &x, Tabulka &y) {
  Tabulka res(x);
  res.pridaj(y);
  return res;
}
\end{lstlisting}
tak by to celé nefungovalo:
pri volaní funkcie \prg!operator+! sa v jej svete vyrobí lokálna premenná \vb{res},
nejak sa upraví a vráti sa referencia (teda pointer) na ňu. Lenže pri skončení
volania sa celý svet funkcie zruší, zavolá sa deštruktor na \vb{res}, a preto
vrátená referencia (teda pointer) je nanič\phantomsection\label{page:move-operator}\footnote{\indexItem{Prg}{move constructor: \vb{lvalue}, \vb{rvalue}, \vb{xvalue}, \vb{prvalue}} 
Poznámka pre fajnšmekrov. Asi si postrehol, že toto riešenie nie je ideálne:
pri volaní \prg!return! sa všetky dáta z \prg!res! skopírujú do výsledku a vzápätí sa volá deštruktor na \prg!res!. Nebolo by lepšie
nejak povedať, aby si kompilátor len ``premenoval adresy'' a v ďalšom používal
adresu premennej \vb{res} ako výsledok? Zatiaľ si v programoch videl tzv. \vb{lvalue} 
(meno premennej, referencia, t.j. niečo, čo môže stáť na ľavej strane priradenia)
a \vb{rvalue} (konštanta, volanie funkcie, tiež meno premennej, t.j. čokoľvek,
čo môže stáť na pravej strane priradenia). C++ má navyše aj tzv. \vb{xvalue}
({\em expiring value}, napr. lokálna premenná pri volaní \prg!return!, 
čo je \vb{lvalue}, ktorá zároveň
kompilátoru hovorí \cmd{s mojimi datami si môžeš robiť, čo chceš, aj tak 
budem za chvíľu volať deštruktor}) a \vb{prvalue} (napr. dočasná premenná,
v ktorej je uložený výsledok výrazu pri volaní; ak volám \prg!pridaj(4+7)!,
tak pri vytváraní sveta funkcie \vb{pridaj} je v pamäti dočasná premenná \prg!int!
s hodnotu $11$). Kompilátor vie pri priradení z \vb{xvalue} alebo \vb{prvalue}
optimalizovať veci tak, aby sa vyhol kopírovaniu. Na to treba, aby tvoja
trieda mala tzv. {\em move constructor} a {\em move assignment} 
(všetky STL triedy ako \prg!vector!, 
\prg!string! a pod. ich majú). Takže keby si okrem copy konštruktora 
\prg!Tabulka(Tabulka& x)! mal aj move konštruktor \prg!Tabulka(Tabulka&& x)! (s dvoma
ampersandami, tzv. {\em rvalue reference}) a move operátor  
\prg!Tabulka& operator=(const Tabulka&& x)!,
kompilátor by pri volaní \prg!return res;! zavolal move operátor. Ten namiesto
kopírovania dát môže len priradiť pointer.
Samozrejme, je dôležité, aby sa  postaral o to, že následné
volanie deštruktora \vb{x} bude v poriadku, napr. 
\prg!Tabulka& operator=(Tabulka&& x){m=x.m; n=x.n; data=x.data; x.data=nullptr; return *this;}!.
Na explicitné pretypovanie z \vb{lvalue} na \vb{xvalue} slúži
funkcia \vb{move}, napr.

\usebox{\TmpBox}

Výsledkom toho je, že premenné z STL (napr. \vb{vector}) môžeš pokojne vracať z funkcie
hodnotou; kompilátor použije \vb{move} sémantiku a dáta sa nebudú kopírovať.
}.


Ale keď tento program skúsiš skompilovať, zistíš, že to stále nejde. Prečo? Z rovnakého
dôvodu, ako nešlo spustiť \vb{zblnk(3)} o pár odstavcov vyššie: \prg!operator+!
vráti {\em hodnotu} typu \vb{Tabulka}, ale \prg!operator=! má ako parameter
{\em referenciu}, teda pointer na premennú. A ten sa nedá zobrať, ak premenná nie je.
Na druhej strane, ona tá premenná v skutočnosti kdesi musí byť: pri vyhodnocovaní výrazu
je uložená na nejakom bezmennom mieste v pamäti. Ak kompilátoru sľúbiš, že ju nebudeš
meniť, bude ochotný z nej pointer (referenciu) zobrať. Takže zmeníme definíciu \indexItem{Prg}{modifikátor \vb{const}}
\prg!operator=! na 
\prg!Tabulka& operator=( const Tabulka& x)!. Teraz ti ale kompilátor bude vyčítať, že si 
nedodržal sľub: v rámci volania \prg!operator=! voláš \vb{pridaj(x)}. Keďže \vb{x}
sa tam posiela referenciou, ktohovie, či sa tam nezmení. Musíš ho upokojiť a sľúbiť, že
ani tam sa \vb{x} nebude meniť: \prg!void pridaj(const Tabulka& x)!. Teraz už všetko
bude fungovať správne.


 Ak si v rámci definície triedy \vb{Tabulka} predefinuješ operátor \vb{()}
takto:

\begin{lstlisting}
  int& operator()(int r, int s) { return data[r * n + s]; }
\end{lstlisting}

môžeš pristupovať k políčkam tabuľky napr. \prg!t(i,j) = r(i,j);!. 
To, že k 
položkám triedy \vb{vector} môžeš pristupovať ako k poľu je preto, že \vb{vector}
si predefinoval operátor \vb{[]}.


Teraz aj vidíš, že som ťa klamal, keď som hovoril, že \prg!cout <<! je {\em príkaz}
na vypísanie. V skutočnosti \prg!cout! je premenná typu \prg!ostream! 
definovaná v súbore \prg!iostream!. Trieda \prg!ostream! má predefinovaný operátor
\prg!<<! (ktorý pri celých číslach robí bitový posun). Ak by si chcel napr. vypisovať
tabuľky, môžeš si definovať\footnote{%
všimni si, že vraciam referenciu na prvý parameter, takže volanie \prg!cout << x << y!
robí to, čo sa očakáva
}(mimo definície triedy \vb{Tabulka})

\begin{lstlisting}
ostream& operator<<(ostream& str, Tabulka& o) {
  for (int i = 0; i < o.m; i++) {
    for (int j = 0; j < o.n; j++) str << o(i, j) << " ";
    str << endl;
  }
  return str;
}
\end{lstlisting}

tak môžeš zavolať napr \prg!Tabulka t; cout << t;! Nemôžeš ale urobiť \prg!cout << s + t!
opäť z rovnakých dôvodov: \prg!operator<<! potrebuje referenciu. Opäť to môžeš
vyriešiť tak, že sľúbiš, že parameter \vb{o} bude konštantný. t.j. 
\prg!const Tabulka& o!. Tým ale vyrobíš iný problém:
zápis \prg!o(i,j)! je volanie funkcie \prg!o.operator()!, čo je metóda objektu \vb{o}. 
No a keďže sa nevie, či ho nemení, kompilátor ju odmietne. To, že daná metóda nemení svoj objekt, 
vieš sľúbiť tak, že \prg!const! napíšeš za definíciu metódy. Takže celý
výsledný program by mohol vyzerať nejak takto (mám dve verzie \prg!operator()!: jedna vráti hodnotu, a tá je \prg!const!,
druhá referenciu a tá nie je \prg!const!; kompilátor si vie vybrať podľa toho, čo práve potrebuje):

\begin{lstlisting}
#include <iostream>
using namespace std;

struct Tabulka {
  int* data;
  int m, n;

  Tabulka(int _m = 3, int _n = 3) { alokuj(_m, _n); }
  Tabulka(const Tabulka& x) {alokuj(x.m, x.n); pridaj(x); }
  ~Tabulka() { delete[] data; }
  void alokuj(int _m, int _n) { ... }
  void pridaj(int x) {...}
  void pridaj(const Tabulka& x) { ... }
  Tabulka& operator=(const Tabulka& x) { pridaj(x); return *this; }
  int& operator()(int r, int s) { return data[r * n + s]; }  
  int operator()(int r, int s) const { return data[r * n + s]; }
  Tabulka& operator+=(int x) { pridaj(x); return *this;}
};

ostream& operator<<(ostream& str, const Tabulka& o) { ... }
Tabulka operator+(const Tabulka& x, const Tabulka& y) { ... }

int main() {
  Tabulka t;  t += 5;
  Tabulka r = t; r += 2;
  r(1,1) = 42;
  cout << t + r;
}
\end{lstlisting}

