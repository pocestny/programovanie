\chapter{Ďalšie cykly}

Poznáme už príkaz cyklu \prg!while! a úplne by nám stačil, aby sme s ním mohli 
naprogramovať všetko, čo chceme. Sú ale aj ďalšie príkazy cyklov, ktoré môžeme
použiť, aby sme dostali kratší a čitateľnejší program. Veľakrát sme potrebovali
 pomocou cyklu postupne niečo prechádzať. Na to sme vždy mali premennú,
ktorú sme na začiatku nastavili (napr.) na nulu
a v cykle sme ju zvyšovali nejak takto:\\

\begin{lstlisting}
int i;
i = 0;
while (i < n) {
  // niečo urob
  i = i + 1;
}
\end{lstlisting}

Na skrátenie programu a zlepšenie čitateľnosti je možné vyrobenie premennej
a priradenie do nej urobiť naraz, t.j. môžeme napísať\\

\begin{lstlisting} 
int i = 0;
while (i < n) {
  // niečo urob
  i = i + 1;
}
\end{lstlisting}

\indexItem{Prg}{príkaz \vb{for}}
Tento typ programu sa dá čitateľnejšie zapísať iným príkazom cyklu, \prg!for!.
Príkaz \prg!for! má v zátvorkách tri časti, oddelené bodkočiarkami. Prvá časť
je príkaz, ktorý sa má vykonať na začiatku, druhá časť je podmienka ako vo \prg!while!
cykle a tretia časť je príkaz, ktorý sa má vykonať po každom vykonaní cyklu.
Predchádzajúci príklad by mohol vyzerať:\\

\begin{lstlisting}
int i;
for (i = 0; i < n; i = i + 1) {
  // niečo urob
}
\end{lstlisting}


\indexItem{Prg}{inkrement \vb{i++}, \vb{i+=1}}
Rovnako častý je aj príkaz \prg!i = i + 1!, ktorý hovorí \cmd{Zober to, čo je uložené
v premennej \prg!i!, prirátaj 1 a výsledok ulož naspäť do \prg!i!}. Tento príkaz je tak
častý, že má vlastnú skratku, stačí napísať\footnote{\phantomsection\label{foot.inc-vyraz}
  Zvýšenie hodnoty premennej o 1 sa zvykne volať {\em inkrementovanie} a zníženie
  o 1 {\em dekrementovanie}.
  Zápis \vb{i++} sa dá použiť
  aj ako výraz, napr. \vb{ if (i++ < 10) ...} znamená \cmd{Najprv zober hodnotu z premennej \vb{i}.
  Premennú \vb{i} vzápätí zvýš o 1 a pôvodnú hodnotu porovnaj s 10}.
  Podobne sa dá použiť aj \vb{++i}, ktorý inkrementuje premennú pred použitím. 
  Ak je to vo forme príkazu, je to jedno, ale výraz \hbox{\vb{if (++i < 10) ...}} znamená
  \cmd{K premennej \vb{i} najprv prirátaj 1 a potom ju porovnaj s 10}.
  Rovnaký zápis funguje aj pre odčítanie, t.j. \prg!--i! a \prg!i--!.
  Ak chcem pridať inú hodnotu, ako 1, môžem napísať \vb{+=}, napr. \prg!i += 3! namiesto
  \prg!i = i + 3!.
}\prg!i++!. Cyklus tvaru
\prg!for (i = 0; i < n; i++) { ... }! 
je veľmi častý. 

\indexItem{Prg}{príkaz \vb{do-while}}
Príkaz \prg!while! vyhodnocuje podmienku na začiatku, pred vykonaním príkazu. Niekedy
sa viac hodí, aby sa podmienka vyhodnocovala na konci. Na to slúži príkaz 
\hbox{\prg!do!-\prg!while!}. Dajme tomu, že chceme čítať čísla zo vstupu
a vypisovať o jedno väčšie, až kým neprečítame -1. Môžeme to napísať takto:\\

\begin{lstlisting}
int x;
do {
  cin >> x;
  cout << x + 1 << endl;
} while (x >= 0);
\end{lstlisting}

\indexItem{Prg}{príkazy \vb{break}, \vb{continue}}
Niekedy je dobré prerušiť vykonávanie cyklu. Na to slúžia príkazy \prg!break! (skonči
vykonávanie celého cyklu) a \prg!continue! (skonči vykonávanie jednej iterácie a prejdi 
na ďalšiu). Napr. nasledujúci program kopíruje čísla zo vstupu na výstup, 
až kým neprečíta $-1$ a potom skončí:\\

\begin{lstlisting}
while (true) {
  cin >> x;
  if (x == -1) break;
  cout << x << endl;
}
\end{lstlisting}


Podobne nasledujúci program vypisuje iba nepárne čísla, až kým neprečíta -1:\\

\begin{lstlisting}
do {
  cin >> x;
  if (x % 2 == 0) continue;
  cout << x << endl;
} while (x != -1);
\end{lstlisting}

