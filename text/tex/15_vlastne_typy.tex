\chapter{Vlastné typy: \vb{struct}}

Základné typy, o ktorých sme si hovorili v kapitole~\ref{sect:cisla}
majú spoločné to, že zaberajú nejakú fixnú časť pamäte. Okrem základných typov je\indexItem{Prg}{\vb{struct}} 
možnosť vyrobiť si aj vlastné typy. Na to slúži kľúčové slovo \prg!struct! (záznam).
Vlastný typ, ktorý si vyrobíš v programe, bude pozostávať z niekoľkých častí,
z ktorých každá má meno a svoj typ. Napríklad pre prácu s farbami sa často
používa RGBA model: farba je pamätaná ako štyri čísla z rozsahu $0\ldots255$,
ktoré udávajú množstvo červenej, zelenej a modrej zložky. Posledné číslo je
tzv. alfa zložka a udáva priesvitnosť: 0 je úplne priesvitná a 255 úplne nepriesvitná.\\


\vbox{
\begin{lstlisting}[] 
struct Farba {
  unsigned char r, g, b, a;
};
\end{lstlisting}
}

Týmto si vyrobil typ \vb{Farba}. Premenné tohto typu budú v pamäti zaberať 
4 byty, pričom každý z nich sa bude správať ako celé číslo bez znamienka (t.j. budú
mať hodnoty $0\ldots255$). Premennú tohto typu vyrobíš rovnako ako akúkoľvek inú, napr.
\prg!Farba f;! vyrobí premennú \vb{f} typu \vb{Farba}. K jej jednotlivým zložkám
sa dá pristupovať pomocou operátora \vb{.} (bodka): \prg!f.r! je premenná typu 
\prg!unsigned char!, ktorá je zložka \vb{r} z premennej \vb{f}. 
Celú premennú môžeš nastaviť zápisom v kučeravých zátvorkách, napr. 
\prg!f = {255, 255, 0, 255};!
urobí žltú farbu. Jednu premennú môžeš skopírovať do druhej pomocou priradenia. 
Priradenie skopíruje celú pamäť, t.j. ak máš premenné \prg!Farba f1,f2;! tak
\prg!f1 = f2;! je to isté ako 
\prg!f1.r = f2.r;! 
\prg!f1.g = f2.g;! 
\prg!f1.b = f2.b;! 
\prg!f1.a = f2.a;! 
Keďže to ale nie sú čísla, nemôžeš použiť iné operácie, napr. sčítanie, \prg!f1 + f2! 
je chyba. Program\\

\vbox{
\begin{lstlisting}[] 
struct Farba {
  unsigned char r, g, b, a;
};

int main() {
  Farba f, h;
  f = {100, 100, 255, 255};
  h = f;
  h.r = 0;
  h.b = 80;
  cout << (int)h.r << " " << (int)h.g << " " << (int)h.b << endl;
}
\end{lstlisting}
}

vypíše \vb{0 100 80}. \indexItem{Prg}{pretypovanie (konverzia)}Všimni si zápis \prg!(int)h.r!. Ten sa volá pretypovanie
({\em konverzia}). Výraz
\vb{($\langle typ\rangle$)$\langle vyraz\rangle$} hovorí \cmd{Zober hodnotu výrazu $\langle
vyraz\rangle$ a urob z nej typ $\langle typ\rangle$.} V našom prípade je 
$\langle vyraz\rangle$ \vb{h.r}, čo je hodnota typu \prg!unsigned char!.
Bez konverzie by sa k nej príkaz \prg!cout <<! správal ako k znaku: vypísal by znak s 
príslušnou ASCII hodnotou. Konverzia spôsobí, že hodnota výrazu sa síce nezmení, ale 
\prg!cout <<! vypíše príslušné číslo. Podobne príkaz


\prg!int i = -1; cout << (unsigned int)i << " " << i << endl;! 

 vypíše
\vb{4294967295 -1} (v pamäti na adrese premennej \vb{i} je stále uložená
tá istá hodnota, 32 jednotiek v dvojkovom zápise).


Jednotlivé zložky typu \prg!struct! môžu byť akékoľvek typy alebo polia, takže 
môžeš mať napr.

\vbox{
\begin{lstlisting}[] 
struct Gradient{
  Farba x[2];  // ak je tu pole, musí mať fixnú veľkosť
  bool linear;
};
\end{lstlisting}
}

a písať\footnote{\indexItem{Prg}{inicializačný zoznam (initializer list)}%
  S takýmto zápisom v kučeravých zátvorkách sa stretneme ešte veľakrát, hovorí sa mu {\em inicializačný zoznam} ({\em initializer list}). Do (takmer) hocijakej premennej,
  ktorá má viac zložiek (napr. pole, \prg!struct!) môžeš priradiť konštantu, ak jednotlivé položky (v poradí, v akom sú definované) vymenuješ
  v kučeravých zátvorkách.
}\prg!Gradient g = {{f, {100, 100, 100, 255}}, false};! alebo 
\prg!g.x[1].r = g.x[0].b;! a pod. V súbore  
\link{\rootpath/obrazok.h}{obrazok.h}
je aj jednoduchá podpora na prácu s farebnými obrázkami, takže napr. program\\


\vbox{
\begin{lstlisting}[] 
#include "obrazok.h"
#include <iostream>
using namespace std;

struct Farba {
  unsigned char r, g, b, a;
};

Farba pal[2] = {{0, 0, 255, 255}, {255, 255, 0, 255}};

int main() {
  int d = 100;
  Farba a[8 * d][8 * d];
  int i, j;
  for (i = 0; i < 8 * d; i++)
    for (j = 0; j < 8 * d; j++)
      a[i][j] = pal[((i / d) + (j / d)) % 2];
  zapis_rgba_png(8 * d, 8 * d, a, "vystup.png");
}
\end{lstlisting}
}


vyrobí súbor \vb{vystup.png}, v ktorom bude žlto-modrá šachovnica.

\def\ulohaTarget{None}
\begin{uloha}
  Pozri si stránku 
  \link{https://sk.wikipedia.org/wiki/Gal\%C3\%A9ria_\%C5\%A1t\%C3\%A1tnych_vlajok}{%
  galéria štátnych vlajok} a napíš program, ktorý pre zadaný názov štátu vyrobí obrázok
  s jeho vlajkou. Pre ktoré štáty to vieš urobiť?
\end{uloha}
\def\ulohaTarget{File}
