\chapter{STL: typ \vb{vector} a \vb{string}}

V predchádzajúcej kapitole sme si navrhli typ \vb{Stack}, ktorý spravoval
zásobník čísel typu \prg!int!. Čo keby si chcel zásobník typu \prg!float!, alebo
\vb{Farba}? Jedna možnosť by bola vyrobiť zakaždým nový typ, napr. \vb{StackInt},
\vb{StackFloat} a pod. Dalo by sa to tak, že skopíruješ \vb{Stack} a na vhodných miestach
prepíšeš \prg!int! na nový typ. Ale to nie je dobrý prístup, lebo keď niekedy v budúcnosti
budeš chcieť niečo zmeniť, musíš to meniť rovnako na veľa miestach \ldots oštara.\indexItem{Prg}{šablóna (template)}
V C++ je na to pohodlný mechanizmus, ktorý sa volá {\em šablóny} (templates).
Sú podobné preprocesoru v tom, že sa spracovávajú predtým, ako sa vôbec program
začne kompilovať a nahradzujú nejaký symbol iným textom. Ale zároveň ich spracováva 
už kompilátor, takže vie, kde sa v programe očakáva meno typu, kde premenná a pod.
Najjednoduchšie je to vidno na príklade:

\vskip 2ex
\vbox{
\begin{lstlisting}[] 
template <typename T>
struct Stack {
  T *a;
  int n, max;

  Stack() {...}
  ~Stack() {...}
  void resize(int _max){...}
  bool push(T x) {...}
  void discard() {...}
};
\end{lstlisting}
}

Týmto hovoríš predpis (šablónu), ako vyrobiť typ zásobník, v ktorom sa budú ukladať prvky 
hocijakého typu \vb{T} (ľahko doplníš, čo treba do vybodkovaných častí; na vhodných
miestach treba nahradiť \prg!int! za \vb{T}). Keď potom v programe napíšeš
\prg!Stack<int> s;! kompilátor zoberie šablónu, dosadí za \vb{T} typ \prg!int!
čím dostane definíciu typu \prg!Stack<int>! a vyrobí premennú príslušného typu. 
Ak neskôr napíšeš \prg!Stack<Farba> sf;! kompilátor zoberie šablónu, vyrobí príslušný typ
a premennú.


Šablóna sa to volá preto, lebo \vb{Stack} sám osebe nie je typ. Je to len predpis,
ktorým hovoríš kompilátoru, ako si má vyrobiť typy \prg!Stack<int>! alebo
\prg!Stack<Farba>!
a pod., ak ich časom bude potrebovať použiť. Jendotlivé typy \prg!Stack<int>!, \prg!Stack<Farba>! a
pod. sú preto nezávislé typy, ktoré sa vyrobia až vtedy, keď sa v programe vyskytnú.
A, samozrejme, šablóna môže mať viacero parametrov,
napr.

\begin{lstlisting}
template <typename S, typename T>
struct Dvojica {
  S s;
  T* t;
};
\end{lstlisting}

a potom môžeš mať premennú \prg!Dvojica<int, Farba> d;!, ktorá bude obsahovať
\prg!int d.s! a \prg!Farba* d.t!. Alebo 
môžeš mať aj premennú \prg!Dvojica<int, Dvojica<int, int>> x;! a potom
\vb{x.t} bude pointer na typ \prg!Dvojica<int, int>!, takže napr.
\prg!x.t->s! je typu \prg!int!.


Podobne môže byť šablóna na funkcie. Ak napíšeš

\begin{lstlisting}
template<typename T>
T lama(T z) { return z;}
\end{lstlisting}

v programe sa nepridá nič, ale hovoríš kompilátoru, ako si vyrobiť (inak celkom
zbytočné) funkcie 

\begin{lstlisting}
int lama<int>(int z){return z;}
Farba lama<Farba>(Farba z){return z;}
\end{lstlisting}

a tak podobne, ak ich niekedy v programe bude potrebovať zavolať.
Kompilátor navyše niekedy vie odvodiť parametre šablóny, a vtedy ich 
netreba písať. Ak napíšeš \prg!lama(42)!, kompilátor pochopí, že tým myslíš
funkciu \prg!lama<int>(42)! a vyrobí si ju podľa šablóny. Pravidlá, čo a ako si má
domyslieť, sú pomerne zložité, nateraz stačí o tom vedieť, aby ťa to neprekvapilo.

 
\indexItem{Prg}{Standard Template Library (STL)}
Súčasťou definície jazyka C++ je {\em Standard Template Library} (STL) -- knižnica, ktorá
obsahuje šablóny na veľa užitočných dátových štruktúr a algoritmov. Jedným z nich je typ\indexItem{Prg}{typ \vb{vector}}
\prg!vector!, ktorý sa veľmi podobá nášmu typu \vb{Stack}: obsahuje v sebe pole prvkov,
vie mu rezervovať veľkosť, vie pridať prvok na koniec a pod. Keď ho chceš použiť,
treba na začiatku programu pridať \prg!#include <vector>!.

 Pre hocijaký typ \prg!T!, premenná typu \prg!vector<T>! obsahuje pointer \prg!T*!,
ktorý ukazuje na dynamicky alokovanú pamäť. Vektor obsahuje \vb{size} prvkov a
má alokovanú pamät na \vb{capacity} prvkov:


\def\var(#1)#2#3{%
  \draw (0.2,-#1) rectangle node[align=center](v#1) {\vb{#3}} (1.2,1-#1);
  \draw (0,0.5-#1) node[anchor=east]{\textcolor{magenta}{\vb{#2}}};
}

\def\ptr(#1,#2)[#3] {
  \draw[blue,-{>[length=4pt,width=2.5pt]}] (1.7,0.5-#1) to[out=#3,in=-#3] (1.7,0.5-#2);
}

\def\world(#1)#2#3{
  \draw[thick,#3](-1,-#1) -- (1.5,-#1) node[anchor=west]{\vb{#2}};
}

\def\kon(#1){ 
  \draw[draw=none] (0.2,-#1) rectangle node[align=center]{\vb{\ldots}} (1.2,1-#1);
}
  
\def\godot#1{\filldraw[blue,scale=0.6](#1) circle (0.4ex and 1.4ex);}

\begin{tikzpicture}[yscale=0.4,xscale=1.4]

  \var(0){a.data}{}
  \var(1){a.size}{10}
  \var(2){a.capacity}{15}
  \world(-1){\vb{vector<T> a;}}{orange}
  \kon(3)

  \godot{v0}
  \draw[blue,-{>[length=8pt,width=5pt]}] (v0) to [out=0,in=180] (3,-2);

  \begin{scope}[shift={(3,-2.5)},xscale=0.285]
    \foreach \i in {10,...,14} {
      \draw[thin,gray](\i,0) rectangle (\i+1,1);
    \node[thin,gray,anchor=south] at (\i+0.5,1){{\scriptsize$\i$}};
    }
    \foreach \i in {0,...,9} {
      \draw[magenta](\i,0) rectangle (\i+1,1);
    \node[magenta,anchor=south] at (\i+0.5,1){$\i$};
    }
  \end{scope}

\end{tikzpicture}



Tu je niekoľko\footnote{%
  Na stránkach ako je napr. 
  \href{https://www.cppreference.com}{\nolinkurl{www.cppreference.com}}
  sa dá nájsť kompletná referencia na všetky funkcie tried z STL.
}užitočných funkcií:



\def\xx#1{\textcolor{magenta}{\prg!#1!}}
\begin{tabularx}{\textwidth}{lX}\toprule
  \xx{vector<T> a;} & vyrobí vektor \vb{a}\\
  \xx{vector<T> a(n);} & vyrobí vektor \vb{a} a alokuje $n$ prvkov\\
  \xx{vector<T> a(n,v);}& vyrobí vektor \vb{a}, alokuje $n$ prvkov a 
              nastaví ich na $v$ (napr. \prg!vector<int> a(5,42);!\\\midrule
  \xx{a[i]} & $i$-ty prvok\footnote{Podobne ako pri prístupe do poľa, je to tzv. {\em referencia}, takže sa dá priraďovať \prg!a[i]=16;!. Ako také niečo urobiť ti poviem neskôr.}\\\midrule 
  \xx{b = a} & 
  kopíruje celý vektor\footnote{Opäť, neskôr sa dozvieš, ako také niečo urobiť}, t.j.
   podobné ako riešenie úlohy~\ref{uloha:arraycopy}\\\midrule
  \xx{a.data()} & pointer (\prg!T*!) na dynamicky alokované pole\\\midrule
  \xx{a.size()} & počet prvkov aktuálne uložených v poli\\
  \xx{a.capacity()} & počet prvkov, na ktoré je rezervovaná pamäť\\\midrule
  \xx{a.reserve(n)} & realokuje pole (podobne ako náš \vb{Stack}), aby malo 
        kapacitu $n$\\\midrule
  \xx{a.shrink_to_fit()} & realokuje pole, aby kapacita
  presne zodpovedala počtu prvkov \\\midrule
  \xx{a.clear()} & vymaže všetky prvky, kapacita sa nezmení \\\midrule
  \xx{a.resize(n)} & zmení počet alokovaných prvkov na $n$; ak sa veľkosť zmenší,
   prvky, ktoré sa do poľa nezmestia, sa vymažú \\ 
   \xx{a.resize(n,v)} & to isté, ale ak sa dopĺňajú prvky, ich hodnota je \vb{v}\\\midrule
  \xx{a.push_back(x)} & pridá \vb{x} na koniec; ak by nestačila kapacita, realokuje 
  pole na väčšie\\
  \xx{a.pop_back()} & odstráni posledný prvok\\\bottomrule
\end{tabularx}

\begin{uloha}
  \label{uloha:faktorial}\indexItem{Mat}{faktoriál}
  Faktoriál z čísla $n$ (zapisuje sa $n!$) je číslo $2\cdot3\cdot4\cdots n$. Napr.
$4!=2\cdot3\cdot4=24$, $6!=4!\cdot5\cdot6=720$. Napíš program, ktorý
na vstupe dostane číslo $n$ a vypíše $n!$. Pozor, $n$ môže byť tak veľké,
že výsledok sa nezmestí ani do premennej typu \prg!unsigned long long int!.
  Napríklad pre vstup \vb{123} má program vypísať \vb{
  1214630436702532967576624324188129585545421708848338231532891816182\\
  923589236216766883115696061264020217073583522129404778259109157041165147218602951\\
  9906261646730733907419814952960000000000000000000000000000} (do jedného riadku).
\end{uloha}


\indexItem{Prg}{typ \vb{string}}
 Veľmi podobný typu \vb{vector} je typ \vb{string} (
\prg!#include <string>!): má rovnaké funkcie
ako \prg!vector<char>! a aj niečo navyše. Napríklad má operátor \prg!>>!
pre načítanie zo vstupu: program \prg!string s; cin >> s;! prečíta zo vstupu
znaky (až po prvý whitespace), alokuje pamäť v reťazci \vb{s} a uloží ich tam.
Užitočná je aj funkcia \vb{getline(cin, s)}, ktorá do reťazca \vb{s}
prečíta riadok zo vstupu (aj s medzerami). Občas sa hodí použiť aj 
verzia s troma parametrami \vb{getline(cin, s, delim)} ktorá za ukončovač
riadku považuje znak \vb{delim} (napr. \prg!getline(cin, s, ',')! uloží do \vb{s} reťazec zo vstupu po
prvú čiarku). K reťazcu sa dá pripojiť ďalší reťazec,
jeho časť, alebo pole znakov pomocou metódy \vb{append}, napr.
\prg!s.append("kuk").append(t,2,4)! do \vb{s} najprv\footnote{%
  Všimni si, ako sa dajú volania \prg!append! reťaziť. Je to preto, že metóda 
  \prg!append! vracia {\em referenciu} na daný \prg!string!; viac sa o tom
  dozvieš v ďalšej kapitole.
}pripojí (pole znakov) 
\prg!"kuk"! a potom úsek 4 znakov z reťazca \vb{t} počnúc od pozície \prg!t[2]!.
Podreťazec (substring) sa dá vyrobiť \prg!t=s.substr(2,4);! Metóda \vb{find} 
vráti prvý výskyt podreťazca od danej pozície: ak \prg!s="barbakan"!, tak
\prg!s.find("ba")! vráti 0 (rovnako ako \prg!s.find("ba",0)!)
a \prg!s.find("ba",1)! vráti 3. Ak sa podreťazec v reťazci nenachádza, vráti 
sa špeciálna hodnota \vb{string::npos}. 


\begin{uloha}
  Predpokladajme, že máme premennú typu \prg!string!, v ktorej je uložený
  zápis čísla (napr. reťazec \prg!"4247"!). Napíš funkciu, ktorá vráti premennú
  typu \prg!int!, v ktorej bude hodnota tohoto čísla.
\end{uloha}

 
Takáto funkcia je už v knižnici \prg!string! naprogramovaná a volá sa \prg!stoi!
\footnote{{\em string-to-integer}, podobná funkcia je \prg!stol!
ktorá vracia \prg!long!.}; dá sa napísať \prg!int x = stoi(s);!. 
V kombinácii s \prg!find! a \prg!substr! sa \prg!stoi! dá použiť na čítanie
čísel z reťazca. Keď treba čítať zložitejšie vstupy,\indexItem{Prg}{\vb{stringstream}}\phantomsection\label{here.stringstream}
hodí sa  trieda \prg!stringstream! (\prg!#include <sstream>!).
Premenná typu \vb{stringstream} funguje podobne ako \prg!cin! a \prg!cout!, ale
namiesto štandardného vstupu a výstupu vie čítať a zapisovať do reťazca. Má špeciálnu metódu \vb{str()}, ktorou sa dá zistiť aj nastaviť
reťazec, nad ktorým práve pracuje. Ukážme si to na príklade:

\begin{uloha}
  Na vstupe je viacero riadkov, ktoré obsahujú rôzne znaky (okrem \vb{\#}).  
  Medzi nimi sú rôzne povkladané aktívne úseky ohraničené \vb{\#}, ktoré obsahujú 
  iba čísla a medzery. Vstup je taký, že ak napíšeme čísla z aktívnych
  úsekov za sebou dostaneme číslice čísla $\pi$, t.j. \vb{31415926535....}.
  Napíšte porgram, ktorý vyberie čísla z aktívnych úsekov a v premennej typu \prg!double!
  uloží číslo $\pi$.

 Príklad vstupu

\begin{outputBox}
1 K3d s0m 1s13l # 3 14 # c3z h0ru#1 5#
2
3 str3t0l  #    9#s0m t4m
4 p0tv0ru#2##65#!!
  \end{outputBox}
\end{uloha}

Príklad riešenia je tu:

\begin{lstlisting}[label=sstream]
#include <iostream>
#include <sstream>
#include <string>
using namespace std;

int main() {
  string s;
  stringstream out;
  double pi;

  out << "0.";
  while (cin.good()) {@\ll1@
    getline(cin, s);
    int i = 0, j;
    while ((i = s.find("#", i)) != string::npos) { @\ll2@
      j = s.find("#", i + 1);
      stringstream in;
      in.str(s.substr(i + 1, j - i - 1));
      int n;
      for (in >> n; !in.fail(); in >> n) @\ll4@
        out << n;
      i = j + 1; @\ll3@
    }
  }
  out >> pi;
  pi = 10 * pi;
  cout << out.str() << endl;
  cout << pi << endl;
}
\end{lstlisting}

Urobil som si  premennú typu \vb{stringstream}
na pozliepanie úsekov zo vstupu. Streamy (\prg!cin!, \prg!cout! aj premenné typu 
\vb{stringstream}) majú metódy \prg!bool good()! a \prg!bool fail()!, ktoré
hovoria, či je stream pripravený čítať a či sa posledné čítanie z neho podarilo.
Preto hlavný \prg!while! cyklus na riadku~\ref{sstream-1} číta zo vstupu riadky, kým tam nejaké sú
a volanie \vb{getline(cin,s)} ich ukladá do reťazca \vb{s}.
Vo vnútornom cykle na riadku \ref{sstream-2}  nastavím 
\vb{i}  a \vb{j}  na pozície 
\footnote{\indexItem{Prg}{priradenie ako výraz}
Tu som použil čudný zápis \hbox{\vb{while ((i=x) == y) ...}}
Priradenie je predsa príkaz. Áno, ale dá sa použiť aj ako výraz a vtedy ako výsledok vracia priradenú hodnotu.
V predchádzajúcom zápise to znamená \cmd{Do i priraď x a ak priradená hodnota bola y, pokračuj vo vykonávaní cyklu}.
}
dvoch po sebe idúcich znakov \vb{\#} (všimni si \vb{i = j + 1;} na konci
cyklu na riadku \ref{sstream-3}). Potom
si vyrobím \vb{stringstream}, ktorém nastavím podreťazec
\vb{s} medzi \vb{i} a \vb{j} ako vstupný reťazec 
volaním \vb{in.str(...)}. V najvnútornejšom cykle na riadku \ref{sstream-4}
čítam z \vb{in} čísla
a zapisujem ich do \vb{out} bez medzier. Po dočítaní vstupu
bude reťazec \vb{out.str()} obsahovať \prg!"0.31415..."!,
preto z \vb{out} môžem prečítať premennú typu \prg!double!.
Všimni si, že \vb{out.str()} aj po načítaní obsahuje celý buffer, znaky
z neho sa pri čítaní nestrácajú. Keby som ho chcel vyprázdiť, mohol by som použiť \prg!out.str("")!.

Streamy sú urobené tak\footnote{%
  Ako je to presne urobené ti poviem v kapitole~\ref{sect:kompresia}.
}, že sa vedia tváriť ako typ \vb{bool}, takže kratšia verzia vyzerá takto:

\begin{lstlisting}
#include <iostream>
#include <sstream>
#include <string>
using namespace std;
int main() {
  string s;
  stringstream out;
  double pi;
  out << "0.";
  while (getline(cin, s)) {
    int i = 0, j;
    while ((i = s.find("#", i)) != string::npos) {
      j = s.find("#", i + 1);
      stringstream in(s.substr(i + 1, j - i - 1));
      int n;
      while (in >> n) out << n;
      i = j + 1;
    }
  }
  out >> pi;
  pi = 10 * pi;
  cout << out.str() << endl;
  cout << pi << endl;
}
\end{lstlisting}

\begin{uloha}
  Na vstupe sú dve čísla. Napíš program, ktorý vypíše ich súčin. Pozor, čísla 
  môžu byť ľubovoľne veľké.
\end{uloha}

