\chapter{Algoritmy v STL}
\label{sect:algstl}

Pri riešení úloh si si isto všimol, že sú niektoré užitočné funkcie, ktoré sa
často opakujú: utriediť pole, nájsť maximum alebo minimum, vymeniť dve premenné,
vyhľadať hodnotu v utriedenom poli a podobné. STL obsahuje knižnicu \vb{<algorithm>},
kde sú rôzne šikovné algoritmy, ktoré ti pomôžu písať programy rýchlejšie. Napríklad
ak máš premennú \prg!vector<int> a;! tak ju vieš utriediť pomocou
\prg!sort(a.begin(), a.end());! V tejto kapitole by som ti chcel vysvetliť, ako 
a prečo to 
funguje.

 Cieľom návrhárov STL bolo, aby algoritmy boli čo najvšeobecnejšie a dali 
sa používať v rôznych situáciách. Preto sa ich snažia oddeliť od tzv. {\em kontajnerov},
čo sú typy, ktoré uchovávajú dáta.
Napr. typ \vb{vector}
má dáta uložené v jednom dynamicky alokovanom poli, ale mohol by ich mať napr. v piatich
rôznych poliach, alebo hocijako inak.
Ak chceš napr. utriediť
pole, nepotrebuješ na to presne vedieť, ako sú prvky uložené, ale potrebuješ sa v nich vedieť
prehŕňať: nejaký konkrétny prvok si zapamätať, vedieť prejsť na ďalší prvok, prejsť
na začiatok, povedať, či dva zapamätané prvky sú ten istý a tak podobne. \indexItem{Prg}{iterátory}
Na to slúžia  tzv. {\em iterátory}. Iterátor nie je súčasť jazyka (ako 
napr. príkazy a výrazy), je to len označenie: iterátor je hocijaká trieda, ktorá sa
vie správnym spôsobom prehŕňať v dátach inej triedy.


Každý kontajner, ktorý chce používať algoritmy STL, musí mať dve metódy \prg!begin()!
a \prg!end()!, ktoré vracajú iterátor. Dajme tomu, že máš triedu 
\prg!struct Puch{ int a,b,c; };! a chceš tie tri čísla vedieť utriediť volaním funkcie
\prg!sort(p.begin(), p.end());! z STL. Potreboval by si spraviť niečo ako \indexItem{Prg}{vnorené typy}

\begin{lstlisting}
struct IteratorPuchu {
  ...
};

struct Puch {
  int a,b,c;
  IteratorPuchu begin() { ... }
  IteratorPuchu end() { ... }
};
\end{lstlisting}

 Pretože \vb{IteratorPuchu} logicky patrí triede \vb{Puch}, využíva sa možnosť C++
definovať typy vovnútri iných typov takto:

\begin{lstlisting}
struct Puch {

  struct iterator {
    ...
  };

  int a,b,c;
  iterator begin() { ... }
  iterator end() { ... }
};
\end{lstlisting}

Teraz je \vb{iterator} súčasťou typu \vb{Puch} a celým menom sa volá 
\prg!Puch::iterator! (dve dvojbodky znamenajú \cmd{patriaci}).
Typ \vb{vector} to má urobené rovnako, môžeš skompilovať

\begin{lstlisting}
vector<int> a;
vector<int>::iterator b = a.begin();
\end{lstlisting}

 Funkcie, ako napr. \vb{sort} sú v skutočnosti šablóny, ktoré vyrobia príslušnú
funkciu pre hocijaký typ iterátora. Napr. \vb{sort} je definovaná ako\footnote{%
  to, že parameter v šablóne sa volá \prg!class! a nie \prg!typename! si  nevšímaj,
  znamená to skoro to isté}

\begin{lstlisting}
template<class RandomIt>
void sort(RandomIt first, RandomIt last);
\end{lstlisting}

Pre triedenie poľa by si mal preto zavolať
\prg!sort<vector<int>::iterator>(a.begin(), a.end());!
To tiež funguje, ale stačí zavolať \prg!sort(a.begin(), a.end());! Je to preto, 
že ak si kompilátor vie odvodiť typ, dosadí ho do šablóny sám. V tomto prípade vidí,
že funkcia \vb{begin()} triedy \prg!vector<int>! vracia typ \prg!vector<int>::iterator!,
takže je jasné, že toto je typ, ktorý treba použiť do šablóny.
Toto odvodzovanie typov, ktoré kompilátor robí sa dá využiť aj pri definovaní premenných:
ak namiesto mena typu použiješ špeciálne meno \prg!auto!, kompilátor sa pokúsi odvodoť typ;\indexItem{Prg}{\vb{auto}}
napr. namiesto \prg!vector<int>::iterator b = a.begin();! môžeš
napísať \prg!auto b = a.begin();!

 Každá funkcia z knižnice \vb{algorithm} má povedané, čo presne očakáva od iterátora.
Ale vždy treba, aby iterátor vedel vrátiť hodnotu operátorom \vb{*}, aby sa mal operátor
\vb{++}, prípadne operátor \vb{+} na posuntie o \vb{i} prvkov ďalej
a pod. K iterátoru sa preto môžeš správať ako k pointru (a napr. pre typ 
\prg!vector<int>! to skutočne aj pointer na \prg!int! je). Všetky funkcie STL pracujú na
{\em polouzavertých intervaloch}: začiatočný iterátor je na prvý prvok, koncový iterátor
je na prvok za posledným. 


\def\var(#1)#2#3{%
  \draw (0.2,-#1) rectangle node[align=center](v#1) {\vb{#3}} (1.2,1-#1);
  \draw (0,0.5-#1) node[anchor=east]{\textcolor{magenta}{\vb{#2}}};
}

\def\ptr(#1,#2)[#3] {
  \draw[blue,-{>[length=4pt,width=2.5pt]}] (1.7,0.5-#1) to[out=#3,in=-#3] (1.7,0.5-#2);
}

\def\world(#1)#2#3{
  \draw[thick,#3](-1,-#1) -- (1.5,-#1) node[anchor=west]{\vb{#2}};
}

\def\kon(#1){ 
  \draw[draw=none] (0.2,-#1) rectangle node[align=center]{\vb{\ldots}} (1.2,1-#1);
}
  
\def\godot#1{\filldraw[blue,scale=0.6](#1) circle (0.4ex and 1.4ex);}

\begin{tikzpicture}[yscale=0.4,xscale=1.4]

  \var(0){a.data}{}
  \var(1){a.size}{10}
  \var(2){a.capacity}{15}
  \world(-1){\vb{vector<T> a;}}{orange}
  \kon(3)

  \godot{v0}
  \draw[blue,-{>[length=8pt,width=5pt]}] (v0) to [out=0,in=180] (3,-2);

  \begin{scope}[shift={(3,-2.5)},xscale=0.285]
    \foreach \i in {10,...,14} {
      \draw[thin,gray](\i,0) rectangle (\i+1,1);
    \node[thin,gray,anchor=south] at (\i+0.5,1){{\scriptsize$\i$}};
    }
    \foreach \i in {0,...,9} {
      \draw[magenta](\i,0) rectangle (\i+1,1);
    \node[magenta,anchor=south] at (\i+0.5,1){$\i$};
    }
    \draw[blue,-{>[length=8pt,width=5pt]}] (0.5,-2) 
    node[magenta,anchor=north] {\vb{a.begin()}}-- (0.5,-0.2);
    \draw[blue,-{>[length=8pt,width=5pt]}] (10.5,-2) 
    node[magenta,anchor=north] {\vb{a.end()}}-- (10.5,-0.2);
  \end{scope}

\end{tikzpicture}

 Takže tieto programy robia to isté

\begin{lstlisting}
for (int i = 0; i < a.size(); i++) 
  cout << a[i] << endl;

for (int *p = a.data(); p < a.data() + a.size(); ++p) 
  cout << *p << endl;

for (vector<int>::iterator it = a.begin(); it < a.end(); it++)
  cout << *it << endl;

for (auto it = a.begin(); it < a.end(); it++)
  cout << *it << endl;
\end{lstlisting}

Len pripomeniem, že iterátory spravidla obsahujú pointre do príslušnej dátovej štruktúry, 
a preto, rovnako ako pointre, môžu prestať byť aktuálne:
ak napr. urobíš \prg!auto it = a.end();! a potom \prg!a.push_back(3);! tak
\vb{it} bude ukazovať stále na to isté miesto a nie za koniec vektora \vb{a}.
V dokumentácii k jednotlivým kontajnerom sa píše, ktoré iterátory sú pri ktorých operáciách
{\em zneplatnené} (invalidated). Napr. pri \prg!push_back! sa píše: {\em ak pri operácii
bolo nutné realokovať pole, zneplatnia sa všetky iterátory, inak iba posledný}.
To treba mať na pamäti: častá chyba je, že sa v cykle 
\prg!for (auto it = a.begin(); it < a.end(); it++)! prechádza nejaký kontajner
a vnútri cyklu sa robia operácie, ktoré môžu iterátory zneplatniť. 

 Ak ťa zaujíma, čo od iterátora potrebuje napr. funkcia \vb{sort}, v súbore
\link{\rootpath/puch.h}{puch.h}
je dorobený \vb{Puch::iterator} zo začiatku kapitoly. Program

\begin{lstlisting}
#include <algorithm>
#include <iostream>

#include "puch.h"
using namespace std;

int main() { 
  Puch p(12, 11, 9);
  sort(p.begin(), p.end()); 
  for (auto it = p.begin(); it < p.end(); ++it) cout << *it << " ";
  cout << endl;
} 
\end{lstlisting}


\begin{lrbox}{\TmpBox}
\begin{lstlisting}[basicstyle=\scriptsize\roboto]
namespace Kacka {
  int zobak;
  void kvak() { cout << "kvak" << endl; }
} 
\end{lstlisting}
\end{lrbox}


vypíše \vb{9 11 12}. Ak sa pozrieš dovnútra, je tam jedna konštrukcia, ktorú som ti zatiaľ\indexItem{Prg}{type alias \vb{using}}
neukazoval: tzv. {\em type alias}. Ak napíšeš\footnote{\indexItem{Prg}{\vb{namespace}}
Podobný alias používame, keď v každom programe píšeme \prg!using namespace std;! na to, aby sme si ušetrili písanie dvojbodiek.
Zápis s dvojbodkou  sme už videli pri vnorených typoch, napr. \prg!vector<int>::iterator! je typ \vb{iterator} definovaný vovnútri typu 
\prg!vector<int>!. Podobne je to aj s funkciami, \vb{Puch::begin()} je celé meno funkcie \vb{begin()} definovanej v type \vb{Puch}. Ak
píšeme metódy typu \vb{Puch}, môžeme ju volať jednoducho iba \vb{begin()}, ale pri volaní zvonka by sme potrebovali písať celé meno.
Konštrukcia \vb{namespace} umožňuje schovať viacero typov, premenných a funkcií do jedného celku so spoločným menom. Je to hlavne kvôli tomu, aby 
sa vo veľkých programoch zabránilo konfliktom, keby sa vyskytlo to isté meno na viacerých miestach. Keby si napísal

\usebox{\TmpBox}

tak v programe bude premenná \prg!Kacka::zobak! a funkcia \prg!Kacka::kvak()!. Nie je to ale typ ako pri \vb{struct} (ani na konci definície nie je bodkočiarka),
takže nemôžeš mať premennú \prg!Kacka c;!

Všetky typy z STL sú v \vb{namespace std}, takže by sme mali písať \vb{std::vector}, \vb{std::cout}, \vb{std::endl} a pod. Ak napíšeš \hbox{\vb{using namespace X}} znamená to,
že všetky veci z \vb{X} chceš vedieť používať aj ich skrátenými menami.
} \prg!using INT = unsigned long int;!
tak \vb{INT} bude všade v programe skratka za \prg!unsigned long int!. 

 Tu je niekoľko užitočných funkcií z knižnice \vb{algorithm} 


\def\xx#1{\textcolor{magenta}{\prg!#1!}}
\begin{tabularx}{\textwidth}{lX}\toprule
  \xx{find} & \prg!it find( it first, it last, const T\& val )! \newline
              Vráti iterátor na prvý výskyt
              hodnoty \prg!val! v intervale 
              \prg![first, last)! alebo \prg!last! ak sa tam \prg!val! nenachádza.\\\midrule
  \xx{search}&\prg!it search( it first, it last, it s_first, it s_last )! \newline
  vráti iterátor na prvý výskyt 
  postupnosti \prg![s_first, s_last)! v intervale \prg![first, last)!
  \\\midrule
  \xx{min_element}&\prg!it min_element(it first, it last)!\newline
  vráti iterátor na najmenší prvok z rozsahu \prg![first, last)!\\
  \xx{max_element}&\prg!it max_element(it first, it last)!\newline
  vráti iterátor na najväčší prvok z rozsahu \prg![first, last)!\\\midrule
  \xx{min}&\prg!T min(T a, T b)! alebo \prg!T min(\{T a1, ... , T an\})!\newline
  Vráti  najmenší prvok  (napr. \prg!min(2, 3)! alebo \prg!min(\{2, 3, 4, 5\})!)\\
  \xx{max}&\prg!T max(T a, T b)! alebo \prg!T max(\{T a1, ... , T an\})!\newline
  Vráti najväčší prvok.\\\midrule
  \xx{fill}&\prg!void fill(it first, it last, t val)!\newline
  Vyplní interval \prg![first, last)! hodnotou \vb{val}\\\midrule
  \xx{copy}&\prg!it copy(it first, it last, it out)!\newline
  Skopíruje interval \prg![first, last)! od iterátora \prg!out! (musí tam byť miesto).
  \newline
  \prg!vector<int> b(a.size());!\newline
  \prg!copy(a.begin(), a.end(), b.begin());!
  \\
  \xx{copy_backward} & to isté, ale skopíruje interval tak, že končí za iterátorom
  \vb{out}\newline
  \prg!vector<int> b(a.size());!\newline
  \prg!copy(a.begin(), a.end(), b.end());!
  \\\midrule
  \xx{merge}&\prg!it merge(it first1, it last1,
                it first2, it last2,
                it out)!\newline
  Urobí operáciu \vb{merge} na dvoch utriedených intervaloch a výsledok
  uloží od iterátora \vb{out} (musí tam byť miesto). 
  Vráti iterátor za posledný vložený prvok.
  \newline
  \prg!vector<int> c(a.size() + b.size());!\newline
  \prg!merge(a.begin(), a.end(), b.begin(), b.end(), c.begin());!
   \\\midrule
   \xx{replace}&\prg!it replace(it first, it last, T\& old_value, T\& new_value)!\newline
   Nahradí všetky výskyty hosnoty \prg!old_value! hodnotou \prg!new_value!\\\midrule
   \xx{swap}&\prg!void swap(T\& a, T\& b)!\newline
   Vymení hodnoty v premenných \vb{a}, \vb{b}.\\\midrule
   \xx{reverse}&\prg!void reverse(it first, it last)!\newline
   Otočí interval \prg![first, last)!\\\midrule
   \xx{sort}&\prg!void sort(it first, it last)!\newline
   Utriedi interval \prg![first, last)!\\\midrule
   \xx{binary_search}&\prg!bool binary_search(it first, it last, const T\& val)!\newline
   Zistí, či sa v utriedenom poli v rozsahu \prg![first, last)! vyskytuje hodnota \prg!val!\\
   \xx{lower_bound}&\prg!it lower_bound(it first, it last, const T\& val)!\newline
   Vráti iterátor na prvú hodnotu v rozsahu \prg![first, last)!, ktorá je väčšia alebo
   rovná ako \prg!val!\\
   \xx{upper_bound}&\prg!it upper_bound(it first, it last, const T\& val)!\newline
   Vráti iterátor na prvú hodnotu v rozsahu \prg![first, last)!, ktorá je väčšia 
   ako \prg!val!\\
  \bottomrule
\end{tabularx}

