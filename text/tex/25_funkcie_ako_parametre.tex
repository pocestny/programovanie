%\section*{Funkcie ako parametre}
\chapter{Funkcie ako parametre}
\label{sec:lambdy}

Funkcie z knižnice \vb{algorithm} sú užitočné, ale ak sa napr. vrátiš
k úlohe \ref{uloha:sortbody}, zistíš, že stále máš problém.
Dajme tomu, že máš typ
\begin{lstlisting}
struct Bod {
  double x,y;
};
\end{lstlisting}
a chceš utriediť pole bodov podľa $x$-ovej súradnice. Ak skúsiš napísať
\begin{lstlisting}
vector<Bod> a;
sort(a.begin(), a.end());
\end{lstlisting}
dostaneš chybovú správu, že typ \vb{Bod} nemá operátor \vb{<}, podľa ktorého
by sa mohol triediť. Môžeš to vyriešiť takto:
\begin{lstlisting}
struct Bod {
  double x,y;
  bool operator<(Bod p) { return x < p.x; }
};
\end{lstlisting}
ale vzápätí sa dostaneš do problémov, ak chceš teraz utriediť body podľa $y$-ovej súradnice.
Hodil by sa nejaký mechanizmus, ako funkcii \vb{sort} povedať, podľa čoho má triediť.
Keďže to má byť všeobecná funkcia, najlepšie by bolo, keby sa jej, ako ďalší
parameter, dala posunúť funkcia, ktorú má použiť na porovnávanie dvoch hodnôt
namiesto operátora \vb{<}.

 Keď skompiluješ program, vytvorí sa z neho postupnosť príkazov pre procesor,
ktorá sa pri spustení programu uloží do pamäte, podobne ako sme to mali pri riešení
úlohy~\ref{uloha:logo}. Keď sa v programe volá funkcia, vytvorí sa v pamäti
pre ňu nový svet, a procesor začne spracovávať príkazy na tom mieste pamäte,
kde je program pre danú funkciu. Keď funkcia skončí, program pokračuje vo vykonávaní
príkazov tam, kde prestal. Funkcia je teda, podobne ako premenná, tiež len postupnosť
núl a jednotiek uložená niekde v pamäti, a teda, podobne ako premenná, má adresu.
Táto adresa sa dá uložiť do premennej typu {\em pointer na funkciu}. Pretože pri vytváraní\indexItem{Prg}{pointer na funkciu}
sveta funkcie treba vedieť, aké má parametre, pre každú kombináciu parametrov existuje
iný typ, napr. typ \cmd{pointer na funkciu, ktorá má jeden parameter \prg!int! a vracia
typ \prg!bool!}. Premenné typu {\em pointer na funkciu} sa zapisujú trochu špeciálnym 
spôsobom. Namiesto \vb{typ nazov;} ako pri \prg!int x;! sa napíše definícia funkcie,
ale namiesto názvu má hviezdičku a názov premennej. Takže napr. 
\prg!bool (*a)(int*);! vyrobí premennú \vb{a} ktorá bude typu \cmd{pointer na funkciu, ktorá
má parameter \prg!int *! a vracia \prg!bool!}. Priradiť do takejto premennej sa dá
adresa funkcie, t.j. ak máš v programe napr. 
\prg!bool parne(int* x) { return (*x) % 2 == 0; }!, 
tak môžeš napísať \prg!a=&parne;!, prípadne ten ampersand môžeš aj vynechať a písať
rovno \prg!a=parne;!. Premenná \vb{a} teda bude pointer na funkciu, a jeho hodnota
bude samotná funkcia, ktorú môžeš zavolať, napr. \prg!(*a)(&x);! (rovnako môžeš
hviezdičku vynechať a funguje aj zápis \prg!a(&x);!. Napokon, ak chceš viackrát 
použiť ten istý typ, je príjemné si naňho urobiť skratku pomocou \prg!using!.
V nasledujúcom programe je \vb{Funkcia} typ \cmd{pointer na funkciu, ktorá má
parameter \prg!int! a vracia \prg!int!}. Funkcia \vb{urob} má parameter \prg!int!
a pointer na funkciu, ktorú zavolá. Pri volaní \vb{urob(20, pridaj)} sa vypíše $30$
a pri volaní \vb{urob(20, uber)} sa vypíše $10$.


\begin{lstlisting}
#include <iostream>
using namespace std;

using Funkcia = int (*)(int);

int pridaj(int x) { return x + 10; }
int uber(int x) { return x - 10; }

void urob(int x, Funkcia f) {
  cout << f(x) << endl;
}

int main() {
  urob(20, pridaj);
  urob(20, uber);
}
\end{lstlisting}

Tento mechanizmus funnguje aj v jazyku C a častokrát sa používa v rôznych knižniciach na
tzv. {\em callbacks}. Napríklad knižnica na prácu so sieťou môže mať funkciu, ktorá čaká, 
kým prídu nejaké dáta a potom zavolá callback, ktorý dostala ako parameter, na ich spracovanie.


V C++ sú aj ďalšie možnosti, ako odovzdať funkciu ako parameter, napr. pomocou tzv.\indexItem{Prg}{funktor}
{\em funktorov}. Funktor je len honosne znejúce slovo pre triedu, ktorá má definovaný 
operátor\vb{()}. Tento program robí to isté, čo ten predchádzajúci:

\begin{lstlisting}
#include <iostream>
using namespace std;

struct Pridaj {
  int operator()(int x) { return x + 10; }
};

struct Uber {
  int operator()(int x) { return x - 10; }
};

template <typename T>
void urob(int x, T f) {
  cout << f(x) << endl;
}

int main() {
  urob(20, Pridaj());
  urob(20, Uber());
}
\end{lstlisting}

Treba to čítať tak, že \vb{Pridaj} je trieda, ktorá nemá žiadne premenné
a má jedinú metódu, ktorou je operátor \vb{()}. Ten berie ako parameter \prg!int!
a vracia \prg!int!. Funkcia \vb{urob} je šablóna, ktorá má prvý parameter \prg!int!
a druhý akéhokoľvek typu \vb{T}. Ten typ ale musí mať príslušný operátor, aby sa dalo
zavolať \vb{f(x)}; keby si skúsil vyrobiť typ \prg!urob<int>! kompilátor vyhlási chybu
pri použití šablóny.
V hlavnom programe je volanie \vb{Pridaj()} konštruktor typu \prg!Pridaj! (keďže si žiaden nedefinoval,
kompilátor si spravil default), ktorý vyrobí premennú typu \vb{Pridaj} a pošle
ju ako parameter do funkcie \vb{urob<Pridaj>}, ktorú si vyrobí podľa šablóny. 
Pri volaní \prg!urob<Pridaj>(20,Pridaj())! si môžem typ v šablóne odpustiť, kompilátor si
ho vie domyslieť podľa typu parametra. 


Ukážem ti ešte ďalšiu možnosť, ako dosiahnuť to isté, ktorá bude pre naše potreby\indexItem{Prg}{lambda výrazy}
častokrát najvhodnejšia. Sú ňou tzv. {\em lambda výrazy}\footnote{%
  Nazvané podľa $\lambda$-kalkulu, čo je funkcionálny jazyk navrhnutý v 30tych rokoch
  minulého storočia Alonsom Churchom (\url{https://en.wikipedia.org/wiki/Lambda_calculus})
}, ktoré umožňujú definovať funkciu formou výrazu a pracovať s pointrom na ňu.
Ešte raz ten istý program, tento raz s lambdami:

\begin{lstlisting}
#include <iostream>
using namespace std;

template <typename T>
void urob(int x, T f) {
  cout << f(x) << endl;
}

int main() {
  urob(20, [](int x) { return x + 10; });
  urob(20, [](int x) { return x - 10; });
}
\end{lstlisting}

Skôr ako začneme hovoriť o lambdách, všimni si, že keď sme funkciu \vb{urob}
definovali pomocou šablóny, dá sa použiť s pointrom na funkciu, funktorom, aj lambdou,
čo sú inak všetko rozdielne zvieratá.

 Ako teda fungujú lambdy? Lambda výraz je {\em výraz}, t.j. ako som povedal úplne 
na začiatku: {\em príklad, ktorý
sa počas behu programu vyráta a zistí sa jeho výsledok}. Výsledkom lambda výrazu je 
funkcia: dá sa predstaviť ako pointer na bezmennú funkciu (v skutočnosti to tak aj je),
ale v programe má špeciálny typ \vb{function}, definovaný v knižnici \vb{functional}.\indexItem{Prg}{šablóna \vb{function}}
Typ \vb{function} je šablóna, ktorá sa dá použiť napr. takto:

\begin{lstlisting}
#include <functional>
using namespace std;

int main() {
  function<int(int)> f = [](int x) { return x + 42; };
}
\end{lstlisting}


\vb{f} je typu \cmd{funkcia, ktorá má parameter \prg!int! a vracia \prg!int!}.
Keď kompilátor vidí tento zápis, vyrobí v programe kód pre funkciu, ktorá vráti 
parameter zväčšený o 42 a pointer na túto funkciu priradí do premennej \vb{f}.
Hranaté zátvorky \vb{[]} hovoria, že ide o lambda výraz. Potom nasledujú
v normálnych zátvorkách parametre, ako pri bežnej funkcii a v kučeravých zátvorkách program
tela funkcie. Keby bolo treba povedať aj návratový typ (napríklad by nebolo
jasné, či sa má vracať \prg!int! alebo \prg!double!), dá sa to urobiť pomocou \vb{->}
takto:

\begin{lstlisting}
function<double(int)> f = [](int x) -> double { return x + 42; };
\end{lstlisting}

V oboch prípadoch sa \vb{f} dá zavolať ako normálna funkcia, t.j. \vb{f(10)}. 

 Hranaté zátvorky nie sú iba značenie lambda výrazu, dovnútra sa píšu tzv.\indexItem{Prg}{captures} 
{\em captures}. Dajú sa predstaviť ako parametre, ktoré sa zafixujú v čase, keď sa
funkcia vytvára (na rozdiel od normálnych parametrov, ktoré sa zafixujú až pri volaní).
Dajme tomu, že chcem urobiť funkciu \vb{zvacsovac}, ktorá dostane parameter \vb{kolko}
a vráti lambdu, ktorá bude mať jeden parameter a pri zavolaní ho zväčší o \vb{kolko}.
Prvý pokus by bol takýto:

\begin{lstlisting}
function<int(int)> zvacsovac(int kolko) {
  return [](int x) { return x + kolko; }; // !! chyba
}
\end{lstlisting}

Tu je problém v tom, že pri vyrábaní lambdy je premenná \vb{kolko} lokálna premenná
funkcie \vb{zvacsovac}: vo vyrobenej lambde ju nevidno (ako v každej funkcii, aj 
v lambde vidno
iba globálne a lokálne premenné). Keď premennú \vb{kolko} uvedieš ako {\em capture},
t.j. spravíš

\begin{lstlisting}
function<int(int)> zvacsovac(int kolko) {
  return [kolko](int x) { return x + kolko; };
}
\end{lstlisting}

bude to fungovať takto: ak sa zavolá napr. \prg!zvacsovac(10)(20)!, vytvorí sa svet
pre funkciu \vb{zvacsovac}, v ktorom bude lokálna premenná \vb{kolko} mať hodnotu $10$.
Volanie \prg!zvacsovac(10)! vráti bezmennú lambda funkciu, v ktorej hodnota \vb{kolko} bude pevne fixovaná
konštanta $10$: funkcia teda pri svojom volaní robí to, že vráti svoj parameter zväčšený o 10.
Pointer na túto funkciu sa vráti ako výsledok volania \prg!zvacsovac(10)! a vzápätí sa funkcia zavolá,
t.j. vytvorí sa nový svet, v ktorom premenná \vb{x} bude nastavená na 20. Pretože 
\vb{kolko} je navždy fixná konštanta 10, bezmenná funkcia vráti 30. Keby nasledovalo
volanie \vb{zvacsovac(30)}, vráti sa nová, nezávislá, bezmenná funkcia, v ktorej je hodnota 
\vb{kolko} fixovaná konštanta 30.

 
Keď sa teda ako {\em capture} napíše meno premennej, napr. \vb{x},
jej aktuálna hodnota sa v lambde 
zoberie ako fixná konštanta. Ako {\em capture} sa dá napísať aj \prg!&x!,
a vtedy sa ako fixná konštanta zoberie referencia (t.j. pointer) na \vb{x}. 
Pozri si tento program:

\begin{lstlisting}
int main() {
  vector<int> a;
  int kde = 0, kolko = 5;

  auto f = [&a, &kde, kolko](int este) { a[kde] = kolko + este; };

  a.resize(5);
  vypis(a);  // 0 0 0 0 0
  f(1);
  vypis(a);  // 6 0 0 0 0
  kde = 3;
  f(2);
  vypis(a);  // 6 0 0 7 0
  kolko = 12234;
  f(2);
  vypis(a);  // 6 0 0 7 0
}
\end{lstlisting}

Hodnota premennej \vb{f} je funkcia s parametrom typu \prg!int!. V tejto funkcii  je vektor \vb{a}
a premenná \vb{kde} prevzatá referenciou, t.j. v čase, keď sa vyhodnocuje 
lambda-výraz (t.j. v našom prípade vtedy, keď sa robí priradenie do premennej \vb{f}),
sa zoberie adresa \vb{a} a adresa \vb{kde}, a vo vzniknutej funkcii sa každá
odvolávka na \vb{a} a \vb{kde} bdue odvolávať na premenné na týchto adresách. Na druhej 
strane, pri premennej \vb{kolko} sa zoberie jej hodnota v čase vytvárania lambdy (t.j. 5),
a všetky výskyty \vb{kolko} sa nahradia touto hodnotou. Preto keď sa zmení premenná 
\vb{kde}, tak nasledujúce volanie \vb{f} bude meniť iný prvok vektora (lebo pri volaní
zoberie hodnotu z adresy \vb{kde}, ktorá sa nezmenila). Ale keď sa zmení premenná 
\vb{kolko}, nič sa nestane, lebo \vb{f} má namiesto \vb{kolko} nastavenú hodnotu 5.


Treba, samozrejme, dávať pozor na to, že referencia môže prestať platiť, napríklad
referencia na lokálnu premennú:

\begin{lstlisting}
function<void()> quak(vector<int>& x) {
  int z = 42;
  // return [\&x, \&z]() \{ x[2] = z; \};   !! problém
  return [&x, z]() { x[2] = z; };   // toto je v poriadku
}
\end{lstlisting}

V prvom prípade sa vyrobí bezmenná lambda, do ktorej sa za \vb{z} zafixuje
adresa premennej \vb{z} zo sveta funkcie \vb{quak}. Lenže \vb{quak} vzápätí skončí
a na tej adrese môže byť čokoľvek. Na záver ešte poznámka: ak chceš do {\em capture}
zoznamu dať všetky lokálne premenné, dá sa použiť \vb{[=]} (resp. \prg![&]!, ak 
sa majú brať ich referencie). Ak sa lambda nachádza v rámci triedy, netreba
zabudnúť dať do {\em capture} zoznamu \prg!this!, inak premenné z triedy nebudú
viditeľné (lambda je nezávislá funkcia\footnote{%
Presnejšie povedané pre fanjnšmekrov: lambda je v skutočnosti iba skrátený zápis pre funktor.
Ak máš v programe premennú \prg!int kvak=17! a zavoláš \prg!auto zaba = [kvak](int x){return kvak+x;}!,
kompilátor vyrobí nejaký typ, napr. 
\prg!struct lambda_el4l59265e {int kvak; operator()(int x){return kvak+x;}};!,
povie si, že \prg!auto zaba! bude \prg!lambda_el4l59265e zaba! a uloží tam premennú typu 
\prg!lambda_el4l59265e!, 
ktorej hodnotu \prg!zaba.kvak! nastaví
na $17$.}).

 Lambda výrazy sú často vhod pri práci s algoritmami v STL. Veľa funkcií
má verzie s dodatočným parametrom, ktorý je funkcia na porovnávanie prvkov.
Špeciálne nás zaujíma funkcia \vb{sort}. Videl si tvar 
\prg!sort(a.begin(),a.end())!, ale dá sa pridať aj tretí parameter, ktorý musí byť 
funkcia, ktorá zoberie dva prvky (takého typu, ako sú v kontajneri, ktorý sa triedi),
a vráti \prg!true!, ak je prvý prvok menší. Napríklad:

\begin{lstlisting}
int main() {
  vector<int> a;
  for (int i = 0; i < 6; i++)
    a.push_back(2 + 2 * (i / 2) - i % 2);

  vypis(a);  // 2 1 4 3 6 5
  sort(a.begin(), a.end());
  vypis(a);  // 1 2 3 4 5 6
  sort(a.begin(), a.end(), [](int a, int b) { return a > b; });
  vypis(a);  // 6 5 4 3 2 1
}
\end{lstlisting}

Ak je treba posielať funkciu ako parameter, treba sa rozhodnúť, či to bude pointer na funkciu, lambda, prípadne nejaký typ funktora. Táto dilema sa častokrát rieši šablónou. Ak si urobíš napr. šablónu

\begin{lstlisting}
template <typename F>
void rob(F f) { cout << f(10) << endl; }
\end{lstlisting}

tak kompilátor bude vedieť vyrobiť funkcie \vb{rob<...>} pre všetky tri verzie. Navyše sa veľakrát nemusíš starať o to, aký presne typ má tvoj parameter, 
lebo kompilátor to pri použití šablóny častokrát zistí. Takže môžeš napr. napísať 

\begin{lstlisting}
int sedem(int x) { return 7; }

struct Uberac {
  Uberac() { y = 42; }
  int operator()(int x) {  y -= x; return y; }
  int y;
};

int main() {
  rob( [](int x) { return x + 5; } );  // zavolá sa rob<function<int(int)> >
  rob( &sedem );                       // zavolá sa rob<int (*)(int)>

  Uberac u;
  rob( u );                            // zavolá sa rob<Uberac>
}
\end{lstlisting}



\begin{uloha}
Na vstupe je $n$ prirodzených čísel. Zoraď ich tak, aby najprv nasledovali všetky
párne čísla utriedené od najmenšieho po najväčšie a potom všetky nepárne čísla
  utriedené od najväčšieho po najmenšie. Napríklad pre vstup \vb{7 3 2 6 4 1 5}
  je výstup \vb{2 4 6 7 5 3 1}.
\end{uloha}

\begin{uloha}
  Na vstupe je číslo $n$ a potom pole $n$ čísel. Napíš program, ktorý zistí, ktoré číslo
  sa najčastejšie opakuje. Napr. pre pole \vb{3 5 7 5 3 2 1 5 6 5} je odpoveď $5$.
\end{uloha}

\begin{uloha}
  Na vstupe je $n$ prirodzených čísel. Pre každé z nich (v tom poradí, ako sa
  nachádzajú na vstupe) vypíš, koľkáte v poradí by bolo, keby sa utriedili.
  Napr. pre vstup \vb{4 10 2 20 1 5 3 6} je výstup \vb{4 7 2 8 1 5 3 6}.
\end{uloha}

\begin{uloha}
  Máme displej, ktorý môže ukazovať kladné aj záporné čísla. Na začiatku ukazuje 0.
  Na vstupe sú čísla $n$, $k$ a potom 
  $n$ čísel, ktoré udávajú, o koľko sa hodnota na displeji zmenila
  každú nasledujúcu sekundu. Napíš program, ktorý vypíše počet takých dvojíc sekúnd
  $i$, $j$, že $i<j\le k+i$ (t.j. $i$ a $j$ sú od seba vzdialené najviac $k$), 
  že displej po $i$ sekundách a po $j$ sekundách
  ukazoval rovnakú hodnotu.
  Napríklad pre vstup $n=9$, $k=4$ a čísla \vb{2 1 2 -2 1 -4 3 1 -4},
  bude po prvej sekunde na displeji hodnota $2$, po druhej hodnota $3$ atď,
  celkovo budú hodnoty na displeji postupne \vb{0 2 3 5 3 4 0 3 4 0}.
  Rovnaké hodnoty sú po nultej, 6. a 9. sekunde (vtedy je hodnota 0), po 2., 4. 
  a 7. sekunde (vtedy je hodnota 3) a po 5. a 8. sekunde (vtedy je hodnota 4).
  Dvojice, koré od seba nie sú ďalej ako $4$ sú preto $(2,4)$, $(4,7)$, $(5,8)$
  a $(6,9)$. Odpoveď je preto $4$. Podobne pre vstup $n=5$ a $k=5$ a čísla 
  \vb{0 0 0 0 0} je odpoveď $15$ 
  (všetky dvojice sú dobré).
\end{uloha}

