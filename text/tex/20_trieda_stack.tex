\chapter{Zásobník ako vlastný typ}
\label{sect:zasobnik}

V kapitole~\ref{sect:stack} sme používali zásobník tak, že sme mali pole fixnej
veľkosti a jednu premennú, ktorá udávala aktuálny počet prvkov v zásobníku. Keby sme chceli
mať viac zásobníkov, mali by sme viac polí a viac premenných. Aby sme v tom udržali
poriadok, je dobré spraviť si vlastný typ, kde budú obe veci pokope, napr.

\vskip 2ex
\vbox{
\begin{lstlisting}[] 
#include <iostream>
using namespace std;

const int N = 100000;

struct Stack {
  int a[N], n;
};
  
int main() {
  Stack s1, s2;

  s1.n = 0;
  s2.n = 0;

  // pridaj prvok
  s1.a[n]=1;
  s1.n++;
}
\end{lstlisting}
}


Takáto definícia má dva problémy: vždy, keď vyrobíme zásobník, vyhradí sa v pamäti
veľké pole, aj keby sme dopredu vedeli, že budeme potrebovať menej. Druhý problém je,
že na pridanie prvku treba dva príkazy, čo pri častom používaní môže byť neprehľadné.
Prvý problém vyriešime dynamickou alokáciou: zásobník si bude pamätať, aké veľké pole
má alokované a na začiatku ho inicializujeme. Inicializáciu aj pridanie prvku zariadime 
funkciami. Tieto musia ako parameter dostávať pointer na zásobník, aby mohli meniť jeho 
premenné\footnote{%
Keby dostali ako parameter namiesto \prg!Stack *s! iba \prg!Stack s!, pri volaní
funkcie sa zásobník (t.j. premenné \prg!a, n, max!)  
skopíruje do lokálneho sveta funkcie, tam sa zmení
a po skončení volania spolu so svetom zanikne. Dynamicky alokované pamäť by ostala,
ale pointer na ňu by sa stratil. To nie je to, čo chceme.
}.


\begin{lstlisting}[] 
#include <iostream>
using namespace std;

struct Stack {
  int *a, n, max;
};
  
void resize(Stack *s, int max) {
  (*s).a = new int[max];
  (*s).max = max;
  (*s).n = 0;
} 

bool push(Stack *s, int x) {
  if ((*s).n == (*s).max) return false;
  (*s).a[(*s).n] = x;
  (*s).n++;
  return true;
} 

void discard(Stack *s) {
  delete[] (*s).a;
  (*s).a = nullptr;
}

int main() {
  Stack s1, s2;

  resize(&s1, 10);
  resize(&s2, 1000);
  push(&s1, 4);
  discard(&s1);
  discard(&s2);

}
\end{lstlisting}

 Funkcia \vb{push} nám teraz kontroluje, aby zásobník nepretiekol: ak je plný, 
prvok nevloží a vráti \prg!false!.
Volanie \vb{(*p).x}, kde \vb{p} je pointer na typ, ktorý má zložku \vb{x} je natoľko\indexItem{Prg}{zápis \vb{->}}%
časté, že má skratku \vb{p->x}. Teda namiesto \prg!(*s).a! môžeš písať \prg!s->a!.
Týmto sme dosiahli, že všetky operácie so zásobníkom sú schované v jednom type a príslušných
funkciách. Keby sme sa rozhodli naprogramovať zásobník nejak inak\footnote{%
  Zásobník je tak jednoduchá dátová štruktúra, že iný spôsob ťažko vymyslieť, ale 
  stretneme aj oveľa zložitejšie veci, kde to bude dávať dobrý zmysel.}
nemusíme meniť zvyšok programu. 



Všetky tri funkcie \vb{resize}, \vb{push}, \vb{discard} sú podobné v tom, že prvý
parameter je pointer na premennú typu \vb{Stack} a funkcia nejakým spôsobom túto premennú 
mení. Pretože logicky patria k typu \vb{Stack}, je možné ich definovať priamo v type.\indexItem{Prg}{trieda, metódy, atribúty}
Typ, ktorý má definované funkcie, sa volá {\em trieda}, jemu priradené funkcie sa volajú
{\em metódy}. Premennej typu trieda sa zvykne hovoriť {\em objekt} a jeho zložkám 
(jednotlivým premenným vovnútri typu) sa občas hovorí {\em atribúty}. Ale to všetko sú
len mená, na ktorých príliš nezáleží\footnote{{\em ''A rose by any other name would 
smell as sweet'' --- W. Shakespeare}}. Dôležité je, že funkcie \vb{resize}, \vb{push}
a \vb{discard} môžeš definovať priamo v type \vb{Stack}. Takéto funkcie
dostávajú prvý ''neviditeľný'' parameter \prg!Stack *this!, ktorý obsahuje pointer
na objekt (premennú), na ktorom pracujú. Jednotlivé atribúty (premenné) sa potom
dajú písať bez zmienky o premennej (t.j. môžeš písať \vb{zzz} namiesto \prg!this->zzz!, 
aj keď to druhé je rovnako správne). Pretože metóda má neviditeľný parameter \prg!this!,
dá sa zavolať iba v spojení s nejakým objektom (premennou).  Na to sa používa rovnaký zápis
pomocou bodky, ako keď sa pristupuje k atribútom (zložkám) objektu (premennej). 
Takže môžeme mať niečo takéto:

\begin{lstlisting}[] 
#include <iostream>
using namespace std;

struct Stack {
  int *a, n, max;

  void resize(int _max) {
    max = _max;
    a = new int[max];
    n = 0;
  }

  bool push(int x) {
    if (n == max) return false;
    a[n] = x;
    n++;
    return true;
  }

  void discard() {
    if (a != nullptr) // aby sme náhodou nevolali delete[] viackrát
      delete[] a;
    a = nullptr;
  }
};

int main() {
  Stack s1, s2;

  s1.resize(10);
  s2.resize(1000);
  s1.push(4);
  s1.discard();
  s2.discard();
}
\end{lstlisting}

Príkaz \prg!s1.resize(10);! zavolá metódu \vb{resize} s parametrom 10 na objekte \vb{s1},
t.j. vyrobí sa nový svet pre funkciu \vb{resize}, 
premenná \vb{\_max} sa v ňom nastaví na $10$ a 
neviditeľná premenná \prg!Stack *this! sa nastaví na \prg!&s1!. Pri vykonávaní funkcie
premenné \prg!max, a, n! nie sú v jej svete definované.  Ale skôr, ako by sa 
začali hľadať medzi globálnymi premennými, skúsi sa
\prg!this->max!, \prg!this->a!, \prg!this->n!. 


Skoro dokonalé, ale ešte tomu zopár vecí chýba. Ak na začiatku zabudneme zavolať \vb{resize}
premenné \vb{a, n, max} budú v nedefinovanom stave, takže potom ďalšie volania budú
robiť paseku. Podobne sa nám môže stať, že zabudneme zavolať \vb{discard} a stratí sa nám
pamäť. Na to, aby sme tieto problémy vuriešili, slúžia \indexItem{Prg}{konštruktor a deštruktor}dve špeciálne metódy: {\em konštruktor}
a {\em deštruktor}. Ak v rámci typu (triedy) definuješ funkciu, ktorá sa volá rovnako ako
typ a nemá návratovú hodnotu\footnote{Konštruktor a deštruktor je jediná výnimka z pravidla,
že funkcia musí mať návratovú hodnotu, keď už nič iné tak aspoň typ \vb{void}}, táto metóda
sa zavolá vždy, keď sa vyrobí nová premenná (či už osamote alebo v poli) hneď potom, ako
sa jej vyhradí v pamäti miesto\footnote{Môžeš skúsiť pridať do konštruktora nejaký výpis
a potom v hlavnom programe zavolať napr. \prg!Stack *a = new Stack[5];!.}. Konštruktorom
vieme zabezpečiť, že náš zásobník nikdy nebude neinicializovaný. Podobne, ak napíšeš
funkciu, ktorej meno je meno typu s prefixom \vb{\textasciitilde}, táto metóda
sa zavolá vždy tesne predtým, ako premenná zanikne (či už preto, že zanikne svet funkcie,
v ktorom vznikla, alebo sa volá \prg!delete[]! na dynamicky alokovanú pamäť, v ktorej 
bola\footnote{%
  Skús si opäť pridať výpis do deštruktora a pozrieť sa, ako sa správa.}.


\begin{lstlisting}[] 
struct Stack {
  int *a, n, max;
  
  Stack() {                 // konštruktor
    a = nullptr;
    n = 0;
    max = 0;
  }

  ~Stack() { discard(); }   // dešstruktor
  
  // zmena veľkosti
  void resize(int _max) {  
    if (a != nullptr) { // ak tam už niečo bolo, skopírujeme
      int *b = new int[_max];
      if (n > _max) n = _max;
      for (int i = 0; i < n; i++) b[i] = a[i];
      discard();        // zahodíme, čo tam bolo doteraz
      a = b; 
    } else { // nemali sme nič naalokované, vyrobíme nové
      a = new int[_max];
      n = 0;
    }
    max = _max;
  } 
  
  // pridanie prvku, iba ak máme miesto
  bool push(int x) {
    if (n == max) return false;
    a[n] = x;
    n++;
    return true;
  } 
  
  void discard() {          // odalokujeme pamäť
    delete[] a;
    a = nullptr;
  } 
};
\end{lstlisting}

Teraz je už náš zásobník presne, ako sme ho chceli mať\footnote{%
  v programe som použil novú skratku: ak v cykle \prg!for! chceš použiť
  premennú, ktorá bude viditeľná iba vo svete zloženého príkazu v cykle,
  môžeš je vyrobiť priamo v príkaze \prg!for(int i=0;...!}.
Vždy, keď budeš potrebovať zásobník, stačí urobiť copy\&paste definície
typu \vb{Stack} a dá sa pohodlne používať. Lenže robiť zakaždým copy\&paste
je otrava a navyše keď budeš mať 5--6 podobných tried, tvoj program sa stane
neprehľadný. Tento problém elegantne rieši {\em preprocesor}: prv než\indexItem{Prg}{preprocesor, direktíva \vb{\#include}}
sa súbor s programom dá kompilátoru, text súboru prejde jednoduchý prekladač.
Ten netuší nič o premenných, typoch, príkazoch atď, len spracováva text. 
Príkazy pre preprocesor sa spravidla začínajú znakom \prg!#! a jeden z nich je
\prg!#include!. Keď v programe napíšeš \prg!#include "velryba"! tak preprocesor
nájde súbor \vb{velryba} v aktuálnom adresári\footnote{Ak namiesto úvodzoviek použiješ
''zobáčiky'', bude sa hľadať nie v aktuálnom adresári, ale v jednom z default, ktoré
sú nastavené pri inštalácii. To je presne to, čo sa deje, keď napíšeš 
\prg!#include <iostream>!} a spraví copy\&paste jeho obsah na mieste \prg!#include!.
Preto ak si celú definíciu typu \vb{Stack} uložíš do suboru \vb{stack.h}, tak
kedykoľvek napíšeš \prg!#include "stack.h"!, na tom mieste sa nakopíruje 
definícia typu \vb{Stack}.


A na záver tejto kapitoly ešte drobnosť. Predstav si, že budeš mať súbor \vb{stack.h}
a spravíš inú triedu, \prg!struct Opica{Stack a,b;};!, ktorú si uložíš do súboru
\vb{opica.h}. Pretože si nechceš pamätať, že vždy pred použitím \prg!#include "opica.h"!
treba napísať \prg!#include "stack.h"!, napíšeš \prg!#include "stack.h"! ako prvý riadok v 
súbore \vb{opica.h}. Teraz keď napíšeš \prg!#include "opica.h"! tak preprocesor nájde súbor
\vb{opica.h} vloží ho do textu a pokračuje v spracovávní už aktualizovaného súboru. V ňom
nájde \prg!#include "stack.h"!, tak nájde súbor \vb{stack.h}, vloží ho tam a pokračuje
a celé to funguje. Problém ale môže nastať, ak neskôr chceš písať program, v ktorom chceš
použiť zásobník aj opicu. Napíšeš


\begin{lstlisting}[] 
#include "opica.h"
#include "stack.h"

int main() {
  // môj program s opicou a zásobníkom
}
\end{lstlisting}

Teraz sa ale stane to, že súbor \vb{stack.h} sa vloží do programu dvakrát (raz v 
hlavnom súbore, raz pri spracovaní súboru \vb{opica.h}) a kompilátor vyhlási chybu. Tento
problém sa spravidla rieši pomocou symbolov, ktoré vie preprocesor spracovávať.\indexItem{Prg}{direktívy \vb{\#define}, \vb{\#ifdef}}
Riadok \prg!#define pumpa 118! definuje v preprocesore symbol \vb{pumpa}: 
kdekoľvek sa v texte programu vyskytne reťazec  \vb{pumpa}, preprocesor ho nahradí 
číslom 118. Preprocesor má aj podmienky, ktoré testujú, či je alebo nie je
nejaký symbol definovaný. 
Zvyčajný spôsob, ako napísať súbor \vb{stack.h} je

\vbox{
\begin{lstlisting}[] 
#ifndef _subor_stack_h_uz_je_v_programe_
#define _subor_stack_h_uz_je_v_programe_

struct Stack{
  // všetko čo treba
};

#endif
\end{lstlisting}
}

Keď preprocesor načítava súbor \vb{stack.h} prvýkrát, symbol 
\prg!_subor_stack_h_uz_je_v_programe_! nie je definovaný, preto sa pokračuje\footnote{%
  všimni si, že podmienka je \vb{ifndef} ako \vb{if not defined}}, preto sa definuje
  a vloží sa definícia \vb{Stack}. 
  Pri každom ďalšom raze už symbol bude definovaný a definícia \vb{Stack} sa preskočí.

